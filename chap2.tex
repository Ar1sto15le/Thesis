\chapter{Gratitude}
	\section{Job}
	
Then Job answered the Lord: ``I know that thou canst do all things, and that no purpose of thine can be thwarted. `Who is this that hides counsel without knowledge?' Therefore I have uttered what I did not understand, things too wonderful for me, which I did not know. `Hear, and I will speak; I will question you, and you declare to me.' I had heard of thee by the hearing of the ear, but now my eye sees thee; therefore I despise myself, and repent in dust and ashes.\footnote{Edited by May, Herbert G., and Bruce Manning Metzger. ''The Book of Job.'' In The \emph{New Oxford Annotated Bible with the Apocrypha: Revised standard version, containing the second edition of the New Testament and an expanded edition of the Apocrypha}. New York: Oxford University Press, 1977. 613-655, Job 42:1-6} 


This is an illustration of the dialectical property of faith --- that faith must be received. Furthermore, it also illustrates the fact that faith operates on a different plane than philosophy. Job has had an intellectual understanding of God. This is the understanding that enables him to question God and approach him so fearlessly. With this intellectual understanding, Job is like a lawyer, defending himself against the charges of his companions and presenting his case to God. Yet, the very fact that Job says, 'my ears had heard of you but now my eyes have seen you', illustrates that his understanding of God was incomplete and, perhaps, naive.

''Hear, and I will speak;''' Job here is learning that faith and understanding of God is much more than simply an intellectual assent to God, His existence and His laws. Faith, ultimately, is not belief in an idea, instead faith is a belief in a person, in Christ. Furthermore, we learn that through his interaction with God, Job has come to understand God differently. He has come to see God, not just hear God. This change in perception, coupled with Job's repentance, illustrates that Job has come to a fuller appreciation of God. What else could explain Job's repentance after such an unusual response from God? 

```Who is this that hides counsel without knowledge?' Therefore I have uttered what I did not understand, things too wonderful for me, which I did not know.'' God does not attempt to explain the calamities that have befallen Job. God does not say, 'Job, I wanted to test your faith, so I let the devil do all of these terrible things to you.' Instead, the response is to question Job and his place in creation. ''Where were you when I laid the foundation of the earth? Tell me, if you have understanding. Who determined its measurements --- surely you know!''\footnote{ibid, Job 38:4-5.} It is certainly true that Job has been humbled by God's speech and certainly terrified. 

''Behold, I am of small account; what shall I answer thee? I lay my hand on my mouth. I have spoken once, and I will not answer; twice, but I will proceed no further.''\footnote{ibid, Job 40:3-5.} Job says in his first response. But has God simply beaten Job into submission? This does not appear to be the case because God continues to question Job, even after he has submitted to God his unworthiness. And after this second line of questioning, we see Job's response: ''I had heard of thee by the hearing of the ear, but now my eye sees thee;''\footnote{ibid, Job 42:5.} Not only is Job appreciating his place as God's creation, he is also growing in appreciation for God.


Then the Lord answered Job out of the whirlwind: ``Who is this that darkens counsel by words without knowledge? Gird up your loins like a man, I will question you, and you shall declare to me. ``Where were you when I laid the foundation of the earth? Tell me, if you have understanding. Who determined its measurements --- surely you know! Or who stretched the line upon it? On what were its bases sunk, or who laid its cornerstone, when the morning stars sang together, and all the sons of God shouted for joy?\footnote{ibid, Job 38:1-7.}


Given the overpowering nature of God's response, some interpreters have claimed that the ''divine speeches reveal God as a capricious, jealous tyrant who abuses his power.''\footnote{Schifferdecker, Kathryn. \emph{Out of the Whirlwind: Creation Theology in the Book of Job}. Cambridge, Mass.: Harvard Theological Studies, Harvard Divinity School :, 2008, 9.} Others suggest that the ''questions of the speeches are not designed to humiliate Job but to remind him of what he already knows. They enable him to realize anew that God establishes order in the cosmos... This order visible in the universe leads Job to trust God even when he does not understand why he suffers.''\footnote{ibid, 9.} While he may not be able to completely understand why he suffers, it is too simple to suggest that he cannot understand any aspect of his suffering. Furthermore, it is not a rationally satisfying answer to suggest that Job must simply trust in God.

Instead of a binary model in which Job either understands or he does not understand, we need a model that allows for growth. This model, then, is predicated upon the maturation of Job's faith. Throughout the course of the text of Job, a more compelling model would be to suggest that Job's faith develops from an immature, purely rational faith to a mature, relational faith in God. The starting point for this interpretation is the selection from Job 42:2-5 that is highlighted at the beginning of this section.

''Lord, I am ignorant of your ways and I was so foolish to trust myself instead of trusting in you,'' Job essentially says. This is easy to understand. Notice, however, that Job has undergone a change. He perceives God differently now: ''My ears had heard of you...'' By this we might understand that Job initially understood God in a rational way --- in the way that he must in order to maintain his innocence and defend himself against the accusations of his companions. ''But now my eyes see you.'' The Lord has made Himself present to Job in a deeper way. This reveals a relationship, perhaps, in which Job is justified in questioning God. Much like Adam and Eve in the Garden of Eden, Job has come into God's presence. This relationship is one between God the creator and Job the created. God's speech serves to humble Job, to keep him in line, this is certainly true. But Job's fault is not so much the questioning, as it is his doubting of God's goodness. ''Lord, I thought that you were all good and most just. Why are you allowing these things to happen to me, your faithful and innocent servant?''

And here God speaks so that Job can hear. ''Job, I am your creator! Do you not see the works I have created? Were you there when I laid the foundations of the earth? Why do you doubt my love for you? Do you not see the way that I care for the foolish ostrich that leaves its eggs on the ground unprotected? Do you not see the way that I care for my creation which I created out of love? Have I not brought you to life with my very own breath? Do not doubt my goodness, for I have created you and you are good.'' God, in his speech is reminding Job of the creator's love for his creation. He is also reminding Job that he is, in fact, a created being entirely dependent on the Lord's goodness.

	\section{Genesis}
Here we arrive at some of the fundamental questions and answers in Genesis: what does it mean to be a created being, entirely dependent on God's goodness? In this relationship, we transcend simply hearing God and we come to see God. God reveals his creative nature to us in Genesis: ''In the beginning, God created the heavens and the earth.'' The whole purpose of the first chapter of Genesis is to ''say one thing: God created the world''\footnote{XVI, Pope Benedict. Translated by Boniface Ramsey, O.P.. \emph{In the Beginning: A Catholic Understanding of the Story of Creation and the Fall}. Grand Rapids, Mich.: W.B. Eerdmans Pub. Co., 1995, 5.}, as the Benedict tells us in his homilies on creation, this one thing is that: 
The world is not, as people used to think then, a chaos of mutually opposed forces... Rather, all of this comes from one power, from God's eternal Reason, which became - in the Word - the power of creation. All of this comes from the same Word of God that we meet in the act of faith. ... In addition, the world was freed so that reason might lift itself up to God and so that human beings might approach this God fearlessly...\footnote{ibid, 5.}


Job is aware of the fact that God created the world and created him, too: ''Naked I came from my mother's womb, and naked shall I return; the Lord gave, and the Lord has taken away; blessed be the name of the Lord.''\footnote{Edited by May, Herbert G., and Bruce Manning Metzger. "The Book of Job." In The \emph{New Oxford Annotated Bible}, Job 1:21.} In other words, he has faith in God the creator. Through this faith, he meets God's eternal reason, the Word of God, and his reason is freed so that it might ''lift itself up to God'' and so that Job might ''approach this God fearlessly.'' And there is no doubting that Job approaches God fearlessly.

And what is the result of Job approaching God fearlessly? Job comes to see God with his eyes. If we understand Job's 'hearing God' as his comprehending God through reason, we must understand Job's seeing God with his eyes as a new way of comprehending God. God's speech points us to Genesis where we learn that man is made in the ''image and likeness'' of God. As Benedict tells us, human ''life stands under God's special protection, because each human being, however wretched or exalted he or she may be, however sick or suffering, however good-for-nothing or important, whether born or unborn, whether incurably ill or radiant with health - [because] each one bears God's breath in himself or herself, each one is God's image'' [emphasis mine].\footnote{Ramsey, O.P., Boniface, and Pope Benedict XVI. \emph{In the Beginning: A Catholic Understanding of the Story of Creation and the Fall}, 45.} This is the God of abundant and gratuitous love of Christian theology.

It is with this loving, creative God in mind that Zosima is able to interpret the story of Job the way that  he does, without being convinced of God in a philosophical sense. In the same way that active love can mysteriously convince us of God's existence, so too does it lead to gratitude. This gratitude and the resulting joy is a mystery of life.  It is the mystery of everyday life in which Job's suffering is gradually healed. The old grief and pain ``gradually passes into quiet, tender joy; instead of young, ebullient blood comes a mild, serene old age: I bless the sun's rising each day and my heart sings to it as before, but now I love its setting even more, its long slanting rays, and with them quiet, mild, tender memories, dear images form the whole of a long and blessed life --- and over all is God's truth, moving, reconciling, all-forgiving!''\footnote{Dostoyevsky, Fyodor. Translated by Richard Pevear, and Larissa Volokhonsky. \emph{The Brothers Karamazov: A Novel in Four Parts with Epilogue}. 292} When we look to Christ, who has been prefigured by Job, we realize that ``the passing earthly image and eternal truth here touched each other. In the face of earthly truth, the enacting of eternal truth is accomplished.''\footnote{ibid, 292.}  The truth in this phrase is that Christ became man and accepted death on a cross in order to redeem humanity.  In Christ, the earthly truth of our created nature and our dependence on God touches the eternal truth of God's love for us.  Christ is the personification of our relationship to God the creator.  Because Christ, the son of God suffered, it is not merely a `child-like' conviction that God exists and that he loves his creation. Instead, it becomes an unavoidable reality that the analytic mind cannot reduce to axioms without losing the meaning of the mystery. 

When a grateful mind receives this mystery, it comes to understand the story of Job in this light, as Zosima explains: ``Here the Creator, as in the first days of creation, crowning each day with praise: ``That which I have created is good, `` looks at Job and again praises his creation. And Job, praising God, does not only serve him, but will also serve his whole creation, from generation to generation and unto ages of ages,'' for to this he was destined. Lord, what a book, what lessons!''\footnote{ibid, 292.} Zosima's interpretation of Job is surprisingly similar to that of Aquinas in that there is a providential element to Job's life. In this case, in God's providence, Job is serving as a light to all those who are suffering. But where we see Zosima's gratitude, a component of his active love, come into play is in the fact that he understands Job to be praising God. In Zosima's eyes, it is considered a blessing for Job to be in the position to reaffirm the Lord's words by living faithfully to the Lord. In this way, Job is blessed to reply to the Lord, ``Yes, Lord. Your creation is good.''

\chapter{Proper Perspective}
	\section{Can it be resolved?}

When Zosima asks Ivan:''Can it be that you really hold this conviction about the consequences of the exhaustion of men's faith in the immortality of their souls?''\footnote{ibid, 70.} We get the sense that Zosima is peering into Ivan's soul and sensing the depths of Ivan's troubled heart. Ivan response is coolly logical: ''Yes, it was my contention. There is no virtue if there is no immortality.''\footnote{ibid, 70.} This is the extension of Ivan's idea that `if there is no God, everything is permitted.'\footnote{ibid, 82.} As we will learn later, Ivan's fundamental problem is that he has both a firm belief in God and that he abhors the suffering of the innocent. He cannot reconcile these two and it is this very paradox that wrestles inside his soul. And yet, Zosima responds to Ivan not with logic, but instead with his sense of Ivan's unhappiness --- his troubled heart.

''You are blessed if you believe so, or else most unhappy!''\footnote{ibid, 82.} Blessedness and unhappiness --- Zosima does not try and refute Ivan's logic. He accepts Ivan's argument and, instead, he turns to Ivan himself and attempts to speak to Ivan's real problem: his tormented heart. Zosima continues to address the real issue at hand: 
This idea [that there is no virtue if there is no immortality] is not yet resolved in your heart and torments it. ... For the time being you, too, are toying, out of despair, with your magazine articles and drawing-room discussions, without believing in your own dialectics and smirking at them with your heart aching inside you. . . The question is not resolved in you, and there lies your great grief, for it urgently demands resolution. . .\footnote{ibid, 70.}

Of course, the accuracy of Zosima's read into Ivan's soul can be questioned. Are we reading what we ourselves want to see? In other words, are we siding with Zosima solely because he is presented as an elder monk? Or has the character of Ivan, as presented by Dostoevsky, been accurately captured in Zosima's? Our answer to this question comes from Ivan himself. '''But can it be resolved in myself? Resolved in a positive way?' Ivan Fyodorovich continued asking strangely, still looking at the elder with a certain inexplicable smile.''\footnote{ibid, 70.} This strange smile comes after a quick admission of sincerity on Ivan's part --- '''But still, I wasn't quite joking either. . . ' Ivan Fyodorovich suddenly and strangely confessed---by the way, with a quick blush.''[ibid, 70.] In both of these statements and gestures (the strange confession and blush; the strange smile), we get the sense that Ivan has emerged from his wall of logic and insincerity. We get the sense that Zosima has connected with Ivan on a deep level --- that, perhaps, Zosima has spoken to Ivan's grief and despair.

''Even if it cannot be resolved in a positive way, it will never be resolved in the negative way either---you yourself know this property of your heart, and therein lies the whole of its torment. But thank the Creator that he has given you a lofty heart, capable of being tormented by such a torment....''\footnote{ibid, 70.} Again, we should be careful to not presume that Zosima's diagnosis is unassailable, but we should give them a lot of weight. We should appreciate the authority of the answer that Zosima has given Ivan. It is an answer that communicates directly with Ivan and causes him to break his detached and unemotional behavior. Immediately after these words, Ivan suddenly rises from his chair, receives his blessing and kisses the elder's hand. He then returns to his chair, firm and serious, and a solemn moment of silence overtakes the room. This answer apparently speaks to Ivan in a profound way.

What, then, is Zosima's answer to Ivan? Zosima's answer is that this question will never be entirely resolved in Ivan's mind. Even according to Zosima, this is a mystery that can never entirely be understood. In many ways, this is precisely the mystery that Job is dealing with. He undoubtedly believes in God, but he cannot understand why he is suffering. It is for this reason that Ivan's argument against accepting God's world is worth studying. In doing so, we might come to a better answer with regards to the question: How are we to understand God's answer to Job? How are we to deal with suffering? If Ivan's argument is worth analyzing, so too is the response put forth in The Brothers Karamazov. This answer, ultimately, is one in which an awareness of Christ's love overwhelms and overcomes our anger towards God, for Job and Ivan's question is not so much whether God exists --- this is a firmly held conviction of both --- as it is: ''How does human suffering affect our relationship with God and can we accept a broken world?''        

\section{Seeing Christ in all People: Interpersonal Faith and Resolution to Suffering}

''And yet happiness, happiness --- where is it? Who can call himself happy? ... let me tell you everything that I held back last time, that I did not dare to say, everything that I suffer with, and have for so long, so long! I am suffering, forgive me, I am suffering!''\footnote{ibid, 55.} Madame Khokhlavov, the lady of little faith, whose daughter has been ill and unable to walk, admits to Zosima. He responds, ''From what precisely... Lack of faith in God?''\footnote{ibid, 55.} Madame Khokhlavov responds: ''Oh, no, no I dare not even think of that, but the life after death --- it's such a riddle! ... I give you my greatest word that I am not speaking lightly now, that this thought about a future life after death troubles me to point of suffering, terror and fright....''\footnote{ibid, 55.} And here Zosima replies with great compassion: ''I believe completely in the genuineness of your anguish.''\footnote{ibid, 56.} The force behind this response is a great realism that knows true and great suffering. This is the great realism that often accompanies deep faith --- the sort of realism that often accompanies the `dark night of the soul' and the great 'perhaps' that gives the atheist troubles in Ratzinger's Introduction to Christianity.\footnote{Pope Benedict XVI quotes this story from M. Buber's Werke, vol 3 (Munich and Heidelberg, 1963), p. 348: ``An adherent of the Enlightenment [writes Buber], a very learned man, who had heard of the Rabbi of Berditchev, paid a visit to him in order to argue, as was his custom, with him, too, and to shatter his old-fashioned proofs of the truth of his faith. When he entered the Rabbi's room, he found him walking up and down with a book in his hand, rapt in thought. The Rabbi paid no attention to the new arrival. Suddenly he stopped, looked at him fleetingly, and said, ``But perhaps it is true after all.'' The scholar tried in vain to collect himself---his knees trembled, so terrible was the Rabbi to behold and so terrible his simple utterance to hear. But Rabbi Levi Yitschak now turned to face him and spoke quite calmly: ``My son, the great scholars of the Torah with whom you have argued wasted their words on you; as you departed you laughed at them. They were unable to lay God and his Kingdom on the table before you, and neither can I. But think, my son, perhaps it is true.'' The exponent of the Enlightenment opposed him with all his strength; but this terrible ``perhaps'' that echoed back at him time after time broke his resistance.''}

``Oh, how grateful I am to you! You see, I close my eyes and think: if everyone has faith, where does it come from? And then they say that it all came orginally from fear of the awesome phenomena of nature, and that there is nothing to it at all. What? I think, all my life I've believed, then I die, and suddenly there's nothing, and only `burdock will grown on my grave,' as I read in one writer? Though I believed only when I was a little child, mechanically, without thinking about anything... How, how can it be proved? I've come now to throw myself at your feet [Zosima's feet] and ask you about it. If I miss this chance, too, then surely no one will answer me for the rest of my life. ... It's devastating, devastating!''
``No doubt it is devastating. One cannot prove anything here, but it is possible to be convinced.''
``How? By what?''
``By the experience of active love. Try to love your neighbors actively and tirelessly. The more you succeed in loving, the more you'll be convinced of the existence of God and the immortality of your soul. And if you reach complete selflessness in the love of your neighbor, then undoubtedly you will believe, and no doubt will even be able to enter your soul. This has been tested. It is certain.''\footnote{Dostoyevsky, Fyodor. Translated by Richard Pevear, and Larissa Volokhonsky. \emph{The Brothers Karamazov: A Novel in Four Parts with Epilogue}, 56.}

Working from our analysis --- the first thing that becomes clear is that we cannot 'prove active love' in the same way that we can prove a philosophical argument. One of the very premises of active love is that the convincing nature of practicing active love does not come from intellectual assent, it comes through the very difficult work of embracing the reality of the human condition, the finitude of life and the darkness that man is capable of doing. It comes in learning to love man and life in all of its aspects, unconditionally. This sort of love encompasses all aspects of the human --- the physical, the intellectual and the spiritual. Just as Christ assumed --- and embraced --- all aspects of the human condition, so too does active love operate (like faith) on all levels of human nature. This being the case, our 'proof' of active love is going to be one in which all these aspects of the human condition are addressed. 

One form that is able to speak on all of these levels is the novel --- and this is precisely what The Brothers Karamazov presents us with. In this novel, we not only hear the intellectual argument for active love, but we also see how it affects the human person through the lives of the characters and we ourselves are touched by the lives of the characters. Thus, in 'proving' active love, we follow Zosima as he explains it and teaches it to all who will listen. And in following Zosima, we can see for ourselves whether or not we are 'convinced'.

Madame Khokhlavov explains to Zosima that she has a great love for humanity: ''You see, I love mankind so much that --- would you believe it? --- I sometimes dream of giving up all, all I have, of leaving Lise and going to become a sister of mercy. I close my eyes, I think and dream, and in such moments I feel an invincible strength in myself.''\footnote{ibid, 56.} Here Zosima says, not without a bit of humor, ''It's already a great deal and very well for you that you dream of that in your mind and not of something else. Once in a while, by chance, you may really do some good deed.''\footnote{ibid, 56.}

Here the madame cuts to the heart of her problem --- ''Yes, but could I survive such a life for long?'' She goes on to explain that one of her primary motivators for her love of mankind is immediate gratitude --- ''if there's anything that would immediately cool my 'active' love for mankind, that one thing is ingratitude. In short, I work for pay and demand my pay at once, that is, praise and a return of love for my love. Otherwise I'm unable to love anyone!''\footnote{ibid, 57.} Zosima continues to explain to Madame Khokhlavov that she will make no progress in convincing herself if she merely speaks sincerely in order to be praised for her sincerity. Zosima is so astute in his analysis that she cries out, ''You've brought me back to myself, you've caught me out and explained me to myself!''\footnote{ibid, 58.}

And here Zosima spells out the difficulty of active love --- it is not naive, nor does it discount human suffering. Instead, it embraces this suffering and accepts reality as it is, without ever losing hope: 
''If you do not attain happiness, always remember that you are on a good path, and try not to leave it. Above all, avoid lies, all lies, especially the lie to yourself.\footnote{Note this realism --- the very realism of Ivan Karamazov!} ... Never be frightened at your own faintheartedness in attaining love, and meanwhile do not even be very frightened by your own bad acts. I am sorry that I cannot say anything more comforting, for active love is a harsh and fearful thing compared with love in dreams. Love in dreams thirsts for immediate action, quickly performed, and with everyone watching. Indeed, it will go as far as the giving even of one's life, provided it does not take long but is soon over, as on stage, and everyone is looking on and praising. Whereas active love is labor and perseverance, and for some people, perhaps, a whole science.''\footnote{ibid, 58.}
 
Here the realism of active love becomes clear. Active love is not naive, it does not seek immediate action (like Alyosha), it is labor and perseverance. Active love is a way of life that is realistic in its very nature --- so much so that it even takes into account the apparent injustices of life, those moments in which acts of love are not returned and in which we cannot make sense of reality. 

''But I predict that even in that very moment when you see with horror that despite all your efforts, you not only have not come nearer your goal but seem to have gotten farther from it, at that very moment --- I predict this to you --- you will suddenly reach your goal and will clearly behold over you the wonder-working power of the Lord, who all the while has been loving you, and all the while has been mysteriously guiding you.''\footnote{ibid, 58.}

This statement sounds surprisingly like a description of the events that befall Job. Despite all of Job's efforts, at the very moment in which he feels that he cannot continue on, his faith is restored by an encounter with God himself, who reveals His love for Job through creation and through the restoration of Job's livelihood. But, for active love, an argument that seemingly embraces such realism, to argue that God's love will be revealed at that very moment ... this just seems to be an incoherent argument. Logically, this argument depends on the premise of actively loving one's neighbor (presupposing faith in God and His goodness), which will then lead to knowing God and His goodness. It sounds as if this is a self-fulfilling prophecy or simply seeing what one wants to see. But at this point we must remember: Ivan and Job do not doubt God (nor His goodness). Rather, they are both simply not fully convinced. This starting premise is fulfilled for both Ivan and Job and the 'self-fulfillingness' of active love is lost through the very trials and tribulations of active love itself, purged by them, so to speak.

\chapter{Accepting Reality}
	\section{Ivan Karamazov's Rebellion}
In this section, then, we will analyze Ivan's argument as it is put forth Dostoevsky's Brothers Karamazov, Book 5, Part 4 (Rebellion) and Part 5 (The Grand Inquisitor) after a short introduction to Rebellion. Ivan and Alyosha are talking over tea when Ivan explains his great desire to live and his love for life to Alyosha: 
''If I did not believe in life, if I were to lose faith in the woman I love, if I were to lose faith in the order of things, even if I were to become convinced on the contrary, that everything is a disorderly, damned, and perhaps devilish chaos, if I were struck even by all the horrors of human disillusionment---still I would want to live...''\footnote{ibid, 230.} 
This love for life is deeply rooted in Ivan and it is clearly associated in the text to the sensual nature of the Karamazov men --- the 'sensualists'. This visceral desire for life is dramatically opposed to Ivan's 'Euclidean mind'.\footnote{ibid, 235.} Ivan is even willing to admit this: ''I want to live, and I do live, even if it be against logic.''\footnote{ibid, 230.} Thus, in this introduction we already see an acknowledgement on Ivan's part that he is not merely a logical being. He is, instead, a human being, with the human desire to live. This point is important to note as it relates to the the discussion of the limitations of reason, the interplay between faith and reason, and the realm of knowledge that faith deals in. In this particular case, Ivan has admitted that his reason cannot explain to him why he desires to live, nor can it explain his love for ''the sticky little leaves that come out in the spring'', nor his loved ones, nor his sensitivity towards the suffering innocent. This sensitivity towards the suffering innocent might, in fact, point back to God himself.

The next important point to notice in this section is that Ivan admits a number of important beliefs: a strong belief in God, a belief in our limited view of God and His intentions, and that it is ''this world of God's, created by God,'' that he does not accept and cannot agree with.\footnote{ibid, 235.}
 
''And so, I accept God, not only willingly, but moreover I also accept his wisdom and his purpose, which are completely unknown to us; I believe in order, in the meaning of life, I believe in eternal harmony, in which we are all supposed to merge, I believe in the Word for whom the universe is yearning, and who himself was 'with God,' who himself is God, and so on....''\footnote{ibid, 235.} 

So Ivan accepts God willingly... and yet he still willingly refuses God's world. This refusal even comes in light of his ''childlike conviction that the sufferings will be healed and smoothed over, that the whole offensive comedy of human contradictions will disappear like a pitiful mirage, a vile concoction of man's Euclidean mind ... it will suffice not only to make forgiveness possible, but also to justify everything that has happened with men.''\footnote{ibid, 235.} In other words, Ivan has articulated a fairly complex framework for suffering. This is a framework in which suffering can be justified and one that even articulates and accepts the mystery of how suffering will be justified.

Notice that Ivan calls this particular belief --- that suffering will be justified --- a 'childlike conviction'. This is a conviction, perhaps, that is similar to his desire for life which cannot be explained, but in a very real sense it is different from a desire for life. It is a conviction --- or as another translation has it, Ivan says: ''I believe like a child that suffering will be healed...'' --- and thus, it certainly has a different sense than desire. A desire implies an urge and propels Ivan towards action, to live life. A conviction or a belief implies something more akin to knowledge or wisdom --- something that is known not particularly through reason, but through faith. A conviction or a belief, particularly a childlike belief, here makes us wonder what a mature belief would be like? Furthermore, what is child-like about the belief? The object of the belief --- that everything will be healed --- or the nature of the belief --- an childlike belief as opposed to a mature belief?

If this is meant to describe that nature of the belief, perhaps Ivan considers to be childlike because it is an expression of faith, as opposed to a logical deduction or a reasonable induction. The evidence against this, however, is Ivan's own recognition of sincere, illogical desires within his own heart. But this evidence is not conclusive because Ivan has made a distinction between desire and belief. And yet, we grow in assurance that this is perhaps what Ivan means because he wants his dilemma to be solved... ''But can it be resolved in myself? Resolved in a positive way?''\footnote{ibid, 68.} he says to Zosima. This question itself shows us that Ivan is looking to solve the paradox himself, within himself and perhaps by himself. It also shows us that Ivan has a desire to resolve it positively --- that is, Ivan has a desire to find a way to make God's existence and human suffering work together. And here we remember Zosima's answer that it cannot be resolved. This particular kind of belief --- a child-like belief --- is paralleled in Alyosha's somewhat naive nature and his own somewhat naive belief as he is seen to have his very own crisis of faith after Zosima's death.

	\section{Alyosha's Lapse of Faith}
	
In fact, Rowan Williams ties Alyosha together with Ivan in that he says both Alyosha and Ivan say: ''God exists but I am not sure whether I believe in him...''\footnote{Williams, Rowan. \emph{Dostoevsky: Language, Faith, and Fiction}, viii.} As we see in the novel, Alyosha's faith is transformed from its naive state to a more mature faith throughout the course of the novel. As Williams puts it: ''Alyosha has sensed a divine abundance and liberty that exceeds human standards of success and failure; his belief has been transformed---but not in the sense that he has become convinced of God's existence. It is rather that he now sees clearly what might be involved in a life that would merit being called a life of faith.''\footnote{ibid, viii.} Thus, in seeing the parallel with Alyosha's own crisis of faith, we can begin to see an answer to Ivan's problem --- we begin to see what a mature faith might look like. This is an idea that we will explore after we more fully layout Ivan's dilemma as expressed in his Rebellion and in the Grand Inquisitor. But, for the sake of thinking while writing, we can express briefly (and primarily in William's thoughts and words) what this mature faith might look like: ''What [Dostoevsky] does in Karamazov is not to demonstrate that it is possible to imagine a life so integrated and transparent that the credibility of faith becomes unassailable;'' --- note that this is precisely what is envisioned in Ivan's child-like faith, a vision of a life that integrates the paradox of suffering in God's world in such a way that it can transparently be understood and framed neatly --- ''it is simply to show that faith moves and adapts, matures and reshapes itself, not by adjusting its doctrinal content ... but by relentless stripping away from faith of egotistical or triumphalistic expectations. The credibility of faith is in its freedom to let itself be judged and to grow.''\footnote{ibid, x.} In other words, in Williams' reading of the novel, Zosima's words ring true --- there will never be a resolution to Ivan's dilemma --- but, instead, there will be a continual growth of faith. This growth in faith will have the hallmark of humility --- humility that is born in the appreciation of the fact that man is a created being and that he is loved by God --- the lessons that God of Genesis to which God points Job. ''Were you there when I laid the foundations of the earth?''

\section{Guilty Before All --- Accepting Man's Fallen Nature}
In continuing our analysis then, we see that both Williams and Zosima agree: Ivan's conviction is also childlike in its object of an integrated and transparent faith, i.e. Ivan's logical framework for the justification of suffering. This being the case, we will now continue into one of Ivan's primary arguments: the impossibility of loving one's neighbors and the resulting evils of the world, particularly in the suffering of innocents. From the very start, notice that this is argument is directed against one of Zosima's primary principles of active love. In Zosima's mind, it is through actively loving one's neighbor that we will grow in humility and in faith. Through active love, Zosima teaches, faith will become mature and the question of suffering will quietly fade away. Ivan's argument is neatly summarized in the statement: ''Christ's love for people is in its kind a miracle impossible on earth.''\footnote{Dostoyevsky, Fyodor. Translated by Richard Pevear, and Larissa Volokhonsky. \emph{The Brothers Karamazov: A Novel in Four Parts with Epilogue}, 237.} He substantiates this by listing example after example of atrocities --- Turks taking delight in torturing children and cutting them from their mother's wombs is a particularly vile example. ''I'm not talking about the suffering of grown-ups, they ate the apple and to hell with them, let the devil take them all, but these little ones!''\footnote{ibid, 242.} What Ivan cannot accept is precisely the sin-damaged world that Pope Emeritus Benedict discusses in his homilies on creation and he even brings out the despair and hatred within Alyosha: ''Shoot him!'' Alyosha says and Ivan yells, ''Bravo!'' --- ''If even you say so, then . . . A fine monk you are! See what a little devil is sitting in your heart, Alyosha Karamazov!''\footnote{ibid, 243.} It is not hard to see Ivan's point. Christ-like love upon this earth is very difficult indeed... but miraculous?

The idea of a miracle is one that also deserves special attention in The Brothers Karamazov. Remember the connection between Ivan and Alyosha, that at similar points in their lives, they are struggling with a somewhat naive faith --- or 'childlike conviction' --- as Ivan says. In the very beginning of the novel, Alyosha is described in a particularly interesting light --- especially in light of Ivan's statement that ''Christ's love for people is in its kind a miracle impossible on this earth.'' Alyosha is described by the narrator as being ''even more of a realist than the rest of us.''\footnote{ibid, 25.} Furthermore, the narrator describes that a ''true realist, if he is not a believer, will always find in himself the strength and ability not to believe in miracles as well, and if a miracle stands before him as an irrefutable fact, he will sooner doubt his own senses than admit the fact.''\footnote{ibid, 25.} And it is in this statement that the narrator has connected us once again with Ivan and brought us to a better understanding of Ivan's true problem: Ivan is struggling with the belief that Christ's love is possible on earth. This is certainly a key in understanding his argument as it sheds light on what is particularly giving Ivan trouble. And it is Christ and the miracle of the Incarnation to which Alyosha explicitly points Ivan, ''you asked just now if there is in the whole world a being who could and would have the right to forgive. But there is such a being, and he can forgive everything, forgive all and for all, because he himself gave his innocent blood for all and for everything. You've forgotten about him, but it is on him that the structure is being built [as opposed to on the soul of an innocent child]...''\footnote{ibid, 246.}
 
In this way, Alyosha responds to Ivan's argument again with a mystery, with an answer similar to Zosima's earlier response to Ivan, but subtly different. Zosima said simply that the paradox would never be resolved within Ivan himself. Within. Alyosha is pointing to a mystery outside of Ivan, the mystery of Christ's incarnation and death, in which the problem of suffering is solved --- but in a non-transparent way. We cannot fully understand the miraculous nature of Christ's death. And it is this particular answer that Ivan that Ivan still cannot even accept.

\section{Christ Crucified --- Ivan's Grand Inquisitor}

''My poem is called 'The Grand Inquisitor'; it's a ridiculous thing, but I want to tell it to you.''\footnote{ibid, 246.} And so begins Ivan's Grand Inquisitor. This prose poem of Ivan's reveals a couple of things about Ivan's argument. Firstly, it reveals that Ivan's primary problem with the world that God has created is the freedom that has been given to man. This freedom is what allows for turkish soldiers to cut babies from their mothers' wombs. This freedom is the necessary condition for evil. Coupling this freedom with the impossibility, in Ivan's eyes at least, of humans to love as Christ and we begin to see the dystopia that has evolved in Ivan's mind. How can anyone possibly continue living in such a world where true love for fellow men is an impossible 'miracle'? This world is inherently a world of suffering. And in suffering, of the innocent in particular, ''lies the secret of Ivan's atheism, the basis for his 'rebellion'.''\footnote{Jens, Walter, and Hans Küng. "Religion in the Controversy Over the End of Religion." In \emph{Literature and Religion: Pascal, Gryphius, Lessing, Hölderlin, Novalis, Kierkegaard, Dostoyevsky, Kafka}. New York: Paragon House, 1991. 227-242, 242.}

''In suffering, especially in that of the innocent, man comes up against his extreme limit, comes to the decisive question of his identity, of the sense and nonsense of his living and dying, indeed, of reality pure and simple. Given the overwhelming reality of suffering the life and history of humanity does the suffering, doubting, despairing person really have any other choice? What alternative is there to the rebellion of an Ivan Karamazov against this world of God that he finds so unacceptable...''\footnote{ibid, 234.} Ivan, himself, is able to recognize an alternative --- although he cannot fully convince himself to believe it. This is a second aspect that becomes apparent in his poem. Ivan has an intuition about an alternative to a world of suffering: a world in which men love as Christ do. This is a world of Zosima's 'active love' --- as spelled out in the 'Lady of Little Faith'.

Here, again, Kung is particularly helpful: ''But this world of Ivan, so subtly portrayed, is now contrasted, in serenity and great inner freedom, with an alternative world that has its own plausibility. While Ivan primarily talks, Alyosha acts. Dostoyevsky was convinced that on the ultimate theological issues rational argumentation was impotent.''\footnote{ibid, 236.} \footnote{Note: this is a theme that has been explored in some detail with our discussion of the importance of faith operating on a separate plane than reason. This is a theme that we will further explore in the form of narrative in Eleonore Stump's work on Wandering in Darkness: Narrative and the Problem of Suffering.} Thus, the very fact that Ivan has Christ kissing the Grand Inquisitor is a certain admission of Ivan's that this argument does not exist purely within the realm of reason. This is corroborated by both Ivan's illogical desire to live as well he his irrational rebellion. Remember that his rebellion is irrational in this sense: despite being able to given an account of how it could be possible for suffering to be permitted and justified, and even having a 'childlike conviction', Ivan still rebels.

Bringing this back to the Grand Inquisitor, we have learned that Ivan does not believe that humans can love as Christ and that freedom is the condition of God's world that enables men bring about suffering of the innocent.\footnote{Note: Zosima's interaction with the Lady of Little Faith is a direct counter to Ivan's main argument. Zosima believes that it is only through loving our neighbors as Christ loved us that we can ultimately find peace and be convinced of God's presence.} Ivan's story begins with Christ coming to earth once more as a man in the town of Seville, where he alludes to a burning during the Inquisition in which ''had burned almost a hundred heretics at once ad majorem gloriam Dei.''\footnote{Dostoyevsky, Fyodor. Translated by Richard Pevear, and Larissa Volokhonsky. \emph{The Brothers Karamazov: A Novel in Four Parts with Epilogue}, 243.} He describes how Christ was both unobserved and yet ''every one recognized Him.'' Here it is worth noting that Ivan thinks these are some of the best lines in the poem: ''This could be one of the best passages in the poem, I mean, why it is exactly that they recognize him. People are drawn to him by an invincible force, they flock to him, surround him, follow him. He passes silently among them with a quiet smile of infinite compassion.''\footnote{ibid, 249.} Ivan continues to describe how Christ performs a couple of miracles --- healing a blind man and raising a child from the dead. While this Christ is certainly similar to the one that we see in the Gospels, it is perhaps important to note three things: Ivan's Christ is incarnated as an adult man, he is described as if everyone suddenly recognizes that he is Christ and people flock to him as if they cannot help but believe in him --- the `invincible force'.

This is in contrast to Christ almost being killed by the people of his home town, the people who had grown up with Christ, in Luke 4:14-30. In other words, the people who had watched Christ grow up right in front of their eyes did not recognize him as the Son of God. ''Who do men say that the Son of man is?'' And they said, ``Some say John the Baptist, others say Eli'jah, and others Jeremiah or one of the prophets. He said to them, ``But who do you say that I am?'' Simon Peter replied, ``You are the Christ, the Son of the living God... For flesh and blood has not revealed this to you, but my Father who is in heaven.''\footnote{Edited by May, Herbert G., and Bruce Manning Metzger. \emph{The New Oxford Annotated Bible}, Matthew 16:13-18.} Only Simon Peter knew the answer to this question --- and this knowledge was not revealed to him by any flesh and blood, but by God himself. This is quite the contrast to Ivan's account of Christ.

Of course, Ivan's description of Christ is fictional and it occurs 15 centuries after Christ first lived. This might account for the fact that people recognize him. But, on the other hand, we still see in Scripture that Christ was not universally recognized, nor was he universally accepted as he appears to be in this poem of Ivan's. If Christ were to come down to earth, is this what Ivan would expect to happen? This image that Ivan has is perhaps in-line with his 'childlike-conviction' that we discussed earlier. It is an image that might be considered immature in this sense: Christ became fully man. Christ is not super-human, he is human. This means that Christ suffered, that he cried and that he confronted the suffering of those around him continually throughout his life. He brought people back from the dead, but he also cried at their deaths. Furthermore, in redeeming man through his death on the cross, Christ did not negate the fact that human beings suffer. In fact, he reaffirmed the harsh reality of the world in his death.

The Grand Inquisitor takes Christ captive and questions him --- although he does not permit Christ to speak. In his questioning, he explains that through the Inquisition he (and the Roman Catholic church) has ``finally overcome freedom'' and they ``have done so in order to make people happy.''\footnote{Dostoyevsky, Fyodor. Translated by Richard Pevear, and Larissa Volokhonsky. \emph{The Brothers Karamazov: A Novel in Four Parts with Epilogue}, 251.} The Grand Inquisitor continues to explain that God created man a rebel, that he knew man was a rebel from the very beginning, and that God knew that there would be only way way of arranging human happiness: through the fear and piety that is fostered by the Grand Inquisitor's church. ```Man was made a rebel; can rebels be happy? ... you rejected the only way of arranging for human happiness, but fortunately, on your departure, you handed the work over to us.''\footnote{ibid, 251.} Here again we see two important ideas relating to Ivan's conception of freedom and human nature: freedom is at odds with men's happiness and that man 'was created a rebel'. If man was created a rebel, it follows logically that he is going to abuse his freedom and from this view a natural skepticism towards man's ability to love in a Christlike way emerges. We are beginning to see, then, that Ivan's rebellion is partially a product of his cynicism.

This cynicism manifests itself particularly in the Grand Inquisitors statement: ''But you did not know that as soon as man rejects miracles, he will at once reject God as well, for man seeks not so much God as miracles.''\footnote{ibid, 255.} It is this very miraculousness that Ivan paints Christ in, Christ with light and power radiating from his eyes, and perhaps this fixation on the miraculous is another key to Ivan's problem. If he is correct in his view, that ''man seeks not so much God as miracles'' there is a certain sense in which God can be proven to men and Christian morality can be instilled in them. In fact, this is exactly the sort of society that the Grand Inquisitor is seeking to uphold --- one in which the Church upholds society and its laws by upholding this miraculousness in a totalitarian sort of way.

This totalitarian society is all based on the mystery that only a few will be able to be Christlike --- that all the rest of men will not be able to live up to the `impossible' standards of Christianity. This sentiment becomes clear in this statement: ''But remember that there were only several thousand of them, and they were gods. What of the rest? Is it the fault of the rest of feeble mankind that they could not endure what the mighty endured?''\footnote{ibid, 256.} In other words, the mystery for Ivan is that man was created as a rebel and, as such, man cannot become Christlike. This is the source of Ivan's cynicism.

\section{Active Love: A Response to Ivan Karamazov}

In our analysis of Ivan's Grand Inquisitor, one of the predominant themes that became clear was that Ivan's primary problem is that he believes it is impossible for man to love as Christ did. It is this premise that leads Ivan to conclude that only by miracles and authority can men, who are born as rebels, be tamed. Furthermore, Ivan's anger at God stems from the fact that Ivan believes that the only possible outcome of giving man free will was that men would become ruthless. Ivan has a lot of evidence to support his claim that men cannot love as Christ did. This is the evidence he presents in the Rebellion. Although Ivan has evidence to the contrary, the text of the Brothers Karamazov presents an argument against Ivan's cynicism, his rebellion and his despair.

This argument consists primarily of Zosima's principle of active love. Active love encompasses a number of principles that we have seen throughout the texts of Job and Genesis. In particular, it is characterized by: mature faith, proper perspective, gratitude, humility and action. I highlight the active nature of 'active love' to elucidate the point that the answer to the problem of suffering is not purely an intellectual endeavor. It is, instead, a way of living that brings together all aspects of the human being --- the physical, the intellectual and the spiritual. After all, in the Incarnation, Christ assumes the full nature of man --- the physical, the intellectual and the spiritual. Christ suffered pain, felt compassion and dealt with the finiteness of humanity. Understanding this is one of the keys to understanding active love and it is this very principle that is embodied in a number of points throughout the book. The decay of Zosima's body, for example, illustrates this very finitude that Christ assumed in becoming fully man. Furthermore, Alyosha's initial despair over Zosima's decay illustrates the immature faith that Ivan is also subject to --- this is a faith that cannot accept the reality of human nature and its finitude.

Ivan and Alyosha's faith, however, is not so simple that it could be called naive. Although Ivan has a 'childlike-conviction' that all suffering will be justified and although his Grand Inquisitor points to miracles making men obedient, we still get the sense that a miracle would not convince Ivan himself to embrace the reality of human nature --- this is clear in the sarcastic and insincere tone in which Ivan tells his story. ''But it's nonsense, Alyosha, it's just the muddled poem of a muddled student who never wrote two lines of verse. Why are you taking it so seriously?''\footnote{ibid, 262.} And, again, with Alyosha we are told explicitly that he is not naive --- rather, it seems that he has not fully embraced the reality of human nature. ''I will say this much,'' says the narrator, ''it was not a matter of miracles. It was not an expectation of miracles, frivolous in its impatience. Alyosha did not need miracles then for the triumph of certain convictions...''\footnote{ibid, 339.} The narrator tells us that Alyosha was seeking ''only a 'higher justice''' and that this 'higher justice' had been violated in the decay of Zosima's body --- and ''it was this that wounded his heart so cruelly and suddenly.''\footnote{ibid, 339.} Perhaps what we are seeing in Ivan and Alyosha's faith is not so much a naive faith seeking miracles --- Alyosha, after all, as he is described by the narrator was ''even more of a realist than the rest of us.''\footnote{ibid, 25.} It is certainly within the text to say that Ivan is a realist, himself. Ivan does not deceive himself and his realism has manifested itself in his cynicism and despair. And this is why miracles are not what will bring these two Karamazov brothers to faith --- for it ''is not miracles that bring a realist to faith.''\footnote{ibid, 25.} Instead, then, perhaps what we are seeing in Ivan and Alyosha is an undeveloped faith coming to grips with its realism. They are both, in their own ways, learning to embrace the reality of the human condition. What else does Christ's kiss of the Grand Inquisitor or Alyosha kissing the ground (the earth --- the very substance from which God formed man) symbolize than embracing man's nature and finitude?

Ivan and Alyosha's faith, however, is not so simple that it could be called naive. Although Ivan has a 'childlike-conviction' that all suffering will be justified and although his Grand Inquisitor points to miracles making men obedient, we still get the sense that a miracle would not convince Ivan himself to embrace the reality of human nature --- this is clear in the sarcastic and insincere tone in which Ivan tells his story. ''But it's nonsense, Alyosha, it's just the muddled poem of a muddled student who never wrote two lines of verse. Why are you taking it so seriously?''\footnote{ibid, 262.} And, again, with Alyosha we are told explicitly that he is not naive --- rather, it seems that he has not fully embraced the reality of human nature. ''I will say this much,'' says the narrator, ''it was not a matter of miracles. It was not an expectation of miracles, frivolous in its impatience. Alyosha did not need miracles then for the triumph of certain convictions...''\footnote{ibid, 339.} The narrator tells us that Alyosha was seeking ''only a 'higher justice''' and that this 'higher justice' had been violated in the decay of Zosima's body --- and ''it was this that wounded his heart so cruelly and suddenly.''\footnote{ibid, 339.} Perhaps what we are seeing in Ivan and Alyosha's faith is not so much a naive faith seeking miracles --- Alyosha, after all, as he is described by the narrator was ''even more of a realist than the rest of us.''\footnote{ibid, 25.} It is certainly within the text to say that Ivan is a realist, himself. Ivan does not deceive himself and his realism has manifested itself in his cynicism and despair. And this is why miracles are not what will bring these two Karamazov brothers to faith --- for it ''is not miracles that bring a realist to faith.''\footnote{ibid, 25.} Instead, then, perhaps what we are seeing in Ivan and Alyosha is an undeveloped faith coming to grips with its realism. They are both, in their own ways, learning to embrace the reality of the human condition. What else does Christ's kiss of the Grand Inquisitor or Alyosha kissing the ground (the earth --- the very substance from which God formed man) symbolize than embracing man's nature and finitude?

Here we can draw parallels with Job. Job was not seeking miracles, so much as he was seeking a way to comprehend what was happening to him and relief from his suffering. ''I cry to thee and thou dost not answer me; I stand, and thou dost not heed me. Thou hast turned cruel to me; ... Yea, I know that thou wilt bring me to death, and to the house appointed for all living.''\footnote{Edited by May, Herbert G., and Bruce Manning Metzger. "The Book of Job." In The \emph{New Oxford Annotated Bible with the Apocrypha: Revised standard version}, Job 30: 20-25.} Job's speech ends with a great bevy of oratorical hypothetical questions --- ``if I have raised my hand against the fatherless because I saw help in the gate; then let my shoulder blade fall from my shoulder, and let my arm b broken from its socket.''\footnote{ibid, Job 31: 21-22.} Job, too, is a realist who is approaching the situation from a perspective of faith in God, just like Ivan and Alyosha. Job does not deny --- and, in fact, he embraces --- the punishment that he would deserve if he had been guilty of these sins. But, like Ivan and Alyosha, he has come to his wits end in trying to understand what is happening to him. ''For I was in terror of calamity from God, and I could not have faced his majesty.''\footnote{ibid, Job 31:23.} Notice, Job once was at a point where he could not question God, but he is now beyond that point. His suffering has brought his faith to a breaking point --- his suffering has brought him to the point of speaking with God directly. 

Zosima's response to Ivan's dilemma is clearest in his interaction with the mysterious visitor. As it turns out, this visitor of Zosima's had committed the murder of a woman with whom he was in love, who had rebuked him. The mysterious visitor escaped the crime without being caught, but the crime continually haunted him. Despite his greatest efforts in philanthropy and in raising his own family, he has not been able to avoid the guilt of this crime. This mysterious visitor is drawn to Zosima, who convinces him that he must confess his crime in order to be cleansed of the guilt. In this interaction, we see one of the greatest examples of active love embracing the darkest depravity that humanity knows: murder. There are a few ideas in this section that become clear, with the primary one being this: human beings can, in fact, love as Christ did (contra Ivan); but they must gradually be remade anew in order to love fully as Christ did, in giving of himself completely. This is an arduous process of purification of the soul --- the process that we see in the Lay of Little Faith --- in which one cannot lie to themselves. It is predicated upon loving one's neighbor as oneself. ``Know, then, that this dream, as you call it [Kingdom of Heaven on earth], will undoubtedly come true, believe it, though not now, for every action has its law. This is a matter of the soul, a psychological matter In order to make the world over anew, people themselves must turn onto a different path psychically. Until one has indeed become the brother of all, there will be no brotherhood.''\footnote{ibid, 303.}

But, in order for this process to happen, we must be entirely remade through the process of active love and grace, which is often revealed during the darkest trials of faith as we have seen in Zosima's explanation to Madame Kholkhov and Alyosha's faith being restored by the smallest act of kindness from the `sinner' Grushenka. This recreation is best exemplified  in John 12:24: ``Verily, verily, I say unto you, except a corn of wheat fall into the ground and die, it abideth alone: but if it die, it bringeth forth much fruit.''\footnote{ibid, 309.} This is Christ's prophecy of his death. It is also the epithet of The Brothers Karamazov. This verse alone embodies the harsh reality: corn must die in order to bring forth life; but it also brings forth the greatest hope in the form of `much fruit'.

\chapter{Summary and Conclusions}
The thesis that has been argued for in this paper proposes that the problem of suffering, an inherently interpersonal problem, must be solved in an all-inclusive way of life that is much more than a `philosophy of life'. The exact way of life that is proposed is Zosima's active love that is displayed in Dostoevsky's Brothers Karamazov. Active love is a way of being that has three particular characteristics: it is rooted in gratitude and a proper understanding of man's relationship with God (i); it is characterized by proper perspective (ii) in two ways: it recognizes that suffering is not always going to be understandable (iii) and it focuses on particulars instead of broad and complicated generalities (it deals with a particular person instead of `humanity') (iv); it accepts reality as it is, whether that be man's fallenness or in a sort of `higher justice or purpose' of suffering manifesting itself in this life (v). In these ways, active love is firmly compatible with Christianity and, in fact, is a clear `way of life' that is rooted in Christianity. 

Gratitude necessarily follows from the Christian belief that we are created beings entirely dependent on God --- the very God who made man in his image and breathed life into him. Just as Christ told us that we serve him by serving the poor, so too does active love focus on seeing Christ in all people. Just as Christ could not avoid death on a cross, so too man cannot avoid suffering. Christ became fully human, accepting man on man's own terms of fallenness, and suffered despite his innocence. Yet, precisely the point that was darkest in the Christian narrative --- Christ's complete gift of himself --- was also its great triumph. This is exactly the message that Zosima teaches: when one has given themselves entirely, when they are at their breaking point, precisely at that point will God's grace restore them and invigorate them.

Gratitude and man's relationship with God (i) and man's proper relationship with God was the focus of the first section. In this section, we learned through analysis of the Book of Job, Genesis and Pope Benedict XVI's homilies on creation that man is a created being who is entirely dependent on God and that God is a loving creator who longs for a personal, loving relationship with His creation. Gratitude is a fundamental element of Zosima's philosophy of active love and this is seen in his interpretation of Job. This particular interpretation is highlighted by the mystery of life --- that suffering fades quietly into joy, that Job is given an opportunity to praise God by essentially saying, ``Look, Lord. Your creation is good'', and that Christ himself could not avoid death on a cross. Suffering is an inescapable reality, but if it is borne with gratitude, joy can spring from its ashes.
 Proper perspective (ii) was discussed in the following section, with the intention of elucidating two points: that Ivan Karamazov's dilemma will never be completely resolved (iii) and that the key to active love is seeing Christ in all people (iv). This proper perspective becomes especially clear in Zosima's interaction with the Lady of Little Faith. The key passages here are those in which Zosima explains to the Lady of Little Faith that her anguish is genuine and not unfounded. The `great perhaps' is a real issue and it does not hide from those with even the greatest faith. He tells her that she can be convinced, but only by living out the arduous task of actively loving her neighbor and giving completely of herself. And, even while she is doing this, great emphasis is placed on the fact that she will be convinced of God only at her darkest point. In this way, we can understand that the dilemma itself cannot really be resolved by us. The grace that resolves the dilemma and convinces us comes as a revelation, at the moment when faith is at its darkest point.  This is a direct counter to Ivan's question of ``will it ever be resolved?'' 
 
The final section of the paper focused on active love's emphasis on accepting life as it is (v). In this final section, we analyzed Ivan's Rebellion and his Grand Inquisitor, who represents Ivan's greatest doubt: that man can love as Christ loved. In this section, we learned that Ivan struggles to accept reality as it is. Reality is that Turkish soldiers kill innocent mothers and their children --- there is much suffering in the world as a direct result of human malevolence. This is reality. Zosima's active love is able to overcome this reality by the very fact that it embraces this reality, just as Christ embraced this reality. True love for one's neighbor embraces the complete depravity of man, accepts it and is able to repair man's depravity. This process is not easy and it involves a genuine death. Death of the fallen man to be reborn as new. Actively loving one's neighbor is the way in which this process occurs: ``Know, then, that this dream, as you call it [Kingdom of Heaven on earth], will undoubtedly come true, believe it, though not now, for every action has its law. This is a matter of the soul, a psychological matter In order to make the world over anew, people themselves must turn onto a different path pyschically. Until one has indeed become the brother of all, there will be no brotherhood.''\footnote{ibid, 303.} In other words, through active love, we are able to die unto ourselves through the gift of ourselves to others and be remade anew. This is why the words of John 12:24 ring true throughout the text of the Brothers Karamazov: ``Verily, verily, I say unto you, except a corn of wheat fall into the ground and die, it abideth alone: but if it die, it bringeth forth much fruit.''\footnote{ibid, 309.} And this, too, is why it ``is a fearful thing to fall into the hands of the living God.''\footnote{ibid, 309.} Reality is undeniably hard, but Zosima's active love, embodied in Christ who is understood through the lens of John 12:24, is a way of life that is able to confront Ivan Karamazov's greatest doubts head on.
