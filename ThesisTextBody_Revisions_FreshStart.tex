\tableofcontents
\chapter{INTRODUCTION}

\pagestyle{myheadings}
\markright{\hfill Griggs \hspace{1 mm}}

\label{introduction}

\section{Statement of Thesis and Layout of Work}

In Dostoyevsky's \emph{Brothers Karamazov}, one of the major themes throughout is a particular articulation of the problem of suffering. In the novel, it is articulated most acutely by Ivan Karamazov. In the chapters "Rebellion" and "The Grand Inquisitor", Ivan tells of the suffering of innocent children at the hands of Turkish soldiers and the death of a young boy at the hands of a vengeful general. Ivan sees the suffering of these innocent children as a fundamental injustice that can never be accounted for and this leads him to rebel against God. Ivan has a ''childlike conviction that the sufferings will be healed and smoothed over, that the whole offensive comedy of human contradictions will disappear like a pitiful mirage, a vile concoction of man's Euclidean mind ... [eternal justice] will suffice not only to make forgiveness possible, but also to justify everything that has happened with men.''\footnote{Dostoyevsky, Fyodor. Translated by Richard Pevear, and Larissa Volokhonsky. \emph{The Brothers Karamazov: A Novel in Four Parts with Epilogue}. New York: Farrar, Straus and Giroux, 2002, 235.} Yet, even with this conviction, he simply does not accept God's world. We learn further that Ivan's rebellion is the result of his belief that men were inevitability bound to bring injustice into the world because ''Christ's love for people is in its kind a miracle impossible on earth.''\footnote{ibid, 237.} This is the ultimate cynicism lying behind Ivan's rebellion. Like the Grand Inquisitor, he cannot understand why God gave men freedom when men were bound to bring such suffering into the world: ```Man was made a rebel; can rebels be happy?'''\footnote{ibid, 251.}

Thus, we have in Ivan's words an articulation of the problem of suffering based in the belief that men cannot love as Christ loved. This is a belief that is diametrically opposed to the beliefs of Father Zosima who preaches the principles of ``active love''. The teaching of ``active love'' has three particular characteristics: (1) gratitude and appreciation for God's creation; (2) a proper perspective that is characterized by (2a) looking at the local, interpersonal level instead of the global, generalized level and (2b) realizing that the ultimate end is in heaven and not earth; and (3) the ability to accept reality as it is, particularly man's fallen nature and the inherent suffering in life. The ultimate result of ``active love'' is a redemptive transformation that enables one to confront and transcend Ivan Karamazov's cynicism and rebellion. Each of the three characteristics has its own role to play in this transformative process. Ivan's rejection of God is not a purely rational problem, in fact, it is most accurately understood as an interpersonal problem. This being the case, (1) gratitude and appreciation for God's creation is essential because it enables one to restore a proper relationship with God himself. Zosima uses Job as the exemplar of this gratitude, in the belief that Job was actually blessed with the opportunity to glorify God through overcoming the temptation to rebel, brought on by his suffering. Another component of Ivan's rebellion is that it is based at an abstracted level from reality. It is not based so much in his own suffering as it is in the suffering of the children that he has read about in newspapers. In order for suffering to enable a transformation, ``active love'' teaches (2) proper perspective by looking at the (2a) local level and by realizing that (2b) heaven is the ultimate end. When one looks at the local level, Zosima teaches, there is a way in which one learns to love as Christ loved. Instead of loving humanity, ``active love'' teaches one to love the people as individuals. Finally, Ivan's rebellion reflects an immature faith in that it cannot come to grips with the fact that there is suffering in the world and that man's nature is fallen. Zosima's ``active love'' wholeheartedly (3) accepts that these are realities and it does not try to avoid them or work around them. Instead, it embraces these realities and transforms through divine grace --- in much the same way that Christ embraced death on a cross in order that men could be redeemed. Ultimately, by wholeheartedly embracing these harsh realities, ``active love'' enables one to be transformed by the harsh realities in such a way that life becomes joyful.

These are lessons that are learned in the lives of the main characters of the novel: Zosima, Ivan, Alyosha and Dmitry. Each, in their own way, has a great encounter with suffering. Zosima suffers the loss of his beloved brother. Ivan struggles with the suffering of innocents and is driven to hallucination by the guilt of his father's murder. Alyosha rebels against his God for a brief moment at the `embarrassment' of Zosima's bodily corruption and Mitya is agonized by his actions and his false sentence. Each character has a particular response, or at least for Mitya and Ivan whose transformation is still in the works, and these responses are characterized by a development from an immature faith to a mature faith. The principles of ``active love'' can be seen in the transformation of each of the characters, whether it be Alyosha's movement away from a faith that seeks swift earthly justice or Ivan's struggle with his own fallenness and guilt. The ultimate conclusion of ``active love'' is that despite the great depths of man's depravity, life is joyful and in the words of Dmitry, describing life in the mines of Siberia: ``we shall be in chains and there will be no freedom, but then, in our great sorrow, we shall rise again to joy, without which man cannot live nor God exist, for God gives joy: it's His privilege...''\footnote{FIND THIS IN THE OTHER VERSION!, pg. 499 in the norton critical edition}

From where do these principles of ``active love'' arise? Alyosha's biographical account begins with an account of Zosima's older brother, Markel, who at a young age was turned to atheism and then reverted back to belief on his deathbed. Markel's death and his final words laid the seed in the teaching of ``active love'' that Zosima makes explicit in his life as a monk.

	\section{Zosima's Brother: The Beginning of ``Active Love''}
	In his biographical account of Father Zosima's life, Alyosha begins with the story of Zosima's brother, Markel. Markel befriends a young philosophy student who has been banished from Moscow for ``freethinking.''\footnote{248,garnett version...} After this encounter with the young ``freethinker'', there is a noted change in Markel in that he does not fast and he says that ``there is no God.''\footnote{248, garnett} Not long after this, however, in the sixth week of Lent, Markel became terminally sick and soon realized the nature of his sickness. Soon, a spiritual change took place in Markel. His faith was restored and even strengthened. He began speaking strangely, saying things such as: ``life is paradise, and we are all in paradise, but we won't see it, if we would, we should have heaven on earth the next day'' and ``everyone is really responsible to all men for all men and for everything.''\footnote{249-250, garnett} These sayings culminate in Markel's grandest admission of sin and gratitude: ``there was such a glory of God all about me; birds, trees, meadows, sky, only I lived in shame and dishonored it all and did not notice the beauty and glory.''\footnote{250,garnett} The last words mentioned in this section that are attributed to Merkel show a theological definition of heaven that is rooted in forgiveness and love that overcomes Merkel's deep sense of guilt: ``If I have sinned against everyone, yet all forgive me, too, and that's heaven. Am I not in heaven now?''\footnote{250, garnett}.
	
	An analysis of Markel's words reveals the beginnings of Zosima's ``active love''. 
	
	\chapter{GRATITUDE}



