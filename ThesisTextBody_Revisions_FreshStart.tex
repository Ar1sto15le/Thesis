\tableofcontents
\chapter{INTRODUCTION}

\pagestyle{myheadings}
\markright{\hfill Griggs \hspace{1 mm}}

\label{introduction}
\section{Ivan Karamazov's ``Problem of Suffering''}
In Dostoyevsky's \emph{Brothers Karamazov}, one of the major themes throughout is a particular articulation of the problem of suffering. In the novel, it is articulated most acutely by Ivan Karamazov. In the chapters ``Rebellion'' and ``The Grand Inquisitor'', Ivan tells of the suffering of innocent children at the hands of Turkish soldiers and the death of a young boy at the hands of a vengeful general. Ivan sees the suffering of these innocent children as a fundamental injustice that can never be accounted for and this leads him to rebel against God. Ivan has a ''childlike conviction that the sufferings will be healed and smoothed over, that the whole offensive comedy of human contradictions will disappear like a pitiful mirage, a vile concoction of man's Euclidean mind ... [eternal justice] will suffice not only to make forgiveness possible, but also to justify everything that has happened with men.''\footnote{Dostoyevsky, Fyodor. Translated by Richard Pevear, and Larissa Volokhonsky. \emph{The Brothers Karamazov: A Novel in Four Parts with Epilogue}. New York: Farrar, Straus and Giroux, 2002, 235.}

Yet, even with this conviction, he simply does not accept God's world. We learn further that Ivan's rebellion is the result of his belief that men were inevitability bound to bring injustice into the world because ''Christ's love for people is in its kind a miracle impossible on earth.''\footnote{ibid, 237.} This is the source of cynicism lying behind Ivan's rebellion. Like the Grand Inquisitor, he cannot understand why God gave men freedom when men were bound to bring such suffering into the world: ```Man was made a rebel; can rebels be happy?'''\footnote{ibid, 251.} These two beliefs denote an overarching belief that created beings are fallen beyond repair. Furthermore, he cannot understand why suffering of innocents seems to be a \emph{means to an end} --- that suffering is necessary in order to enable holiness. This concern is expressed most clearly in the Devil's description of Job: ``how many souls had to be destroyed ... in order to get just one righteous Job ... ''\footnote{ibid, 648} Ultimately, Ivan asks, why not renounce a God who necessitates such suffering and depend on man with ``his will and his science no longer limited''\footnote{ibid, 649.} to replace his heavenly hope with an earthly paradise.

\section{Statement of Thesis and Layout of Work}

Thus, we have in Ivan's words an articulation of the problem of suffering based in the belief that men cannot love as Christ loved and that suffering, particularly of the innocents, can never be accounted for. These beliefs are diametrically opposed to the beliefs of Father Zosima who preaches a Christian way of life based in ``active love''. The teaching of ``active love'' has three particular characteristics: (1) a gratitude that is indicative of a proper relationship with God; (2) a proper perspective that is characterized by (2a) looking at the local, interpersonal level instead of the global, generalized level and (2b) realizing that the ultimate end is in heaven and not earth; and (3) the ability to accept reality as it is, particularly man's fallen nature and the inherent suffering in life. \emph{Through the combination of these three characteristics, ``active love'' enables a redemptive transformation that overcomes Ivan Karamazov's ``problem of suffering''.}

Ivan's rejection of God is not a purely rational problem, in fact, it is most accurately understood as an interpersonal problem. This being the case, (1) gratitude and appreciation for God's creation is essential because it enables one to restore a proper relationship with God himself. Zosima uses Job as the exemplar of this gratitude, in the belief that Job was actually blessed with the opportunity to glorify God through overcoming the temptation to rebel, brought on by his suffering. Furthermore, there is a pertinent undercurrent in Job of a disbelief in God's goodness and the goodness of God's creation. This disbelief in God's goodness is a cause of an improper relationship with God. Furthermore, this doubt in the goodness of creation is also present in Ivan's disbelief in man's ability to love as Christ loved. In the text of Job, Zosima sees a way in which he goodness of creation is re-affirmed and man's relationship with God is restored and gratitude is the key to this restoration.

Another component of Ivan's rebellion is that it is based at an abstracted level from reality. It is not based so much in his own suffering as it is in the suffering of the children that he has read about in newspapers. In order for suffering to enable a transformation, ``active love'' teaches (2) proper perspective by looking at the (2a) local level and by realizing that (2b) heaven is the ultimate end. When one looks at the local level, Zosima teaches, there is a way in which one learns to love as Christ loved. Instead of loving humanity, ``active love'' teaches one to love the people as individuals.

Finally, Ivan's rebellion reflects an immature faith in that it cannot come to grips with the fact that there is suffering in the world and that man's nature is fallen. Zosima's ``active love'' wholeheartedly (3) accepts that these are realities and it does not try to avoid them or work around them. Instead, it embraces these realities and transforms men to cope with these realities through divine grace --- in much the same way that Christ embraced death on a cross in order that men could be redeemed. By wholeheartedly embracing these harsh realities, ``active love'' enables one to be transformed by the harsh realities in such a way that life becomes joyful.

These are lessons that are learned in the lives of the main characters of the novel: Zosima, Ivan, Alyosha and Dmitry. Each, in their own way, has a great encounter with suffering. Zosima suffers the loss of his beloved brother. Ivan struggles with the suffering of innocents and is driven to hallucination by the guilt of his father's murder. Alyosha rebels against his God for a brief moment at the `embarrassment' of Zosima's bodily corruption and Mitya is agonized by his actions and his false sentence. Each character has a particular response to suffering and these responses are characterized by a development from an immature faith to a mature faith. The principles of ``active love'' can be seen in the transformation of each of the characters, whether it be Alyosha's movement away from a faith that seeks swift earthly justice or Ivan's struggle with his own fallenness and guilt. The end product of ``active love'' is that despite the great depths of man's depravity, life is joyful and must be joyful. In the words of Dmitry, describing life in the mines of Siberia: ``we'll be in chains, and there will be no freedom, but then, in our great grief, we will arise once more into joy, without which it's not possible for man to live, or for God to be, for God gives joy, it's his prerogative, a great one.''\footnote{ibid, 592.}

From where do these principles of ``active love'' arise? Alyosha's biographical account begins with an account of Zosima's older brother, Markel, who at a young age was turned to atheism and then reverted back to belief on his deathbed. Markel's death and his final words laid the seed in the teaching of ``active love'' that Zosima makes explicit in his life as a monk. After an analysis of the Markel episode, the paper will begin working three each characteristic of ``active love'' in the order that they have been presented.

    \section{Zosima's Brother: The Beginning of ``Active Love''}
    
    In his biographical account of Father Zosima's life, Alyosha begins with the story of Zosima's brother, Markel. Markel befriends a young philosophy student who has been banished from Moscow for ``freethinking.''\footnote{ibid, 287.} After this encounter with the young ``freethinker'', there is a noted change in Markel in that he does not fast and he says that ``there isn't any God.''\footnote{ibid, 287.} Not long after this, however, in the sixth week of Lent, Markel became terminally sick and realized the nature of his sickness. Soon, a spiritual change took place in Markel. His faith was restored and even strengthened. He began speaking strangely, saying things such as: ``life is paradise, and we are all in paradise, but \emph{we do not want to know it} [emphasis mine]'' and ``that verily each of us is guilty before everyone, for everyone and everything.''\footnote{ibid, 288-289.} These sayings culminate in Markel's grandest admission of sin and gratitude: ``there was so much of God's glory around me: birds, trees, meadows, sky, and I alone lived in shame, I alone dishonored everything, and did not notice the beauty and glory of it all.''\footnote{ibid, 289.} Markel's last words in this section show a theological definition of heaven that is rooted in forgiveness and love that overcomes Merkel's deep sense of guilt: ``Let me be sinful before everyone, but so that everyone will forgive me, and that is paradise.''\footnote{ibid, 290.}.
    
    Markel has been transformed by his terminal illness in such a way that he is able to see that ``life is paradise''. This transformation is evident when he says that ``we do not want to know it'' --- if not for his own illness, he would not be able know, nor would he want to know, this paradise. Without his own illness and imminent death, he would not be able to notice the ``glory of God'' all around him. This, then, reveals that suffering can be transformative in such a way that it enables one to see God's glory and joy. Furthermore, his admission that he has dishonored God reveals that his illness, and consequent ability to see that ``life is paradise'', reveals a change in Markel's relationship with God. He seems to be repentant for his former way of relating to God and his relationship has now been transformed into a proper relationship with God. His abundant (1) gratitude for God's glory is a result of this proper relationship.
    
    Markel's suffering is transformative in another sense, too. It leads him to a great sense of awareness of his guilt and responsibility towards others: ``each of us is guilty before everyone.'' This responsibility is seen in the very personal and local interaction that he has with his nurse who is lighting a candle in front of an icon in his room. Previously, Markel had treated her harshly in refusing to let her pray and light the candle in front of an icon. His illness, however, causes him to change his ways: ``Light it, light it, dear, I was a wretch to have prevented your doing it.''\footnote{249, garnett} Markel has a similar transformation in his treatment of his mother. His statements of being responsible ``for all'', then, should be understood as having a local and particular application. Markel is not talking of a broad notion or a ``love of humanity''. Instead, he is talking of a love for individual persons that is expressed through encounters with the people right in front of him. Furthermore, notice that Markel's perspective is focused on heaven, the ultimate end, through the lens of this local application of love. He recognizes that he has sinned against those around him, particularly his mother and his nurse, and they have forgiven him. This leads him to ask: ``Am I not in heaven now?'' This is (2) proper perspective that is focused (2a) locally and (2b) on heaven, the second element of ``active love''.
    
    Finally, Markel is struck with an overwhelming sense of guilt, a feeling that he has sinned against everyone, including God and His creation. Furthermore, there is a deeper, more personal way in which this guilt shows itself, a very sickly way in which guilt causes men to bring suffering upon themselves. They feel so guilty that they \emph{``do not want to know'' the paradise before their very eyes}. Markel's acceptance of this guilt and his `sinfulness' is, perhaps, a recognition of the fallenness of man's nature. Man is guilty before God and he dishonors God's creation because he is fallen. Similarly, man is guilty before all because he is fallen and focused solely on himself --- as opposed to focusing on his neighbors, as the Bible instructs and ``active love'' teaches. Markel's sickness leads him to focus on his mother and his nurse, to be sensitive to their suffering and his poor treatment of them. In other words, Markel embraces his own sinfulness while simultaneously working to overcome his sinfulness. This the third and final element of ``active love'', (3) embracing the reality that man is depraved, but also the reality that because of Christ's sacrifice, man can overcome his fallen nature and ``love as Christ loved.''
    
    Once Markel accepts this guilt and \emph{accepts} forgiveness, he becomes joyful --- and this is the great mystery behind ``active love'', the joy that follows and comes often at the darkest moments in one's existence. The first key to such an overwhelming joy is gratitude, which stems from a proper relationship with God. It is not surprising, then, that the very next section in the novel is Zosima's interpretation of the ``Book of Job''. The ``Book of Job'' is Zosima's paradigm for (1) gratitude that is the result of a proper relationship with God. 
    
    \chapter{GRATITUDE: RESTORING A PROPER RELATIONSHIP WITH GOD}
    Alyosha's biographical sketch continues with Zosima's teachings on the ``Book of Job'', a Biblical text that is central to his teaching of ``active love''. We learn that Job was the first Biblical text that Zosima felt that he clearly understood from a young age. In Alyosha's words, a brief paraphrase of Job is given in which it becomes clear that Zosima sees Job as God's beloved servant: ``And have you seen my servant Job?'' God asks [Satan].\footnote{ibid, 291.} Further in the account, Zosima speaks of ``scoffers and blasphemers'' who asked why the Lord would allow His saint to suffer so greatly.\footnote{ibid, 292.} In a tone that foreshadows Ivan's devil, these ``scoffers'' go so far as to mock God as saying: ``See what my saint can suffer for my sake!''\footnote{ibid,292.} As if to ask, as Ivan does, is this suffering really necessary to prove Job's righteousness?
    
    Zosima answers this question by saying that what \emph{is actually} happening in the story of Job is that the trials of Job are not so much to prove Job's righteousness as they are an opportunity for Job to affirm the goodness of God's creation. God looks at his creation and praises it saying: ``That which I have created is good.''\footnote{ibid, 292.} And, Zosima says that Job is praising God and that he serves not only God, but also serves all of creation and all generations.\footnote{ibid, 292.} It is almost as if Zosima envisions Job saying, ``Yes, Lord. Your creation is indeed good.'' Furthermore, for Zosima and his teaching of ``active love'', Job illustrates the mystery at the heart of ``active love.'' This mystery is the way in which suffering and grief gradually pass into quiet, tender joy.\footnote{ibid, 292.} But Job's trial is severe and it is not quite as clear-cut as Zosima envisions. There are moments in which Job seems to call into question God's goodness and, consequently, the goodness of His creation. It is for this very reason, perhaps, that God points Job back to creation in his speech ``out of the whirlwind.'' In pointing back to creation, God is also pointing back to Genesis and, this being the case, a brief interpretation of Genesis will be included in this section as well.
    
    Markel's gratitude was a product of being able to see the goodness of God's creation, the ``glory all around him''. As Zosima tells it, one of the fundamental lessons of Job is that Job is given the opportunity to re-affirm the goodness of God's creation by enduring his trials. Markel was unable to see the goodness of creation until he underwent his own trial. Job, too, seems to suffer from an inability to fully comprehend the goodness of creation and, furthermore, his own faith undergoes a period of growth through his suffering and his encounter with God. Job was able to hear God, but he could not see God until he underwent his trial.
    
    \section{The Book of Job: A Trial That Enabled Job to See God}
    \begin{quote}
    \onehalfspacing
    Then Job answered the Lord: ``I know that thou canst do all things, and that no purpose of thine can be thwarted. `Who is this that hides counsel without knowledge?' Therefore I have uttered what I did not understand, things too wonderful for me, which I did not know. `Hear, and I will speak; I will question you, and you declare to me.' I had heard of thee by the hearing of the ear, but now my eye sees thee; therefore I despise myself, and repent in dust and ashes.\footnote{Edited by May, Herbert G., and Bruce Manning Metzger. ''The Book of Job.'' In The \emph{New Oxford Annotated Bible with the Apocrypha: Revised standard version, containing the second edition of the New Testament and an expanded edition of the Apocrypha}. New York: Oxford University Press, 1977. 613-655, Job 42:1-6} 
    \end{quote}
    
    This passage comes at the very end of the ``Book of Job'' and, for that reason, it summarizes the transformation of Job's relationship with God. He has gone from a relationship in which he only heard God with his ears to a relationship in which he sees God with his eyes. Without such a trial, it is not clear that Job would have had this transformation. He clearly knew God well before his trial, as he sacrificed on a regular basis and was ``was blameless and upright, one who feared God and turned away from evil.''\footnote{ibid, Job 1:1.}. Similarly, he must have been familiar with the religious laws of the time in order to successfully maintain his innocence and righteousness in the face of criticism from his friends. The success of his defense is even verified by God, Himself, when He says that Jobs friends have not spoken correctly, as Job has.\footnote{ibid, Job 42:7-9.}
    
    Yet the question emerges: if Job was correct in all that he had said of God, how could his suffering enable him to ``see'' God as opposed to ``hearing'' him? Perhaps the answer lies in an analysis of Job's potential error --- an underlying doubt in the goodness of God's creation --- an error that is related to Markel's inability to see God and Ivan's great doubt in the goodness of men.
    
    \begin{quote}
    \onehalfspacing
    ``Let the day perish in which I was born, and the night that said, `A man-child is conceived.' Let that day be darkness! May God above not seek it, or light shine on it. Let gloom and deep darkness claim it. Let clouds settle upon it; let the blackness of the day terrify it.''\footnote{ibid, Job 3:3-5}
    \end{quote}
    
    In the passage quoted above, it is certainly plausible that Job has called into question the goodness of God's creation and he has moved into ingratitude at God's creation. This ``lament is radically nihilistic, calling on forces of darkness, chaos, and death to negate light, life, conception, birth, and ultimately creation itself... In the egocentrism of despair, Job closes in upon himself and wills creation, too, to collapse into darkness and chaos.''\footnote{Schifferdecker, Kathryn. \emph{Out of the Whirlwind: Creation Theology in the Book of Job}. Cambridge, Mass.: Harvard Theological Studies, Harvard Divinity School :, 2008, 9.} This self-centeredness seems to be \emph{the result of doubt in God's goodness}. Without faith in God's goodness, man is left on his own terms in a world that he must master with his `will and his science', as Ivan's Devil says, and man cannot help but be self-centered. This, however, clashes with Mitya's great, ecstatic proclamation that ``man cannot live without joy''. And, if Mitya's statement is true, gratitude for God's creation is necessary for joy. Thus, another question emerges: how is Job's faith in God's goodness secured?   
    

\begin{quote}
Then the Lord answered Job out of the whirlwind: ``Who is this that darkens counsel by words without knowledge? Gird up your loins like a man, I will question you, and you shall declare to me. ``Where were you when I laid the foundation of the earth? Tell me, if you have understanding. Who determined its measurements --- surely you know! Or who stretched the line upon it? On what were its bases sunk, or who laid its cornerstone, when the morning stars sang together, and all the sons of God shouted for joy?\footnote{ibid, Job 38:1-7.}
\end{quote}

Given the overpowering nature of God's response, some interpreters have claimed that the ''divine speeches reveal God as a capricious, jealous tyrant who abuses his power.''\footnote{Schifferdecker, Kathryn. \emph{Out of the Whirlwind: Creation Theology in the Book of Job}. Cambridge, Mass.: Harvard Theological Studies, Harvard Divinity School :, 2008, 9.} Others suggest that the ''questions of the speeches are not designed to humiliate Job but to remind him of what he already knows. They enable him to realize anew that God establishes order in the cosmos... This order visible in the universe leads Job to trust God even when he does not understand why he suffers.''\footnote{ibid, 9.} And it is this second line of interpretation that is more consistent with Zosima's principles in that God's reference to creation is a way in which God can remind Job of his proper place in creation. Man's place in creation is one of the primary themes in the ``Book of Genesis''. And it is for this reason that God's speech, in which the majority of his words have to do with creation, can be seen as a pointer back to ``Genesis.''

\subsection{Genesis: The Lesson of Man's Place in Creation and His Relationship With God}
In Genesis it is written that man is made in the ''image and likeneGod's speech points us to Genesis where we learn that man is made in the ''image and likeness'' of God. As Benedict tells us, human ''life stands under God's special protection, because each human being, however wretched or exalted he or she may be, however sick or suffering, however good-for-nothing or important, whether born or unborn, whether incurably ill or radiant with health - [because] each one bears God's breath in himself or herself, each one is God's image'' [emphasis mine].\footnote{Ramsey, O.P., Boniface, and Pope Benedict XVI. \emph{In the Beginning: A Catholic Understanding of the Story of Creation and the Fall}, 45.} This is the God of abundant and gratuitous love of Christian theology.

It is with this loving, creative God in mind that Zosima is able to interpret the story of Job the way that  he does, without being convinced of God in a philosophical sense. In the same way that active love can mysteriously convince us of God's existence, so too does it lead to gratitude. This gratitude and the resulting joy is a mystery of life.  It is the mystery of everyday life in which Job's suffering is gradually healed. The old grief and pain ``gradually passes into quiet, tender joy; instead of young, ebullient blood comes a mild, serene old age: I bless the sun's rising each day and my heart sings to it as before, but now I love its setting even more, its long slanting rays, and with them quiet, mild, tender memories, dear images form the whole of a long and blessed life --- and over all is God's truth, moving, reconciling, all-forgiving!''\footnote{Dostoyevsky, Fyodor. Translated by Richard Pevear, and Larissa Volokhonsky. \emph{The Brothers Karamazov: A Novel in Four Parts with Epilogue}. 292} When we look to Christ, who has been prefigured by Job, we realize that ``the passing earthly image and eternal truth here touched each other. In the face of earthly truth, the enacting of eternal truth is accomplished.''\footnote{ibid, 292.}  The truth in this phrase is that Christ became man and accepted death on a cross in order to redeem humanity.  In Christ, the earthly truth of our created nature and our dependence on God touches the eternal truth of God's love for us.  Christ is the personification of our relationship to God the creator.  Because Christ, the son of God suffered, it is not merely a `child-like' conviction that God exists and that he loves his creation. Instead, it becomes an unavoidable reality that the analytic mind cannot reduce to axioms without losing the meaning of the mystery. ss'' of God and this is the interpretative key to understanding man's proper place in creation. In order to better grasp this concept, a collection of homilies by Pope Benedict XVI will be used. In this homilies, Pope Benedict highlights a few key themes which reveal that: man is given special affection as God's creation; that as a created being, man is entirely dependent on God's goodness for his existence and, therefore, it is man's proper place to praise God; and, finally, Pope Benedict reflects on `the fall' and gives an interpretation of `original sin' and `fallenness' that is especially helpful in understanding how `the fall' changed man's relationship to God and to other human beings. 

One theme is that since man is made in God's image, every single human life ''life stands under God's special protection, because each human being, however wretched or exalted he or she may be, however sick or suffering, however good-for-nothing or important, whether born or unborn, whether incurably ill or radiant with health - [because] each one bears God's breath in himself or herself, each one is God's image'' [emphasis mine].\footnote{Ramsey, O.P., Boniface, and Pope Benedict XVI. \emph{In the Beginning: A Catholic Understanding of the Story of Creation and the Fall}, 45.} 

It is with this loving, creative God in mind that Zosima is able to interpret the story of Job the way that  he does, without being convinced of God in a philosophical sense. In the same way that active love can mysteriously convince us of God's existence, so too does it lead to gratitude. This gratitude and the resulting joy is a mystery of life.  It is the mystery of everyday life in which Job's suffering is gradually healed. The old grief and pain ``gradually passes into quiet, tender joy; instead of young, ebullient blood comes a mild, serene old age: I bless the sun's rising each day and my heart sings to it as before, but now I love its setting even more, its long slanting rays, and with them quiet, mild, tender memories, dear images form the whole of a long and blessed life --- and over all is God's truth, moving, reconciling, all-forgiving!''\footnote{Dostoyevsky, Fyodor. Translated by Richard Pevear, and Larissa Volokhonsky. \emph{The Brothers Karamazov: A Novel in Four Parts with Epilogue}. 292} When we look to Christ, who has been prefigured by Job, we realize that ``the passing earthly image and eternal truth here touched each other. In the face of earthly truth, the enacting of eternal truth is accomplished.''\footnote{ibid, 292.}  The truth in this phrase is that Christ became man and accepted death on a cross in order to redeem humanity.  In Christ, the earthly truth of our created nature and our dependence on God touches the eternal truth of God's love for us.  Christ is the personification of our relationship to God the creator.  Because Christ, the son of God suffered, it is not merely a `child-like' conviction that God exists and that he loves his creation. Instead, it becomes an unavoidable reality that the analytic mind cannot reduce to axioms without losing the meaning of the mystery. 
