\tableofcontents
\chapter{INTRODUCTION}

\pagestyle{myheadings}
\markright{\hfill Griggs \hspace{1 mm}}

\label{introduction}
\section{Ivan Karamazov's ``Problem of Suffering'': A Brief Introduction}
In Dostoyevsky's \emph{Brothers Karamazov}, one of the major themes throughout is a particular articulation of the problem of suffering. In the novel, it is articulated most acutely by Ivan Karamazov. In the chapters ``Rebellion'' and ``The Grand Inquisitor'', Ivan tells of the suffering of innocent children at the hands of Turkish soldiers and the death of a young boy at the hands of a vengeful general. Ivan sees the suffering of these innocent children as a fundamental injustice that can never be accounted for and this leads him to rebel against God. At the same time, however, Ivan has a ''childlike conviction that the sufferings will be healed and smoothed over, that the whole offensive comedy of human contradictions will disappear like a pitiful mirage, a vile concoction of man's Euclidean mind ... [eternal justice] will suffice not only to make forgiveness possible, but also to justify everything that has happened with men.''\footnote{Dostoyevsky, Fyodor. Translated by Richard Pevear, and Larissa Volokhonsky. \emph{The Brothers Karamazov: A Novel in Four Parts with Epilogue}. New York: Farrar, Straus and Giroux, 2002, 235.}

Yet, even with this conviction, he simply does not accept God's world. We learn further that Ivan's rebellion is the result of his belief that men were inevitability bound to bring injustice into the world because ''Christ's love for people is in its kind a miracle impossible on earth.''\footnote{ibid, 237.} This is the source of cynicism behind Ivan's rebellion. Like the Grand Inquisitor, he cannot understand why God gave men freedom when men were bound to bring such suffering into the world: ```Man was made a rebel; can rebels be happy?'''\footnote{ibid, 251.} These two beliefs lie under an overarching belief that created beings are fallen beyond repair. Furthermore, he cannot understand why suffering of innocents seems to be a \emph{means to an end} --- that suffering is necessary in order to enable holiness. This concern is expressed most clearly in the Devil's description of Job: ``how many souls had to be destroyed ... in order to get just one righteous Job ... ''\footnote{ibid, 648} Ultimately, Ivan asks, why not renounce a God who necessitates such suffering and depend on man with ``his will and his science no longer limited''\footnote{ibid, 649.} to replace his heavenly hope with an earthly paradise?

\section{Statement of Thesis and Layout of Work}

Thus, we have in Ivan's words an articulation of the problem of suffering based in the belief that men cannot love as Christ loved and that suffering, particularly of the innocents, can never be accounted for. These beliefs are diametrically opposed to the beliefs of Father Zosima who preaches a Christian way of life based in ``active love''. The teaching of ``active love'' has three particular characteristics: (1) a gratitude that is indicative of a proper relationship with God; (2) a proper perspective that is characterized by (2a) looking at the local, interpersonal level instead of the global, generalized level and (2b) realizing that the ultimate end is in heaven and not earth; and (3) the ability to accept reality as it is, particularly man's fallen nature and the inherent suffering in life, but also the reality that Christ died in order that men could be redeemed from their fallen nature. \emph{Through the combination of these three characteristics, ``active love'' enables a redemptive transformation and a way of life that overcomes Ivan Karamazov's ``problem of suffering''.}

Ivan's rejection of God is not a purely rational problem, in fact, it is most accurately understood as an interpersonal problem. Even with an intellectual framework that would allow for suffering, Ivan rebels against God. This being the case, (1) gratitude and appreciation for God's creation is essential because it enables one to restore a proper relationship with God and created beings. Zosima uses Job as the exemplar of this gratitude, in the belief that Job was actually blessed with the opportunity to glorify God through overcoming the temptation to rebel, brought on by his suffering. Furthermore, there is a pertinent undercurrent in Job of a disbelief in God's goodness and the goodness of God's creation. This disbelief in God's goodness is a cause of an improper relationship with God. This doubt in the goodness of creation is also present in Ivan's disbelief in man's ability to love as Christ loved. In the text of Job, Zosima sees a way in which he goodness of creation is re-affirmed and man's relationship with God is restored and gratitude is the key to this restoration.

Another component of Ivan's rebellion is that it is based at an abstracted level from reality. It is not based so much in his own suffering as it is in the suffering of the children that he has read about in newspapers. In order for suffering to enable a transformation, ``active love'' teaches (2) proper perspective by looking at the (2a) local level and by realizing that (2b) heaven is the ultimate end. When one looks at the local level, Zosima teaches, there is a way in which one learns to love as Christ loved. Instead of loving humanity, ``active love'' teaches one to love the people as individuals.

Finally, Ivan's rebellion reflects an immature faith in that it cannot come to grips with the fact that there is suffering in the world and that man's nature is fallen. Zosima's ``active love'' wholeheartedly (3) accepts that these are realities and it does not try to avoid them or work around them. Instead, it embraces these realities and transforms men to cope with these realities through divine grace --- in much the same way that Christ embraced death on a cross in order that men could be redeemed. By wholeheartedly embracing these harsh realities, ``active love'' enables one to be transformed by the harsh realities in such a way that life becomes joyful.

These are lessons that are learned in the lives of the main characters of the novel: Zosima, Ivan, Alyosha and Dmitry. Each, in their own way, has a great encounter with suffering. Zosima suffers the loss of his beloved brother. Ivan struggles with the suffering of innocents and is driven to hallucination by the guilt of his father's murder. Alyosha rebels against his God for a brief moment at the `embarrassment' of Zosima's bodily corruption and Mitya is agonized by his actions and his false sentence. Each character has a particular response to suffering and these responses are characterized by a development from an immature faith to a mature faith. The principles of ``active love'' can be seen in the transformation of each of the characters, whether it be Alyosha's movement away from a faith that seeks swift earthly justice or Ivan's struggle with his own fallenness and guilt. The end product of ``active love'' is that despite the great depths of man's depravity, life is joyful and must be joyful. In the words of Dmitry, describing life in the mines of Siberia: ``we'll be in chains, and there will be no freedom, but then, in our great grief, we will arise once more into joy, without which it's not possible for man to live, or for God to be, for God gives joy, it's his prerogative, a great one.''\footnote{ibid, 592.}

From where do these principles of ``active love'' arise? Alyosha's biographical account begins with an account of Zosima's older brother, Markel, who at a young age was turned to atheism and then reverted back to belief on his deathbed. Markel's death and his final words laid the seed in the teaching of ``active love'' that Zosima makes explicit in his life as a monk. After an analysis of the Markel episode, the paper will work through each characteristic of ``active love'', highlighting the way in which it is a response to Ivan Karamazov's doubts. In order to assess ``active love's'' effectiveness, however, the proper place to begin this paper is in a close analysis of Ivan's arguments in the sections ``Rebellion'' and ``The Grand Inquisitor''.

\chapter{IVAN KARAMAZOV'S ``PROBLEM OF SUFFERING''}
\section{``Can it be resolved?''}
Ivan Karamazov's challenge with his ``problem of suffering'' is particularly striking because it is ambiguous as to whether or not Ivan completely believes his own argument. Furthermore, it is striking because of Ivan's wide ranging character. On the one hand, he is presented as the logician: cool, composed, calculating and rational; playing intellectual parts as opposed to fully believing them; and apathetic towards the needs of others. On the other hand, Ivan is shown to be a sensualist who loves life and who is particularly sensitive towards the suffering of others. Ivan is a conflicted character and at the center of this conflict is his inability to reconcile a loving God with the suffering of innocent children. Yet, at the same time, he recognizes that God is the key to morality: ``if there is no God, everything is permitted.''\footnote{ibid, 82.} In typical Karamazov fashion, Ivan holds these two seemingly irreconcilable beliefs and is tormented by this conflict. This is why he is so struck when Zosima addresses this internal torment directly.

After Ivan has explained this belief, Zosima asks Ivan:''Can it be that you really hold this conviction about the consequences of the exhaustion of men's faith in the immortality of their souls?''\footnote{ibid, 70.} It seems that Zosima is peering into Ivan's soul and sensing the depths of Ivan's troubled heart. Ivan's response is coolly logical: ''Yes, it was my contention. There is no virtue if there is no immortality.''\footnote{ibid, 70.} This is the extension of Ivan's idea that ``if there is no God, everything is permitted.''\footnote{ibid, 82.} It is important to note the nature of Zosima's response. Zosima responds to Ivan not with logic, but instead with his sense of Ivan's unhappiness --- his troubled heart. His response is one of genuine compassion and sincerity: ''You are blessed if you believe so, or else most unhappy!''\footnote{ibid, 82.} Blessedness and unhappiness --- Zosima does not try and refute Ivan's logic. He embraces Ivan's argument and, instead, he turns to Ivan himself and attempts to speak to Ivan's real problem: his tormented heart. Zosima continues to address the real issue at hand: ``This idea [that there is no virtue if there is no immortality] is not yet resolved in your heart and torments it. ... For the time being you, too, are toying, out of despair, with your magazine articles and drawing-room discussions, without believing in your own dialectics and smirking at them with your heart aching inside you. . . The question is not resolved in you, and there lies your great grief, for it urgently demands resolution. . .\footnote{ibid, 70.}

How accurately has Zosima diagnosed Ivan's problem? An answer to this question comes from Ivan himself: '''But can it be resolved in myself? Resolved in a positive way?' Ivan Fyodorovich continued asking strangely, still looking at the elder with a certain inexplicable smile.''\footnote{ibid, 70.} This strange smile comes after a quick admission of sincerity on Ivan's part --- '''But still, I wasn't quite joking either. . . ' Ivan Fyodorovich suddenly and strangely confessed---by the way, with a quick blush.''\footnote{ibid, 70.} In both of these statements and gestures (the strange confession and blush; the strange smile), it appears that Ivan has emerged from his wall of logic and insincerity and that Zosima has connected deeply with Ivan's great despair.

''Even if it cannot be resolved in a positive way, it will never be resolved in the negative way either---you yourself know this property of your heart, and therein lies the whole of its torment. But thank the Creator that he has given you a lofty heart, capable of being tormented by such a torment...''\footnote{ibid, 70.} This is an answer that communicates directly with Ivan and causes him to break his detached and unemotional behavior. Immediately after these words, Ivan suddenly rises from his chair, receives his blessing and kisses the elder's hand. He then returns to his chair, firm and serious, and a solemn moment of silence overtakes the room. This answer apparently speaks to Ivan in a profound way.

\section{Ivan Karamazov's Rebellion}
``Rebellion'' begins with Ivan and Alyosha talking over tea. It is during this conversation that Ivan reveals his Karamazov sensuality. Importantly, this conversation is the context in which Ivan's rejection of God takes place --- a rejection that happens despite an intellectual, though naive, framework in which suffering could be understood. In this conversation, Ivan explains his great desire to live and his love for life to Alyosha: ``If I did not believe in life, if I were to lose faith in the woman I love, if I were to lose faith in the order of things, even if I were to become convinced on the contrary, that everything is a disorderly, damned, and perhaps devilish chaos, if I were struck even by all the horrors of human disillusionment---still I would want to live...''\footnote{ibid, 230.}

This love for life is deeply rooted in Ivan and it is clearly associated in the text to the sensual nature of the Karamazov men --- the `sensualists'. This visceral desire for life is dramatically opposed to Ivan's `Euclidean mind.'\footnote{ibid, 235.} Ivan is even willing to admit this: ''I want to live, and I do live, even if it be against logic.''\footnote{ibid, 230.} Thus, in this introduction we already see an acknowledgement on Ivan's part that he is not merely a logical being. He is, instead, a human being, with the desire to live. This point is important to note as it relates to the discussion of the limitations of reason, the interplay between faith and reason, and the realm of knowledge that faith deals in. In this particular case, Ivan has admitted that his reason cannot explain to him why he desires to live, nor can it explain his love for ''the sticky little leaves that come out in the spring'', nor his loved ones, nor his sensitivity towards the suffering innocent. This sensitivity towards the suffering innocent might, in fact, point back to God himself.

The next important point to notice in this section is that Ivan admits a number of important beliefs: a strong belief in God, a belief in our limited view of God and His intentions, and that it is ''this world of God's, created by God,'' that he does not accept and cannot agree with.\footnote{ibid, 235.}

\begin{quote}
And so, I accept God, not only willingly, but moreover I also accept his wisdom and his purpose, which are completely unknown to us; I believe in order, in the meaning of life, I believe in eternal harmony, in which we are all supposed to merge, I believe in the Word for whom the universe is yearning, and who himself was 'with God,' who himself is God, and so on....\footnote{ibid, 235.}
\end{quote} 

So Ivan accepts God willingly... and yet he still refuses God's world. This refusal even comes in spite of his ''childlike conviction that the sufferings will be healed and smoothed over, that the whole offensive comedy of human contradictions will disappear like a pitiful mirage, a vile concoction of man's Euclidean mind ... it will suffice not only to make forgiveness possible, but also to justify everything that has happened with men.''\footnote{ibid, 235.} In other words, Ivan has articulated a fairly complex framework for suffering. This is a framework in which suffering can be justified and one that even articulates and accepts the mystery of how suffering will be justified. It appears, then, that Ivan's rejection is even against his own logic --- he refuses God simply because he chooses to not accept the world as it is.

One possible reason that Ivan cannot accept this ``childlike conviction'' or give himself over to his childlike faith is because his conception of faith is immature. It lacks substance in the form of personal experience but also in its naivety. His is a faith that seeks swift justice on earth and cannot accept man's responsibility for his own fallenness and depravity: ``What do I care that \emph{none are to blame and that I know it} [emphasis mine]---I need retribution ... retribution not somewhere and sometime in infinity, but here and now, on earth, so that I see it myself.''\footnote{ibid, 244.} Furthermore, Ivan's faith is immature in the sense that he cannot understand the great and overwhelming power of forgiveness. His faith has not matured beyond vindication and retribution. This is seen in his statement that the mother of the murdered child ``has no right to forgive the suffering of her child who was torn to pieces, she dare not forgive the tormentor, even if the child himself were to forgive him!''\footnote{ibid, 245.}

In response to Ivan's great desire for retribution, Alyosha is quick to remind Ivan that there ``is in the whole work a being who could and would have the right to forgive'', Christ Jesus who ``can forgive everything, forgive all \emph{and for all}, because he himself gave his innocent blood for all and for everything.''\footnote{ibid, 246.} Alyosha is referring to Christ's transformational sacrifice that redeemed humanity and allowed for the potential to overcome the effects of the fall and original sin. Interestingly, however, even Ivan brings out a vindicative, retribution-seeking faith in his pious brother Alyosha. Ivan asks Alyosha what he would do to punish the military commander who hunted down a young boy: ```Shoot him!' Alyosha said softly, looking up at his brother with a sort of pale, twisted smile.''\footnote{ibid, 243.} It is not difficult to see that Alyosha, in his own way, suffers from an immature faith as well. This is a point that Rowan Williams emphasizes in his interpretation of the \emph{Brothers Karamazov}.

\section{Alyosha's Lapse of Faith}
In fact, Rowan Williams ties Alyosha together with Ivan in that he says both Alyosha and Ivan say: ''God exists but I am not sure whether I believe in him...''\footnote{Williams, Rowan. \emph{Dostoevsky: Language, Faith, and Fiction}, viii.} As we see in the novel, Alyosha's faith is transformed from its naive state to a more mature faith throughout the course of the novel. As Williams puts it: ''Alyosha has sensed a divine abundance and liberty that exceeds human standards of success and failure; his belief has been transformed---but not in the sense that he has become convinced of God's existence. It is rather that he now sees clearly what might be involved in a life that would merit being called a life of faith.''\footnote{ibid, viii.} The particular incident that Williams is referring to is the embarrassment of Alyosha's beloved elder, Zosima, in its odor of corruption. In that case, Alyosha was seeking a a swift justice based in the \emph{human standards} of holiness and sainthood. This is seen particularly in ``An Opportune Moment'', when Alyosha has just fled the monastery and is in his greatest moment of despair. The narrator explains that ``it was not miracles [Alyosha] needed, but only a ``higher justice'', which, as he believed, had been violated---it was this that wounded his heart so cruelly and suddenly.''\footnote{ibid, 339.} Eventually this wound causes Alyosha to mimic Ivan in saying: ``I do not rebel against my God, I simply `do not accept his world.'''\footnote{ibid, 341.}

Alyosha's dejection was a result of a naive faith that sought a higher justice for personal edification. Williams, with a view towards the whole novel, explains that the novel itself puts forth a form of mature faith:
\begin{quote}
What [Dostoevsky] does in Karamazov is not to demonstrate that it is possible to imagine a life so integrated and transparent that the credibility of faith becomes unassailable;'' --- note that this is precisely what is envisioned in Ivan's child-like faith, a vision of a life that integrates the paradox of suffering in God's world in such a way that it can transparently be understood and framed neatly --- ''it is simply to show that faith moves and adapts, matures and reshapes itself, not by adjusting its doctrinal content ... but by relentless stripping away from faith of egotistical or triumphalistic expectations. The credibility of faith is in its freedom to let itself be judged and to grow.\footnote{ibid, x.} 
\end{quote}
In other words, in Williams' reading of the novel, Zosima's words ring true --- there will never be a resolution to Ivan's dilemma --- but, instead, there will be a continual growth of faith. This form of faith is Zosima's ``active love.''

The act that restores Alyosha's faith is clearly categorized as a moment that `strips away Alyosha's egotistical expectations' and as one that is an act of genuine, personal kindness from the least expected woman in the novel, Grushenka for whom Alyosha has developed a particular repugnance. Grushenka restores Alyosha's faith! Grushenka, the woman who is able to elicit the closest thing to a moral judgment on the part of Alyosha! This is a moment that illustrates the great power of ``active love'' --- in Alyosha's darkest moment, such a sinful woman as Grushenka has pity on him and restores his faith. It is a small, personal act of kindness that reveals a great capacity for love even in the soul of a woman who has been the center of a love triangle between a father and his son. These are the sorts of moments that serve to counter-act Ivan's belief that men cannot love as Christ loved.

\section{Ivan's Grand Inquisitor: Crucified Christ and Ivan's Immature Faith}
In their conversation at the tavern, Ivan continues to explain to Alyosha why he does not accept God's world. This explanation takes the form of a short poem called ``The Grand Inquisitor''. The core belief operating throughout this poem is that men are incapable of being redeemed and that they must be controlled through hunger and fear. This viewpoint is explained through the Grand Inquisitor himself.

''My poem is called 'The Grand Inquisitor'; it's a ridiculous thing, but I want to tell it to you.''\footnote{ibid, 246.} And so begins Ivan's Grand Inquisitor. This prose poem of Ivan's reveals a couple of things about Ivan's argument. Firstly, it reveals that Ivan's primary problem with the world that God has created is the freedom that has been given to man. This freedom is what allows for turkish soldiers to cut babies from their mothers' wombs. This freedom is the necessary condition for evil. Coupling this freedom with the impossibility, in Ivan's eyes at least, of humans to love as Christ and it is possible to see the dystopia that has evolved in Ivan's mind. How can anyone possibly continue living in such a world where true love for fellow men is an impossible 'miracle'? This world is inherently a world of suffering. Hans Kung is particularly helpful in understanding Ivan's rebellion and, he explains, in suffering of he innocent, ''lies the secret of Ivan's atheism, the basis for his 'rebellion'.''\footnote{Jens, Walter, and Hans Kung. ``Religion in the Controversy Over the End of Religion.'' In \emph{Literature and Religion: Pascal, Gryphius, Lessing, Hölderlin, Novalis, Kierkegaard, Dostoyevsky, Kafka}. New York: Paragon House, 1991. 227-242, 242.}

Kung again explains: ''In suffering, especially in that of the innocent, man comes up against his extreme limit, comes to the decisive question of his identity, of the sense and nonsense of his living and dying, indeed, of reality pure and simple. Given the overwhelming reality of suffering the life and history of humanity does the suffering, doubting, despairing person really have any other choice? What alternative is there to the rebellion of an Ivan Karamazov against this world of God that he finds so unacceptable...''\footnote{ibid, 234.} Ivan, himself, is able to recognize an alternative --- although he cannot fully convince himself to believe it. This is a second aspect that becomes apparent in his poem. Ivan has an intuition about an alternative to a world of suffering: a world in which men love as Christ do. This is a world of Zosima's 'active love.'

Here, again, Kung is particularly helpful: ''But this world of Ivan, so subtly portrayed, is now contrasted, in serenity and great inner freedom, with an alternative world that has its own plausibility. While Ivan primarily talks, Alyosha acts. Dostoyevsky was convinced that on the ultimate theological issues rational argumentation was impotent.''\footnote{ibid, 236.} Thus, the very fact that Ivan has Christ kissing the Grand Inquisitor is a certain admission of Ivan's that this argument does not exist purely within the realm of reason. This is corroborated by both Ivan's illogical desire to live as well he his irrational rebellion. Remember that his rebellion is irrational in this sense: despite being able to give an account of how it could be possible for suffering to be permitted and justified, and even having a 'childlike conviction', Ivan still rebels.

Ivan's story begins with Christ coming to earth once more as a man in the town of Seville, where he alludes to a burning during the Inquisition in which ''had burned almost a hundred heretics at once ad majorem gloriam Dei.''\footnote{Dostoyevsky, Fyodor. Translated by Richard Pevear, and Larissa Volokhonsky. \emph{The Brothers Karamazov: A Novel in Four Parts with Epilogue}, 243.} He describes how Christ was both unobserved and yet ''every one recognized Him.'' Here it is worth noting that Ivan thinks these are some of the best lines in the poem: ''This could be one of the best passages in the poem, I mean, why it is exactly that they recognize him. People are drawn to him by an invincible force, they flock to him, surround him, follow him. He passes silently among them with a quiet smile of infinite compassion.''\footnote{ibid, 249.} Ivan continues to describe how Christ performs a couple of miracles --- healing a blind man and raising a child from the dead. While this Christ is certainly similar to the one that we see in the Gospels, it is perhaps important to note three things: Ivan's Christ is incarnated as an adult man, he is described as if everyone suddenly recognizes that he is Christ and people flock to him as if they cannot help but believe in him --- the `invincible force'. This is a naive, childish understanding of Christ as a man.

Ivan's naive view is in contrast to Christ almost being killed by the people of his home town, the people who had grown up with Christ, in Luke 4:14-30. In other words, the people who had watched Christ grow up right in front of their eyes did not recognize him as the Son of God. They, in fact, attempted to kill him. Even the apostles, who left their entire lives behind to follow Christ were unable to recognize him without divine grace: ''Who do men say that the Son of man is?'' And they said, ``Some say John the Baptist, others say Eli'jah, and others Jeremiah or one of the prophets. He said to them, ``But who do you say that I am?'' Simon Peter replied, ``You are the Christ, the Son of the living God... For flesh and blood has not revealed this to you, but my Father who is in heaven.''\footnote{Edited by May, Herbert G., and Bruce Manning Metzger. \emph{The New Oxford Annotated Bible}, Matthew 16:13-18.} Only Simon Peter knew the answer to this question --- and this knowledge was not revealed to him by any flesh and blood, but by God himself. This is quite the contrast to Ivan's account of Christ.

Of course, Ivan's description of Christ is fictional and it occurs 15 centuries after Christ first lived. This might account for the fact that people recognize him. But, on the other hand, we still see in Scripture that Christ was not universally recognized, nor was he universally accepted as he appears to be in this poem of Ivan's. If Christ were to come down to earth, is this what Ivan would expect to happen? This image that Ivan has is perhaps in-line with his 'childlike-conviction' that we discussed earlier. It is an image that might be considered immature in this sense: Christ became fully man. Christ is not super-human, he is human. This means that Christ suffered, that he cried and that he confronted the suffering of those around him continually throughout his life. He brought people back from the dead, but he also cried at their deaths. Furthermore, in redeeming man through his death on the cross, Christ did not negate the fact that human beings suffer. In fact, he reaffirmed the harsh reality of the world in his death. Ivan, on the other hand, seems to believe (as his Grand Inquisitor does) that Christ could have controlled the world by performing miracles --- to keep people in line and essentially force them to believe him.

The Grand Inquisitor takes Christ captive and questions him --- although he does not permit Christ to speak. In his questioning, he explains that through the Inquisition he (and the Roman Catholic church) has ``finally overcome freedom'' and they ``have done so in order to make people happy.''\footnote{Dostoyevsky, Fyodor. Translated by Richard Pevear, and Larissa Volokhonsky. \emph{The Brothers Karamazov: A Novel in Four Parts with Epilogue}, 251.} The Grand Inquisitor continues to explain that God created man a rebel, that he knew man was a rebel from the very beginning, and that God knew that there would be only way way of arranging human happiness: through the fear and piety that is fostered by the Grand Inquisitor's church. ```Man was made a rebel; can rebels be happy? ... you rejected the only way of arranging for human happiness [by forcing belief through miracles and making the state the church], but fortunately, on your departure, you handed the work over to us.''\footnote{ibid, 251.} Here again are two important ideas related to Ivan's conception of freedom and human nature: freedom is at odds with men's happiness and that man 'was created a rebel'. If man was created a rebel, it follows logically that he is going to abuse his freedom and from this view a natural skepticism towards man's ability to love in a Christlike way emerges.

This cynicism manifests itself particularly in the Grand Inquisitors statement: ''But you did not know that as soon as man rejects miracles, he will at once reject God as well, for man seeks not so much God as miracles.''\footnote{ibid, 255.} It is this very miraculousness that Ivan paints Christ in, Christ with light and power radiating from his eyes, and perhaps this fixation on the miraculous is another key to Ivan's problem. If he is correct in his view, that ''man seeks not so much God as miracles'' there is a certain sense in which God can be proven to men and Christian morality can be instilled in them. In fact, this is exactly the sort of society that the Grand Inquisitor is seeking to uphold --- one in which the Church upholds society and its laws by upholding this miraculousness in a totalitarian sort of way. And yet this belief completely contradicts certain instances in scripture where Christ's miracles were rejected. Furthermore, it is opposed to a certain realism that is part of ``active love's'' acceptance of reality: namely, that miracles will not convince someone; even if a miracle stands before the unbeliever, the unbeliever will not believe it and will doubt his senses. 

\chapter{THE BEGINNING OF ``ACTIVE LOVE''}
\section{Zosima's Brother: The Inspiration}
In his biographical account of Father Zosima's life, Alyosha begins with the story of Zosima's brother, Markel. Markel befriends a young philosophy student who has been banished from Moscow for ``freethinking.''\footnote{ibid, 287.} After this encounter with the young ``freethinker'', there is a noted change in Markel in that he does not fast and he says that ``there isn't any God.''\footnote{ibid, 287.} Not long after this, however, in the sixth week of Lent, Markel became terminally sick and realized the nature of his sickness. Soon, a spiritual change took place in Markel. His faith was restored and even strengthened. He began speaking strangely, saying things such as: ``life is paradise, and we are all in paradise, but \emph{we do not want to know it} [emphasis mine]'' and ``that verily each of us is guilty before everyone, for everyone and everything.''\footnote{ibid, 288-289.} These sayings culminate in Markel's grandest admission of sin and gratitude: ``there was so much of God's glory around me: birds, trees, meadows, sky, and I alone lived in shame, I alone dishonored everything, and did not notice the beauty and glory of it all.''\footnote{ibid, 289.} Markel's last words in this section show a theological definition of heaven that is rooted in forgiveness and love that overcomes Merkel's deep sense of guilt: ``Let me be sinful before everyone, but so that everyone will forgive me, and that is paradise.''\footnote{ibid, 290.}.

Markel has been transformed by his terminal illness in such a way that he is able to see that ``life is paradise''. This transformation is evident when he says that ``we do not want to know it'' --- if not for his own illness, he would not be able know, nor would he want to know, this paradise. Without his own illness and imminent death, he would not be able to notice the ``glory of God'' all around him. This, then, reveals that suffering can be transformative in such a way that it enables one to see God's glory and joy. Furthermore, his admission that he has dishonored God reveals that his illness, and consequent ability to see that ``life is paradise'', reveals a change in Markel's relationship with God. He seems to be repentant for his former way of relating to God and his relationship has now been transformed into a proper relationship with God. His abundant (1) gratitude for God's glory is a result of this proper relationship.

Markel's suffering is transformative in another sense, too. It leads him to a great sense of awareness of his guilt and responsibility towards others: ``each of us is guilty before everyone.'' This responsibility is seen in the very personal and local interaction that he has with his nurse who is lighting a candle in front of an icon in his room. Previously, Markel had treated her harshly in refusing to let her pray and light the candle in front of an icon. His illness, however, causes him to change his ways: ``Light it, light it, dear, I was a wretch to have prevented your doing it.''\footnote{249, garnett} Markel has a similar transformation in his treatment of his mother. His statements of being responsible ``for all'', then, should be understood as having a local and particular application. Markel is not talking of a broad notion or a ``love of humanity''. Instead, he is talking of a love for individual persons that is expressed through encounters with the people right in front of him. Furthermore, notice that Markel's perspective is focused on heaven, the ultimate end, through the lens of this local application of love. He recognizes that he has sinned against those around him, particularly his mother and his nurse, and they have forgiven him. This leads him to ask: ``Am I not in heaven now?'' This is (2) proper perspective that is focused (2a) locally and (2b) on heaven, the second element of ``active love''.

Finally, Markel is struck with an overwhelming sense of guilt, a feeling that he has sinned against everyone, including God and His creation. Furthermore, there is a deeper, more personal way in which this guilt shows itself, a very sickly way in which guilt causes men to bring suffering upon themselves. They feel so guilty that they \emph{``do not want to know'' the paradise before their very eyes}. This sickly guilt is best characterized by Ivan Karamazov's great and overwhelming guilt --- and Ivan's greatest trial is owning up to this guilt. Markel's acceptance of this guilt and his `sinfulness' is, perhaps, a recognition of the fallenness of man's nature. Man is guilty before God and he dishonors God's creation because he is fallen. Similarly, man is guilty before all because he is fallen and focused solely on himself --- as opposed to focusing on his neighbors, as the Bible instructs and ``active love'' teaches. Markel's sickness leads him to focus on his mother and his nurse, to be sensitive to their suffering and his poor treatment of them. In other words, Markel embraces his own sinfulness while simultaneously working to overcome his sinfulness. This the third and final element of ``active love'', (3) embracing the reality that man is depraved, but also the reality that because of Christ's sacrifice, man can overcome his fallen nature and ``love as Christ loved.''

Once Markel accepts this guilt and \emph{accepts} forgiveness, he becomes joyful --- and this is the great mystery behind ``active love'', the joy that follows and comes often at the darkest moments in one's existence. The first key to such an overwhelming joy is gratitude, which stems from a proper relationship with God. It is not surprising, then, that the very next section in the novel is Zosima's interpretation of the ``Book of Job''. The ``Book of Job'' is Zosima's paradigm for (1) gratitude that is the result of a proper relationship with God. 

\chapter{GRATITUDE: RESTORING A PROPER RELATIONSHIP WITH GOD}
Alyosha's biographical sketch continues with Zosima's teachings on the ``Book of Job'', a Biblical text that is central to his teaching of ``active love''. We learn that Job was the first Biblical text that Zosima felt that he clearly understood from a young age. In Alyosha's words, a brief paraphrase of Job is given in which it becomes clear that Zosima sees Job as God's beloved servant: ``And have you seen my servant Job?'' God asks [Satan].\footnote{ibid, 291.} Further in the account, Zosima speaks of ``scoffers and blasphemers'' who asked why the Lord would allow His saint to suffer so greatly.\footnote{ibid, 292.} In a tone that foreshadows Ivan's devil, these ``scoffers'' go so far as to mock God as saying: ``See what my saint can suffer for my sake!''\footnote{ibid,292.} As if to ask, as Ivan does, is this suffering really necessary to prove Job's righteousness?

Zosima answers this question by saying that what \emph{is actually} happening in the story of Job is that the trials of Job are not so much to prove Job's righteousness as they are an opportunity for Job to affirm the goodness of God's creation. God looks at his creation and praises it saying: ``That which I have created is good.''\footnote{ibid, 292.} And, Zosima says that Job is praising God and that he serves not only God, but also serves all of creation and all generations.\footnote{ibid, 292.} It is almost as if Zosima envisions Job saying, ``Yes, Lord. Your creation is indeed good.'' Furthermore, for Zosima and his teaching of ``active love'', Job illustrates the mystery at the heart of ``active love.'' This mystery is the way in which suffering and grief gradually pass into quiet, tender joy.\footnote{ibid, 292.} But Job's trial is severe and it is not quite as clear-cut as Zosima envisions. There are moments in which Job seems to call into question God's goodness and, consequently, the goodness of His creation. It is for this very reason, perhaps, that God points Job back to creation in his speech ``out of the whirlwind.'' In pointing back to creation, God is also pointing back to Genesis and, this being the case, a brief interpretation of Genesis will be included in this section as well.

Markel's gratitude was a product of being able to see the goodness of God's creation, the ``glory all around him''. As Zosima tells it, one of the fundamental lessons of Job is that Job is given the opportunity to re-affirm the goodness of God's creation by enduring his trials. Markel was unable to see the goodness of creation until he underwent his own trial. Job, too, seems to suffer from an inability to fully comprehend the goodness of creation and, furthermore, his own faith undergoes a period of growth through his suffering and his encounter with God. Job was able to hear God, but he could not see God until he underwent his trial.

\section{The Book of Job: A Trial That Enabled Job to See God}
\begin{quote}
\onehalfspacing
Then Job answered the Lord: ``I know that thou canst do all things, and that no purpose of thine can be thwarted. `Who is this that hides counsel without knowledge?' Therefore I have uttered what I did not understand, things too wonderful for me, which I did not know. `Hear, and I will speak; I will question you, and you declare to me.' I had heard of thee by the hearing of the ear, but now my eye sees thee; therefore I despise myself, and repent in dust and ashes.\footnote{Edited by May, Herbert G., and Bruce Manning Metzger. ''The Book of Job.'' In The \emph{New Oxford Annotated Bible with the Apocrypha: Revised standard version, containing the second edition of the New Testament and an expanded edition of the Apocrypha}. New York: Oxford University Press, 1977. 613-655, Job 42:1-6} 
\end{quote}

This passage comes at the very end of the ``Book of Job'' and, for that reason, it summarizes the transformation of Job's relationship with God. He has gone from a relationship in which he only heard God with his ears to a relationship in which he sees God with his eyes. Without such a trial, it is not clear that Job would have had this transformation. He clearly knew God well before his trial, as he sacrificed on a regular basis and was ``was blameless and upright, one who feared God and turned away from evil.''\footnote{ibid, Job 1:1.}. Similarly, he must have been familiar with the religious laws of the time in order to successfully maintain his innocence and righteousness in the face of criticism from his friends. The success of his defense is even verified by God, Himself, when He says that Jobs friends have not spoken correctly, as Job has.\footnote{ibid, Job 42:7-9.}

Yet the question emerges: if Job was correct in all that he had said of God, how could his suffering enable him to ``see'' God as opposed to ``hearing'' him? Perhaps the answer lies in an analysis of Job's potential error --- an underlying doubt in the goodness of God's creation --- an error that is related to Markel's inability to see God and Ivan's great doubt in the goodness of men.

\begin{quote}
\onehalfspacing
``Let the day perish in which I was born, and the night that said, `A man-child is conceived.' Let that day be darkness! May God above not seek it, or light shine on it. Let gloom and deep darkness claim it. Let clouds settle upon it; let the blackness of the day terrify it.''\footnote{ibid, Job 3:3-5}
\end{quote}

In the passage quoted above, it is certainly plausible that Job has called into question the goodness of God's creation and he has moved into ingratitude at God's creation. This ``lament is radically nihilistic, calling on forces of darkness, chaos, and death to negate light, life, conception, birth, and ultimately creation itself... In the egocentrism of despair, Job closes in upon himself and wills creation, too, to collapse into darkness and chaos.''\footnote{Schifferdecker, Kathryn. \emph{Out of the Whirlwind: Creation Theology in the Book of Job}. Cambridge, Mass.: Harvard Theological Studies, Harvard Divinity School :, 2008, 9.} This self-centeredness seems to be \emph{the result of doubt in God's goodness}. Without faith in God's goodness, man is left on his own terms in a world that he must master with his `will and his science', as Ivan's Devil says, and man cannot help but be self-centered. This, however, clashes with Mitya's great, ecstatic proclamation that ``man cannot live without joy''. And, if Mitya's statement is true, gratitude for God's creation is necessary for joy. Thus, another question emerges: how is Job's faith in God's goodness secured?  


\begin{quote}
Then the Lord answered Job out of the whirlwind: ``Who is this that darkens counsel by words without knowledge? Gird up your loins like a man, I will question you, and you shall declare to me. ``Where were you when I laid the foundation of the earth? Tell me, if you have understanding. Who determined its measurements --- surely you know! Or who stretched the line upon it? On what were its bases sunk, or who laid its cornerstone, when the morning stars sang together, and all the sons of God shouted for joy?\footnote{Edited by May, Herbert G., and Bruce Manning Metzger. ''The Book of Job.'' In The \emph{New Oxford Annotated Bible with the Apocrypha: Revised standard version, containing the second edition of the New Testament and an expanded edition of the Apocrypha}. New York: Oxford University Press, 1977. 613-655, Job 42:1-6}
\end{quote}

Given the overpowering nature of God's response, some interpreters have claimed that the ''divine speeches reveal God as a capricious, jealous tyrant who abuses his power.''\footnote{Schifferdecker, Kathryn. \emph{Out of the Whirlwind: Creation Theology in the Book of Job}. Cambridge, Mass.: Harvard Theological Studies, Harvard Divinity School :, 2008, 9.} Others suggest that the ''questions of the speeches are not designed to humiliate Job but to remind him of what he already knows. They enable him to realize anew that God establishes order in the cosmos... This order visible in the universe leads Job to trust God even when he does not understand why he suffers.''\footnote{ibid, 9.} And it is this second line of interpretation that is more consistent with Zosima's principles in that God's reference to creation is a way in which God can remind Job of his proper place in creation. Man's place in creation is one of the primary themes in the ``Book of Genesis''. And it is for this reason that God's speech, in which the majority of his words have to do with creation, can be seen as a pointer back to ``Genesis.''

\section{Genesis: The Lesson of Man's Place in Creation and His Relationship With God}
In Genesis it is written that man is made in the ''image and likeness'' of God and this is the interpretative key to understanding man's proper place in creation. In order to better grasp this concept, a collection of homilies by Pope Benedict XVI will be used. In this homilies, Pope Benedict highlights a few key themes which reveal that: man is given special affection as God's creation; that as a created being, man is entirely dependent on God's goodness for his existence and, therefore, it is man's proper place to praise God; and, finally, Pope Benedict reflects on `the fall' and gives an interpretation of `original sin' and `fallenness' that is especially helpful in understanding how `the fall' changed man's relationship to God and to other human beings. 

One theme is that since man is made in God's image, every single human life ''life stands under God's special protection, because each human being, however wretched or exalted he or she may be, however sick or suffering, however good-for-nothing or important, whether born or unborn, whether incurably ill or radiant with health - [because] each one bears God's breath in himself or herself, each one is God's image'' [emphasis mine].\footnote{Ramsey, O.P., Boniface, and Pope Benedict XVI. \emph{In the Beginning: A Catholic Understanding of the Story of Creation and the Fall}, 45.} Pope Benedict sees in the image of God breathing life into man, as described in Genesis 2:7, a loving and creative God who pours Himself into His own creation. 

This image of God calls to mind Psalm 139, particularly lines 13-14: ``For it was you who formed my inward parts; you knit me together in my mother's womb. I praise you, for I am fearfully and wonderfully made.''\footnote{Edited by May, Herbert G., and Bruce Manning Metzger. ''The Book of Job.'' In The \emph{New Oxford Annotated Bible with the Apocrypha: Revised standard version, containing the second edition of the New Testament and an expanded edition of the Apocrypha}. New York: Oxford University Press, 1977. Psalm 139:13-14.} God, in this view, loves His creation and cares for each particular individual that He knits them each in their mother's wombs. The Psalmist continues on to say that \emph{because ``I am fearfully and wonderfully made'', I praise you.} The Psalmist is articulating a relationship with God that is laced with gratitude. In fact, the relationship articulated here is a paradigm of what it means to be a created being, entirely dependent on God. Job, too, is aware of his dependence on the Lord: ''Naked I came from my mother's womb, and naked shall I return; the Lord gave, and the Lord has taken away; blessed be the name of the Lord.''\footnote{ibid, Job 1:21.} It is with this loving, creative God in mind that Zosima is able to interpret the story of Job the way that he does. 

But the question still remains: if God is such a loving God, why does Job suffer as he does and why does suffering exist in the world? Would such a loving God truly stand for this? This, of course, is the question that plagues Ivan and its answer, too, can be found in the text of Genesis that God has pointed Job to in His ``whirlwind'' speech. If suffering is a product of a doubt in God, a fracture in human beings' relationship with God, and if after this fracture, suffering has become an inherent part of existence in this world, then the suffering is understood in a completely different way than Ivan conceives it. Suffering does not necessarily have to be a means to an end --- a purification process, as Ivan's Devil suggests --- it might just be an unfixable aspect of human existence. The ``fall'', as Pope Benedict conceives it, describes exactly a fracturing of man's relationship with God and with other man in the creation of a `sin-damaged world'.\footnote{Ramsey, O.P., Boniface, and Pope Benedict XVI. \emph{In the Beginning: A Catholic Understanding of the Story of Creation and the Fall}, 73.}

\section{The Fall: Doubt Fractures Man's Relationship With God and Fellow Men}
Leading up to his discussion about original sin and `the fall', Pope Benedict highlights a very important detail about humans: they are relational and that ``they live in those whom they love and in those who love them and to whom they are `present'''\footnote{ibid, 72.} He puts it very succinctly when he says to ``be truly a human being means to be related in love, to be of and for...''\footnote{ibid, 72.} Sin destroys our relationship with God and with other humans. This is because sin is ``a rejection of relationality because it wants to make the human being a god... Consequently sin is always an offense that touches others, that alters the world and damages it... every human being enters into a world that is marked by relational damage...''\footnote{ibid, 73.} Job, too, is born into this sin-damaged world. He is not born with a fully intact relationship with God. This is, perhaps, why he has only heard of God but not seen God with his eyes. Similarly, Markel's guilt and `sin' is tied to improper relationality with God and with his nurse and mother. Markel's suffering made him aware of the fractures in his relationships. So to did Job's admonishment from God enable him to overcome his limited, fallen relationship with God and arrive at a higher, more proper relationship. 

This understanding of original sin also gives another way to comprehend suffering: suffering is the result of this sin-damaged world. Furthermore, it enables us to escape the problem of necessary suffering. Instead, we can say that suffering is unavoidable because we are all born into this ``sin-damaged world''. This ``sin-damaged world'' is clearly present throughout Dostoyevsky's novel in all of its characters and stories. In Zosima's relationship with Afanasy, Mitya's relationships with his father and Grushenka and Katerina Ivanovna and the relationships between the young boys. Even from a young age, grade school boys are capable of such great depravity as stabbing one another with pen-knives and pelting each other with rocks. The result of this ``sin-damaged'' world is one in which relationships with oneself are also damaged --- damaged in such a way that there is an all-consuming guilt that makes one ``not want to see paradise.''

Pope Benedict's explanation that sin is a ``rejection of relationality because it wants to make the human being a god...'' is particularly striking when taken together with Ivan's Devil's statement of men building a world for themselves. Words from the mouth of the devil that bring to mind the original temptation from the mouth of a snake: ``Now the serpent was more crafty than any other wild animal that the Lord God had made. He said to the woman, ``Did God say, `You shall not eat from any tree in the garden'?''\footnote{Edited by May, Herbert G., and Bruce Manning Metzger. ''The Book of Job.'' In The \emph{New Oxford Annotated Bible with the Apocrypha: Revised standard version, containing the second edition of the New Testament and an expanded edition of the Apocrypha}. New York: Oxford University Press, 1977. Genesis 3:1.} In analyzing this line, Pope Benedict is quick to point out that the serpents first words take the form of a question, in a form that is meant to bring doubt. Pope Benedict continues to explain that the first step in sin, as seen in Genesis 3:1, ``is not the denial of God but rather doubt about his covenant...''\footnote{Ramsey, O.P., Boniface, and Pope Benedict XVI. \emph{In the Beginning: A Catholic Understanding of the Story of Creation and the Fall}, 66.} In this line of analysis, in which doubt is the first step of sin, Job's ``nihilistic'' lamentations are also on the first way to sin --- which has been clarified as a fracturing of the relationship with God. Doubt, sin and Job are connected then in that doubt is the first step in sin that causes Job's relationship with God to be fractured. 

In this framework, Job can be reassured of God's goodness by the fact that sin and doubt from the very first instance changed the dynamics of the world in such a way that the world was damaged by sin and suffering became unavoidable. Only by restoring a proper relationship with God and with the world is Job able to ``see'' God and His glory, in the same way that Markel is able to see God's glory. And while suffering remains an inherent part of the world, once this proper relationship has been restored, there is a mysterious way in which God is able to heal the wounds of the suffering. This is what Zosima concludes his analysis on Job with: the old grief and pain ``gradually passes into quiet, tender joy; instead of young, ebullient blood comes a mild, serene old age: I bless the sun's rising each day and my heart sings to it as before, but now I love its setting even more, its long slanting rays, and with them quiet, mild, tender memories, dear images form the whole of a long and blessed life --- and over all is God's truth, moving, reconciling, all-forgiving!''\footnote{Dostoyevsky, Fyodor. Translated by Richard Pevear, and Larissa Volokhonsky. \emph{The Brothers Karamazov: A Novel in Four Parts with Epilogue}. 292} In accordance with the fact that ``active love'' is a way of life, and not simply a philosophical framework, Zosima does not try to understand how suffering is reconciled --- instead he embraces it as a ``great mystery of life''.\footnote{ibid, 292.}

In conversation with Ivan, Alyosha is quick to remind Ivan that Christ suffered for the sins of humanity in order that humanity might be redeemed. In Christ, ``the passing earthly image and eternal truth here touched each other. In the face of earthly truth, the enacting of eternal truth is accomplished.''\footnote{ibid, 292.} The truth in this phrase is that Christ became man and accepted death on a cross in order to redeem humanity. In Christ, the earthly truth of man's created nature and our dependence on God touches the eternal truth of God's love for humanity. Christ is the personification of man's relationship with God the creator. As such, Christ's sacrifice is the ultimate act that enables a redemptive transformation to take place. It is no wonder, then, that Zosima's ``active love'' is a direct counter to Ivan's critique that men cannot love as Christ loved. This love of neighbor is what the second principle of ``active love'' is primarily about --- along with the long-term vision that Christ's death was for eternal salvation and not earthly salvation.

\pagebreak
