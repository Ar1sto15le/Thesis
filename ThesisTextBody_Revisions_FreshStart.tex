\tableofcontents
\chapter{INTRODUCTION}

\pagestyle{myheadings}
\markright{\hfill Griggs \hspace{1 mm}}

\label{introduction}
\section{Ivan Karamazov's ``Problem of Suffering'': A Brief Introduction}
One of the major themes throughout Dostoevsky's {\emph{The Brothers Karamazov}}} is the particularly striking articulation of the problem of suffering by Ivan Karamazov. In the chapters ``Rebellion'' and ``The Grand Inquisitor'', Ivan shares stories from newspapers of babies being cut from their mother's wombs, thrown into the air and impaled by bayonets. He also tells a story of a young boy who was hunted down by a vengeful general, with hunting dogs and horses, as punishment for trespassing and killing game on his property. Ivan sees the suffering of these children as a fundamental injustice that can never be accounted for and this injustice leads him to rebel against God. At the same time, however, Ivan has a:

\begin{quote}
\singlespacing
childlike conviction that the sufferings will be healed and smoothed over, that the whole offensive comedy of human contradictions will disappear like a pitiful mirage, a vile concoction of man's Euclidean mind \ldots [eternal justice] will suffice not only to make forgiveness possible, but also to justify everything that has happened with men.\footnote{Fyodor Dostoevsky. Translated by Richard Pevear, and Larissa Volokhonsky. \emph{The Brothers Karamazov: A Novel in Four Parts with Epilogue}. (New York: Farrar, Straus and Giroux, 2002, Print), 235.}
\end{quote}

Yet, even with this conviction, he simply does not accept God's world. Ivan's rebellion is the result of his belief that men were inevitably bound to bring injustice into the world because ``Christ's love for people is in its kind a miracle impossible on earth.''\footnote{\emph{Ibid.}, 237.} This doubt is the source of cynicism behind Ivan's rebellion. Like the Grand Inquisitor, he cannot understand why God gave men freedom when men were bound to bring such suffering into the world: ```Man was made a rebel; can rebels be happy?'''\footnote{\emph{Ibid.}, 251.} These two beliefs lie under an overarching belief that created beings are fallen beyond repair. Theologically ``Ivan accepts original sin without the possibility of grace; he correctly sees the propensity for evil as fundamental but mistakenly denies that the potential for goodness goes just as deep.''\footnote{Gary Morson. ``The God of Onions''. In \emph{The Brothers Karamazov}, trans. Constance Garnett (New York: Random House, 1950), 788.}

Furthermore, he cannot understand why the suffering of innocents seems to be a \emph{means to an end} --- that suffering is necessary in order to enable holiness. This concern is expressed most clearly in the Devil's description of Job, the eponymous hero of the biblical text which pursues similar lines of questioning: ``how many souls had to be destroyed \ldots in order to get just one righteous Job \ldots''\footnote{\emph{Ibid.}, 648.} In other words, Ivan is trapped in an \emph{economic} mode of justice that is based in retribution. This economic justice is also seen in the logic of Job's interlocutors who essentially declare that Job must be guilty \emph{because} he is suffering, \emph{otherwise, if he were not guilty}, he would not be suffering.  Ultimately, Ivan asks, why not renounce a God who necessitates such suffering and depend on man with ``his will and his science no longer limited''\footnote{\emph{Ibid.}, 649.} to replace his heavenly hope with an earthly paradise?

\section{Statement of Thesis and Layout of Work}

Thus, in Ivan's words there is an articulation of the problem of suffering based in the belief that men cannot love as Christ loved and that suffering, particularly of the innocents, can never be accounted for. These beliefs are diametrically opposed to the beliefs of Father Zosima, who preaches a Christian way of life based in what he calls ``active love''. The teaching of ``active love'' has three particular characteristics: (1) a proper relationship with God which leads to gratitude and appreciation; (2) a perspective that is characterized by (2a) working towards truly loving one's neighbor as opposed to humanity in general and (2b) stepping outside of human standards of justice; and (3) honesty with oneself that enables the acceptance of reality as it is, particularly man's fallen nature and the inherent suffering in life, but also the reality that Christ died in order that men could be redeemed from their fallen nature. \emph{Through the combination of these three characteristics, ``active love'' enables a redemptive and transformative way of life that provides an alternative to Ivan Karamazov's rebellion.}

Ivan's rejection of God is not a purely rational problem; in fact, it is most accurately understood as an interpersonal problem. Even with an intellectual framework that would allow for suffering, Ivan rebels against God and does not accept creation. His problem, then, is with God himself. This being the case, there is a way in which Ivan's problem can be understood as his having a (1) fractured relationship with God. In particular, Ivan is calling into doubt the goodness of God's creation --- directly counter to God's affirmation in ``Genesis'' that all of creation is good.\footnote{c.f. Genesis 1:31.}

Reading Ivan in this way has its analogue in a reading of the Book of Job in which Job's error is understood as him doubting the goodness of Creation. In his teaching on Job, Zosima makes an explicit connection to God's affirmation of the goodness of creation. This idea is extended further and related to God's speech to Job, in which creation is explicitly invoked. As a result of this speech, Job's relationship with God is restored. It is in this vein that Zosima sets Job as an example of a proper relationship with God --- one in which God affirms the goodness of creation, while creation affirms the goodness of God.

Another component of Ivan's rebellion is that it is abstracted from reality in that it deals with humanity as a whole --- and when it does deal with particulars, these particulars are stories from newspaper clippings. In this abstract world, Ivan is convinced that it is impossible for men to love as Christ loved. Such a belief is representative of an improper relationship to his fellow human beings, in that he sees humanity in its \emph{the fallen state} as opposed to the state in which it \emph{is meant to be}. 

Through ``active love'', Zosima teaches that this relationship can be restored with (2) a correct perspective. Such a perspective begins to restore this relationship by (2a) focusing on local, particular applications of love towards one's neighbor as opposed to humanity in general and it also allows for \emph{true love} by (2b) casting aside earthly standards of justice, such as retribution and immediate gratification, for the sake of forgiveness and God's subtle, mysterious, non-triumphalistic glory. Zosima understands Job to be the great paradigm of this perspective shift, in that he seems to understand Job as a model of faith \emph{precisely because} the book shows that God's justice is incommensurable with human standards of justice. While loving as Christ loved is a difficult task --- it is ``labor and perseverance''\footnote{\emph{Ibid.}, 58.} --- with the (2) correct perspective it is not impossible.

Finally, Ivan's rebellion reflects an immature faith in that it cannot come to grips with the fact that there is suffering in the world and that man's nature is fallen. This inability to accept reality starts with his inability to settle his own guilt at implicitly allowing his father's murder. In this way, Ivan can be understood as having a fractured relationship with his own self. Above all, ``active love'' teaches one (3) to be honest with oneself. In restoring one's relationship with oneself, one is enabled to accept the extent of man's fallen nature, but also to accept forgiveness and grace. 

This particular aspect is best characterized in Zosima's encounter with the ``Mysterious Visitor''. This visitor is consumed by guilt at having murdered the woman he loved. Through his interaction with Zosima, the visitor is able to accept responsibility for his actions --- in exactly the same way that Zosima preaches taking responsibility for all --- and confesses publicly to the crime. Once he completes this great and terrible confession, ``by a great mystery of human life''\footnote{\emph{Ibid.}, 292}, he passes into ``joy and peace for the first time after so many years.''\footnote{\emph{Ibid.}, 311}  

Zosima's ``active love'', then, is the map by which the great doubts of Ivan can be confronted in an \emph{indirect manner} that ultimately leads to tender joyfulness. ``Active love'' is a way of life, rather than a discursive regime that enables one to heal relationships with God, with others and with oneself in such a way that the glory of God's creation and the forgiveness enabled by Christ's sacrifice become readily apparent. It should be noted that while ``active love'' does not give a \emph{point by point} rebuttal of Ivan's argument, it does, in its own way, \emph{actually embrace} Ivan's argument more fully than Ivan does himself. In other words, ``active love'' does not \emph{evade} the terrible crimes of which Ivan speaks --- instead, it embraces the human beings who commit them and offers redemption and transformation. 

This acceptance is the transformative, redemptive element that is crucial to ``active love'' and it is the same element to which John 12:24, the epigraph of the novel and a passage that is revisited throughout the novel, alludes: ``Verily, verily, I say unto you, Except a corn of wheat fall into the ground and die, it abideth alone: but if it die, it bringeth forth much fruit.'' This passage represents a hopeful, but realistic view of reality that is at the heart of ``active love'' --- a view that accounts for Ivan's great doubts but also \emph{transcends them.}


\chapter{IVAN KARAMAZOV'S ``PROBLEM OF SUFFERING''}
\section{``Can it be resolved?''}
Ivan Karamazov's ``problem of suffering'' is particularly striking because the novel is ambiguous as to whether or not Ivan completely believes his own argument. Furthermore, it is striking because of Ivan's complex character. On the one hand, he is presented as the logician: cool, composed, calculating, rational, and apathetic towards the needs of others. There is a sense in which Ivan merely plays the intellectual parts as opposed to believing them. On the other hand, Ivan is shown to be a sensualist who loves life and who is sensitive particularly toward the suffering of children. 

Ivan is a conflicted character and at the center of this conflict lies his inability to reconcile a loving God with the suffering of innocent children. At the same time, however, he recognizes that the postulate of God is the key to morality: ``if there is no God, everything is permitted.''\footnote{\emph{Ibid.}, 82.} Ivan \emph{is a Karamazov, after all}, marked by the capacity for contemplating ``two abysses, and both at the same time.''\footnote{\emph{Ibid.}, 718.} In typical Karamazov fashion, Ivan operates at opposite ends of the spectrum: he cannot completely let go of God, but he cannot accept Him either. He is tormented by this dilemma and this is why he is so struck when Zosima addresses this internal torment directly.

After Ivan has explained his belief that immortality, and consequently morality, depend upon God, Zosima asks Ivan: ``Can it be that you really hold this conviction?''\footnote{\emph{Ibid.}, 70.} It seems that Zosima is peering into Ivan's soul and sensing the depths of Ivan's troubled heart. Ivan's response is coolly logical: ``Yes, it was my contention. There is no virtue if there is no immortality.''\footnote{\emph{Ibid.}, 70.} This is the extension of his idea that ``if there is no God, everything is permitted.''\footnote{\emph{Ibid.}, 82.} Zosima responds to Ivan not with logic, but instead with genuine compassion and sincerity: ``You are blessed if you believe so, or else most unhappy!''\footnote{\emph{Ibid.}, 82.}

Blessedness and unhappiness --- Zosima does not try to refute Ivan's logic nor lessen the polarity. He takes Ivan's argument in stride and, instead, he turns to Ivan himself and addresses his inability to believe his own arguments: 

\begin{quote}
\singlespacing
This idea [that there is no virtue if there is no immortality] is not yet resolved in your heart and torments it. \ldots For the time being you, too, are toying, out of despair, with your magazine articles and drawing-room discussions, without believing in your own dialectics and smirking at them with your heart aching inside you\ldots The question is not resolved in you, and there lies your great grief, for it urgently demands resolution\ldots \footnote{\emph{Ibid.}, 70.}
\end{quote}

How accurately has Zosima diagnosed Ivan's problem? An answer to this question comes from Ivan himself: ``\thinspace`But can it be resolved in myself? Resolved in a positive way?' Ivan Fyodorovich continued asking strangely, still looking at the elder with a certain inexplicable smile.''\footnote{\emph{Ibid.}, 70.} This strange smile comes after a quick admission of sincerity on Ivan's part --- ``\thinspace`But still, I wasn't quite joking either\ldots' Ivan Fyodorovich suddenly and strangely confessed---by the way, with a quick blush.''\footnote{\emph{Ibid.}, 70.} In both of these statements and gestures, it appears that Ivan has emerged from his wall of logic and insincerity and that Zosima has connected deeply with Ivan's great despair. Zosima answers: 

\begin{quote}
\singlespacing
Even if it cannot be resolved in a positive way, it will never be resolved in the negative way either---you yourself know this property of your heart, and therein lies the whole of its torment. But thank the Creator that he has given you a lofty heart, capable of being tormented by such a torment\ldots\footnote{\emph{Ibid.}, 70.} 
\end{quote}

This answer communicates directly with Ivan and causes him to break from his detached and unemotional behavior. Immediately after these words, Ivan rises suddenly from his chair, receives his blessing and kisses the elder's hand. He then returns to his chair, firm and serious, and a solemn moment of silence overtakes the room. This answer apparently speaks to Ivan in a profound way. 

Zosima's answer is particularly revealing in that it causes Ivan to acknowledge a certain ineluctable mystery in life: that God's existence and one's relationship with God will always include an element of blind faith. As Ivan's torment and response suggest, Ivan cannot shake his own blind faith in God --- but neither can he understand that suffering is part of God's world. Ivan's inability to account for suffering leads to his great ``Rebellion''.

\section{Ivan Karamazov's Rebellion}
``Rebellion'' begins with Ivan and his younger brother Alyosha talking over tea. In this conversation, Ivan begins by explaining his great desire to live and his love for life to Alyosha:

\begin{quote}
\singlespacing
If I did not believe in life, if I were to lose faith in the woman I love, if I were to lose faith in the order of things, even if I were to become convinced on the contrary, that everything is a disorderly, damned, and perhaps devilish chaos, if I were struck even by all the horrors of human disillusionment---still I would want to live\ldots\footnote{\emph{Ibid.}, 230.}
\end{quote}

This love for life is deeply rooted in Ivan and it is clearly associated to the sensual nature of the Karamazov men --- the `sensualists'. This visceral desire for life is dramatically opposed to Ivan's `Euclidean mind.'\footnote{\emph{Ibid.}, 235.} Ivan is even willing to admit this: ``I want to live, and I do live, even if it be against logic.''\footnote{\emph{Ibid.}, 230.} \emph{In its irrationality, Ivan's desire to live is a matter of faith and mystery and is related to the mysterious ability of ``active love'' to find joy amidst great suffering.}

As the conversation continues, Ivan admits a number of important beliefs: a strong belief in God, a belief in a limited view of God and His intentions, and that it is ``this world of God's, created by God,'' that he does not accept and with which he cannot agree.\footnote{\emph{Ibid.}, 235.} This is an unusual way to frame his rejection because it signifies, at the very least, that he is not completely convinced in his doubt. This ambiguity is seen in his comment below:

\begin{quote}
\singlespacing
And so, I accept God, not only willingly, but moreover I also accept his wisdom and his purpose, which are \emph{completely unknown to us} [emphasis mine]; I believe in order, in the meaning of life, I believe in eternal harmony, in which we are all supposed to merge \ldots.\footnote{\emph{Ibid.}, 235.}
\end{quote} 

Despite Ivan's words above, he goes on to reject God and creation. Ivan has articulated a framework in which suffering can be justified by a higher, mysterious order --- and yet he still rejects God. It appears, then, that Ivan's rejection is even against his own logic and his own words --- he refuses God simply because he chooses not to accept the world as it is. This rejection indicates a fracture in his relationship with God.

At the root of this fracture is Ivan's immature faith which seeks swift justice on earth and cannot accept man's responsibility for his own fallenness: ``What do I care that \emph{none are to blame and that I know it} [emphasis mine]---I need retribution \ldots retribution not somewhere and sometime in infinity, but here and now, on earth, so that I see it myself.''\footnote{\emph{Ibid.}, 244.} His faith has not matured beyond vindication and retribution, in spite of the logical contradiction that even retribution cannot balance the scales of justice by undoing the crime. Furthermore, Ivan's faith is immature in the sense that he cannot understand the great and overwhelming power of forgiveness. This is seen especially in his statement that the mother of the murdered child ``has no right to forgive the suffering of her child who was torn to pieces, she dare not forgive the tormentor, even if the child himself were to forgive him!''\footnote{\emph{Ibid.}, 245.}

In response to Ivan's great desire for retribution, Alyosha is quick to remind Ivan that there ``is in the whole world a being who could and would have the right to forgive,'' Christ Jesus who ``can forgive everything, forgive all \emph{and for all}, because he himself gave his innocent blood for all and for everything.''\footnote{\emph{Ibid.}, 246.} Alyosha is referring to Christ's transformational sacrifice that redeemed humanity and allowed for original sin to be overcome. Interestingly, however, Ivan brings out a vindicative, retribution-seeking faith even in his pious brother Alyosha. Ivan asks Alyosha what he would do to punish the military commander who hunted down a young boy: ``\thinspace`Shoot him!' Alyosha said softly, looking up at his brother with a sort of pale, twisted smile.''\footnote{\emph{Ibid.}, 243.} 

Ivan hit a nerve and revealed Alyosha's own desire for retribution, a product of his own immature faith. Alyosha's immature faith is described by the narrator as needing a ``higher justice''\footnote{\emph{Ibid.}, 339.}, which is exemplified in his disillusionment when his beloved elder's body begins to decay after Zosima's death. It is no coincidence that Ivan and Alyosha both seek a swift, earthly justice. In fact, the narrator explains that if Alyosha ``had decided that immortality and God do not exist, he would immediately have joined the atheists and socialists.''\footnote{\emph{Ibid.}, 26.} 

\section{Alyosha's Lapse of Faith}
In his \emph{Dostoevsky: Language, Faith, and Fiction}, Rowan Williams ties Alyosha together with Ivan in that they both say: ``God exists but I am not sure whether I believe in him\ldots''\footnote{Rowan Williams. \emph{Dostoevsky: Language, Faith, and Fiction}. (London: Continuum, 2009, Print), viii.} Alyosha's faith is transformed from its na\"{\i}ve state to a more mature faith throughout the course of the novel. As Williams puts it: 

\begin{quote}
\onehalfspacing
Alyosha has sensed a divine abundance and liberty that exceeds human standards of success and failure; his belief has been transformed---but not in the sense that he has become convinced of God's existence. It is rather that he now sees clearly what might be involved in a life that would merit being called a life of faith.\footnote{\emph{Ibid.}, viii.} 
\end{quote}

The particular incident to which Williams is referring is the embarrassment of the body of Alyosha's beloved elder, Zosima, in its odor of corruption. In his expectation of preservation from decay, Alyosha was seeking a swift, vindicative justice based in the \emph{human standards} of holiness and sainthood. This is seen particularly in ``An Opportune Moment'', when Alyosha has just fled the monastery and is in his greatest moment of despair. The narrator explains that ``it was not miracles [Alyosha] needed, but only a ``higher justice'', which, as he believed, had been violated---it was this that wounded his heart so cruelly and suddenly.''\footnote{Dostoevsky, 339.} Eventually this wound causes Alyosha to mimic Ivan in saying: ``I do not rebel against my God, I simply `do not accept his world.'\thinspace''\footnote{\emph{Ibid.}, 341.}

Alyosha's dejection was a result of a na\"{\i}ve faith that sought a higher justice for personal edification. Williams explains that the novel itself puts forth a form of mature faith: ``What [Dostoevsky] does in Karamazov is not to demonstrate that it is possible to imagine a life so integrated and transparent that the credibility of faith becomes unassailable'' --- note that this is precisely what is envisioned in Ivan's child-like faith, a vision of a life that integrates the paradox of suffering in God's world in such a way that it can transparently be understood and framed neatly --- ``it is simply to show that faith moves and adapts, matures and reshapes itself, not by adjusting its doctrinal content \ldots but by relentless stripping away from faith of egotistical or triumphalistic expectations. The credibility of faith is in its freedom to let itself be judged and to grow.''\footnote{Williams, x.}

In Williams' reading of the novel, Zosima's words ring true --- there will never be a resolution to Ivan's dilemma --- but, instead, there will be a continual \emph{growth of faith}, embodied by ``active love''. Ivan must learn that a true, mature faith will endure a great amount of suffering and it will not seek vindication or retribution. A mature faith will be secure in its (2b) perspective that God's justice does not adhere to earthly standards. Furthermore, Ivan must become aware that his desire for retribution is based in a (2a) fractured relationship with others, in that retribution requires the very same punishment to be given to the criminal, and that he is being (3) dishonest with himself, in that \emph{he too desires to see bloodshed} and that he must grapple with his responsibility in his father's death. 

The act that restores Alyosha's faith is a moment that `strips away Alyosha's egotistical expectations'\footnote{\emph{Ibid.},x}, in which Grushenka shows sincere, genuine compassion at Zosima's death. This moment is an act of genuine, personal kindness from the least expected woman in the novel, Grushenka who, as the beloved of both Fyodor and Dmitry Karamazov, is the primary source of tension between son and father. In his darkest moment of despair, Alyosha let himself be taken to the house of Grushenka --- that horrible woman ``of whom he was afraid most of all.''\footnote{Dostoevsky, 349.} This is the same woman who manipulated Katerina Ivanovna for the sake of embarrassing her and gaining power over her saying: ``And you can keep this as a memory---that you kissed my hand, and I did not kiss yours.''\footnote{\emph{Ibid.}, 152.} This same Grushenka restores Alyosha's faith!

\begin{quote}
\onehalfspacing
I came here looking for a wicked soul---I was drawn to that, because I was low and wicked myself \ldots but I found a treasure---a loving soul \ldots She spared me just now \ldots I'm speaking of you, Agrafena Alexandrovna [Grushenka]. You restored my soul just now.\footnote{\emph{Ibid.}, 351.}
\end{quote}

This moment illustrates the great, mysterious power of ``active love'' --- in Alyosha's darkest moment, such a sinful woman has pity on him and restores his faith. It is a small, personal act of kindness that reveals a great capacity for love even in the soul of a self-acknowledged sinful woman. Moments like these serve to counter-act Ivan's belief that men cannot love as Christ loved. With this restored faith, Alyosha is able to endure the terrible murder of his father, Mitya's trial and Ilyushechka's death, the young boy that he had been mentoring. He is also able to labor patiently and restore the relationships of the young boys and Ilyushechka. Through it all, his matured faith is a source of hope and tender joy, in that it allows him to see goodness in all aspects of creation in spite of its fallenness. 

In the same way that Alyosha's triumphalistic, immature faith had to die in order to be reborn anew; so, too, does the process of ``active love'' lead one's soul through this transformation. After this transformation, he can recognize rightly ``not the false miracle,'' the triumphalistic miracle of Zosima's bodily preservation, but the ``true miracle that arises each day in each person when he freely follows the verdict of his heart and prefers doing good to doing evil.''\footnote{Nathan Rosen. ``Style and Structure in \emph{The Brothers Karamazov}''. In \emph{The Brothers Karamazov}, trans. Constance Garnett (New York: Random House, 1950), 741.}

\section{Ivan's Grand Inquisitor: Ivan's Immature Faith Blinds From ``Active Love''}
In their conversation at the tavern, Ivan continues to explain to Alyosha why he does not accept God's world. This explanation takes the form of a prose poem called ``The Grand Inquisitor''. The core belief operating throughout this poem is that men are incapable of being redeemed and that they must be controlled through hunger and fear. With such an emphasis, this work highlights a number of Ivan's doubts.

Firstly, it reveals that Ivan's primary problem with creation is that man has been given freedom. This freedom is what allows Turkish soldiers to cut babies from their mothers' wombs and is the necessary condition for evil. Couple this freedom with man's inability to love as Christ loved and it is possible to see the dystopia that has evolved in Ivan's mind. How can anyone possibly continue living in such a world where true love for fellow men is an impossible `miracle'? This is a world of suffering. Hans K\"{u}ng believes this reality to be the basis for Ivan's rebellion: ``Given the overwhelming reality of suffering the life and history of humanity does the suffering, doubting, despairing person really have any other choice [but to rebel]?''\footnote{Walter Jens and Hans K\"{u}ng. ``Religion in the Controversy Over the End of Religion''. In \emph{Literature and Religion: Pascal, Gryphius, Lessing, Ho\"{o}̈lderlin, Novalis, Kierkegaard, Dostoevsky, Kafka}. (New York: Paragon House, 1991), 227-242, 234.}

As a cool headed logician, Ivan can see no world in which God should allow suffering, and therefore he rebels. At the same time, however, Ivan clearly has an unshakable desire to live and a belief that life is joyful. This belief is closely related to an alternative reality, a world in which men love as Christ do --- the world of ``active love''. Here, again, K\"{u}ng is particularly helpful: ``But this world of Ivan, so subtly portrayed, is now contrasted, in serenity and great inner freedom, with an alternative world that has its own plausibility. While Ivan primarily talks, Alyosha acts. Dostoevsky was convinced that on the ultimate theological issues rational argumentation was impotent.''\footnote{\emph{Ibid.}, 236.} ``Active love'' \emph{is the alternative world} of which K\"{u}ng speaks. Its great strength is that it operates on the whole human plane instead of merely the rational plane in which Ivan is stuck.

Thus, the very fact that Ivan has Christ kissing the Grand Inquisitor at the close of the prose poem is a certain admission of this alternative world: ``But suddenly [Jesus] approaches the old man in silence and gently kisses him on his bloodless, ninety-year-old lips. That is the whole answer.''\footnote{Dostoevsky, 262.} His intuition of an alternative world is corroborated by both his illogical desire to live as well as his irrational rebellion. Furthermore, it is seen in his sincere reaction to Zosima's compassion: Ivan kisses his hand and accepts his blessing.

Throughout the novel, kissing signifies love, joyfulness and a love of life itself. Because of this irrational nature, kissing exemplifies the love for life found in Alyosha's statement to Ivan: ``I think that everyone should love life before everything else in the world.'' It is precisely this love for life that enables one to ``understand its meaning.''\footnote{\emph{Ibid.}, 231.} The truth of Alyosha's belief is made known by the act of kissing the ground. In spite of an inability to \emph{understand} suffering in life, his love for life brings him to a rapturous ecstasy in which he comes to a similar state as Markel: ``he did not try to understand why he longed so irresistibly to kiss it, to kiss all of it, but he was kissing it\ldots It was as if threads from all those innumerable worlds of God all came together in his soul\ldots''\footnote{\emph{Ibid.}, 362} With such an intuition, \emph{how is Ivan still blind to the alternative of ``active love''?} The answer lies in his immature faith.

Ivan's story begins with Christ coming to earth once more as a man in the town of Seville. He describes how Christ was both unobserved and yet ``everyone recognized Him \ldots People are drawn to him by an invincible force, they flock to him, surround him, follow him.''\footnote{\emph{Ibid.}, 249.} Ivan continues to describe how Christ performs a couple of miracles --- healing a blind man and raising a child from the dead. While Ivan's Christ represents Christ as known through the Gospels, there are three important differences: Ivan's Christ is incarnated as an adult man; he is described as if everyone suddenly recognizes him as Christ; and people flock to him as if they cannot help but believe in him --- as if by some `invincible force'. This is a na\"{\i}ve, childish understanding of Christ --- one that is triumphalistic and based in earthy standards. 

Ivan's na\"{\i}ve view is in contrast to Christ almost being killed by the people of his hometown, the people who had grown up with Christ. As seen in Luke 4:14-30, they actually attempted to kill him. Even the apostles, who left their entire lives behind to follow Christ, were unable to recognize him without divine grace: ``\thinspace`But who do you say that I am?' Simon Peter replied, `You are the Christ, the Son of the living God'\ldots `For flesh and blood has not revealed this to you, but my Father who is in heaven.'\thinspace''\footnote{Matthew 16:13-18.} Only Simon Peter knew the answer to this question, with the aid of grace. This account is quite the contrast to Ivan's.

Of course, Ivan's description of Christ is fictional and it occurs fifteen centuries after Christ first lived, which might account for the fact that people recognize him. But, on the other hand, Scripture affirms that Christ was not universally recognized nor universally accepted. If Christ were to come down to earth, is this what Ivan would expect to happen? This image is in line with Ivan's immature faith in this sense: Christ became fully human. This means that Christ suffered, that he cried and that he confronted the suffering of those around him continually throughout his life. He brought people back from the dead, but he also cried at their deaths. Furthermore, in redeeming man through his death on the cross, Christ did not negate the fact that human beings suffer. In fact, he reaffirmed the harsh reality of the world in his death. Ivan, on the other hand, seems to believe that Christ could have controlled the world by performing miracles --- to keep people in line and essentially force them to believe him.

The Grand Inquisitor takes Christ captive and questions him --- although he does not permit Christ to speak. In his questioning, he explains that through the Inquisition he (and the Roman Catholic church) have ``finally overcome freedom'' and they ``have done so in order to make people happy.''\footnote{Dostoevsky, 251.} The Grand Inquisitor continues to explain that God created man a rebel, that he knew man was a rebel from the very beginning, and that God knew that there would be only one way of arranging human happiness: through the fear and piety that is fostered by the Grand Inquisitor's church. ``\thinspace`Man was made a rebel; can rebels be happy? \ldots you rejected the only way of arranging for human happiness\ldots'\thinspace''\footnote{\emph{Ibid.}, 251.} If man was created a rebel, it follows logically that he is going to abuse his freedom and from this view a natural cynicism towards man's ability to love in a Christlike way emerges.

This cynicism manifests itself particularly in the Grand Inquisitor's statement: ``But you did not know that as soon as man rejects miracles, he will at once reject God as well, for man seeks not so much God as miracles.''\footnote{\emph{Ibid.}, 255.} It is this very miraculousness that Ivan paints Christ in and this fixation reveals another way in which Ivan's faith is immature. His belief in the miraculous completely contradicts certain instances in scripture where Christ's miracles were rejected, such as the miracle of the man with the withered hand in Mark 3. Ivan's emphasis on miracles also contradicts the third principle of ``active love'', (3) the acceptance of reality, in that miracles will not convince someone; even if a miracle stands before the unbeliever, the unbeliever will not believe it and will doubt his senses.\footnote{In the words of the narrator: ``It is not miracles that bring a realist to faith. A true realist, if he is not a believer, will always find in himself the strength and ability not to believe in miracles as well, and if a miracle stands before him as an irrefutable fact, he will sooner doubt his own senses than admit the fact.'' \emph{Ibid.}, 25.} 

In his doubts, Ivan can be understood as having a disconnect with God. He distrusts God and doubts the goodness of creation and in this way Ivan is similar to Job. This disconnect is addressed in the first principle of Zosima's ``active love,'' (1) restoring one's relationship with God. Furthermore, Ivan has an earthly and selfish perspective that seeks retribution, as well as a global focus on humanity and `society' as a whole. The second principle of ``active love,'' (2) proper perspective, addresses these issues and it encourages a perspective that will allow Ivan's faith to mature past its triumphalistic expectations. Finally, Ivan's emphasis on the miraculous, his emphasis on suffering and his unwavering belief that man cannot overcome his fallen nature is indicative of Ivan's inability to accept that human nature is fallen. The third principle of ``active love,'' (3) an acceptance of reality, enables a mature faith in that it embraces original sin but it also sees the ability for transformation. \emph{Through the combination of these three characteristics, ``active love'' enables a redemptive and transformative way of life that provides an alternative, however non-discursive,  to Ivan Karamazov's rebellion.}

\chapter{THE BEGINNING OF ``ACTIVE LOVE''}
\section{Zosima's Brother: The Inspiration}
In his biographical account of Father Zosima's life, Alyosha begins with the story of Zosima's brother, Markel. Markel befriends a young philosophy student who has been banished from Moscow for ``freethinking''.\footnote{\emph{Ibid.}, 287.} After this encounter with the young ``freethinker'', there is a noted change in Markel in that he does not observe the fast and denies the existence of God. Not long after this, however, in the sixth week of Lent, Markel becomes terminally ill. Soon, a spiritual change took place in Markel. His faith was restored and even strengthened. He began speaking strangely, saying things such as: ``life is paradise, and we are all in paradise, but \emph{we do not want to know it} [emphasis mine]'' and ``verily each of us is guilty before everyone, for everyone and everything.''\footnote{\emph{Ibid.}, 288-289.} These sayings culminate in Markel's grandest admission of sin and gratitude: ``there was so much of God's glory around me: birds, trees, meadows, sky, and I alone lived in shame, I alone dishonored everything, and did not notice the beauty and glory of it all.''\footnote{\emph{Ibid.}, 289.} Markel's last words in this section show a theological definition of heaven that is rooted in forgiveness and love that overcomes Markel's deep sense of guilt: ``Let me be sinful before everyone, but so that everyone will forgive me, and that is paradise.''\footnote{\emph{Ibid.}, 290.}

Markel has been transformed by his terminal illness in such a way that he is able to see that ``life is paradise.'' This transformation is evident when he says that ``we do not want to know it'' --- if not for his own illness, he would not be able to know, nor would he want to know, this paradise. In Markel's particular case, his illness is the occasion that leads him to restore his relationship with God, at his mother's prompting. His mother's concern for him leads him to realize the terminal nature of his illness. Hearing her weeping and prompting, Markel ``became angry and swore at God's Church, but still he grew thoughtful: he understood at once that he was dangerously ill''\footnote{\emph{Ibid.}, 288.} and he began going to church and receiving communion. In other words, an act of love on his mother's part prompted Markel to restore his relationship with God. 

His admission that he has dishonored God reveals that his illness, and consequent ability to see that ``life is paradise,'' reveals a change in Markel's relationship with God. He seems to be repentant for his former way of relating to God and his relationship has now been transformed into a proper relationship with God. His abundant (1) gratitude for God's glory is a result of this proper relationship. This episode, then, reveals that Markel's illness and his mother's love were catalysts for the restoration of his relationship with God. Furthermore, his illness and her love were catalysts in his decision to (2) restore his relationship with his mother: ``I'm [going to church and keeping the fast] only for your sake, mother, to give you joy and peace.''\footnote{\emph{Ibid.}, 288.} His admission of sinning against God also reveals that he has become honest with himself, thereby restoring his relationship with himself (3).

Markel's suffering is also transformative in another sense. It leads him to a great sense of awareness of his guilt and responsibility towards others. This responsibility is seen in the very personal and local interaction that he has with his nurse who is lighting a candle in front of an icon in his room. Previously, Markel had treated her harshly in refusing to let her pray and light the candle. His illness, however, causes him to change his ways: ``Light it, my dear, light it, what a monster I was to forbid you before!''\footnote{\emph{Ibid.}, 288.} Markel has a similar transformation in his treatment of his mother --- seen in his commitment to renew his faith and religious practices. 

Taking these actions into consideration, his statements of being responsible ``for all'' should be understood as having a local and particular application. Markel is not talking of a broad notion of a ``love of humanity''. Instead, he is talking of a love for individual persons that is expressed through encounters with the people right in front of him. Markel's perspective is focused on forgiveness through the lens of this local application of love and that his conception of paradise is based in forgiveness. He recognizes that he has sinned against those around him, particularly his mother and his nurse, and they have forgiven him. This leads him to ask: ``Am I not in heaven now?'' This is (2) proper perspective that is focused (2a) locally and (2b) on forgiveness, the second element of ``active love''.

Finally, Markel is struck with an overwhelming sense of guilt, a feeling that he has sinned against everyone, including God and creation. There is a deeper, more personal way in which this guilt shows itself, a very sickly way in which guilt causes men to bring suffering upon themselves. They feel so guilty that they \emph{``do not want to know'' the paradise before their very eyes}. This sickly guilt is best characterized by Ivan Karamazov's great and overwhelming sense of guilt over his involvement in his father's murder --- and Ivan's greatest sign of progress in the maturation of his faith is his admission of guilt. Markel's acceptance of this guilt and his `sinfulness' is a recognition of the fallenness of human nature. 

Man is guilty before God and he dishonors God's creation because he is fallen. Similarly, man is guilty before all because his fallenness leads to selfishness --- as opposed to focusing on his neighbors, as the Bible instructs and ``active love'' teaches. Markel's sickness leads him to focus on his mother and his nurse, to be sensitive to their suffering and his poor treatment of them. In other words, Markel embraces his own sinfulness while simultaneously working to overcome his sinfulness. This the third and final element of ``active love,'' (3) embracing the reality that man is capable of terrible crimes, but also the reality that because of Christ's sacrifice, man can overcome fallen human nature and ``love as Christ loved.'' 

Once Markel \emph{accepts this guilt and embraces forgiveness}, he becomes joyful --- and this is the great mystery behind ``active love'', the joy that follows and comes often at the darkest moments in one's existence. The first key to such an overwhelming joy is gratitude, which stems from a proper relationship with God. It is fitting that the very next section in the novel is Zosima's interpretation of the Book of Job, which serves as Zosima's paradigm for (1) gratitude that is the key to a proper relationship with God.

First, however, a brief note on Zosima's hermeneutical principles: Nathan Rosen is quick to point out that two ``things are noteworthy in [Zosima's] account of the Book of Job: he retells the story in great detail, and he leaves much out.''\footnote{Rosen, 728.} Rosen mentions that Job's defense of his innocence, his personal encounter with God and God's reply from the whirlwind \emph{are left out} of Zosima's interpretation. Rosen interprets this as Zosima's omission of ``Job's integrity and independence, his intellectual and spiritual energy---which in the end win God's favor\ldots''\footnote{\emph{Ibid.}, 728.} God's reply from the whirlwind and Job's personal encounter with God are \emph{direct allusions} to the creation accounts of Genesis and for this reason, Rosen's interpretation is incomplete.

Rosen's interpretation, however helpful, fails to recognize an important theme in Zosima's hermeneutics, namely, his \emph{explicit connection of Job to the Book of Genesis}. In particular, Zosima quotes God as saying ``That which I have created is good,'' which is a direct allusion to Genesis.\footnote{Dostoevsky, 292.} Furthermore, Zosima uses another direct allusion to the Genesis accounts of creation as the foundation for his ``active love'' in that his own `turning point' with Afanasy is based in being made in the image and likeness of God: ``Indeed, how did I deserve that another man, just like me, the image and likeness of God should serve me?''\footnote{\emph{Ibid.}, 298.} For this reason, it is textually more accurate to understand Zosima's interpretation not as \emph{omitting} Job's defense and God's whirlwind speech, but instead to understand his interpretation to be \emph{building on top of} this explicit connection with Genesis.

\chapter{GRATITUDE: RESTORING A PROPER RELATIONSHIP WITH GOD}
Alyosha's biographical sketch continues with Zosima's teachings on the Book of Job, a Biblical text that is central to his teaching of ``active love''. Job was the first Biblical text that Zosima understood as a child. In Alyosha's words, a brief paraphrase of Job is given in which it becomes clear that Zosima understands Job as God's beloved servant: ``And have you seen my servant Job?'' God asks [Satan].\footnote{\emph{Ibid.}, 291.} Further in the account, Zosima speaks of ``scoffers and blasphemers'' who asked why the Lord would allow His saint to suffer so greatly.\footnote{\emph{Ibid.}, 292.} In a tone that foreshadows Ivan's devil, these ``scoffers'' go so far as to mock God as saying: ``See what my saint can suffer for my sake!'',\footnote{\emph{Ibid.}, 292.} as if to ask, as Ivan does, if this suffering is really necessary to prove Job's righteousness. Is suffering really necessary as a means to an end?

Zosima answers this question by saying that what \emph{is actually} happening in the story of Job is that the trials of Job are not so much to prove Job's righteousness as they are an opportunity for Job to affirm the goodness of God's creation. God looks at his creation and praises it saying: ``That which I have created is good.''\footnote{\emph{Ibid.}, 292.} And, Zosima says that Job is praising God and that he serves not only God, but also serves all of creation and all generations.\footnote{\emph{Ibid.}, 292.} It is almost as if Zosima envisions Job saying, ``Yes, Lord. Your creation is indeed good.'' In the sense that Job's interlocutors who are pushing an \emph{earthly conception of justice}, Job also serves as a model in that his example (2b) \emph{justifies escaping from an earthly perspective into a divine, mysterious perspective on justice.}

Furthermore, Job illustrates the mystery at the heart of ``active love.'' This mystery is the way in which suffering and grief gradually pass into quiet, tender joy.\footnote{\emph{Ibid.}, 292.} But Job's trial is severe and it is not quite as clear-cut as Zosima envisions. There are moments in which Job seems to call into question God's goodness and, consequently, the goodness of creation, such as when he curses the day of his birth and calls for darkness to descend on the earth. It is for this very reason, perhaps, that God points Job back to creation in his speech ``out of the whirlwind'', as if \emph{to show Job creation from God's own perspective and re-affirm its goodness.} In pointing back to creation, the author of Job is also referring to Genesis and, this being the case, a brief interpretation of Genesis will be included in this section as well.

Markel's gratitude was a product of being able to see the goodness of God's creation, the ``glory all around him.'' Markel, however, was unable to see the goodness of creation until his illness and his mother's urging acted as catalysts for him to restore his relationship with God. Similarly, Job seems to mistrust God and the goodness of creation --- which is strikingly similar to Ivan's own mistrust of God. God's speech to Job, with its primary theme of creation, acts as a catalyst for Job to grow in his relationship with God. This growth is represented by a change in the nature of the relationship, from merely \emph{hearing God} to \emph{seeing God}. The way in which this works, from the perspective of ``active love'' and Zosima's interpretation, is that God affirms the goodness of creation and (1) restores his relationship with Job by means of (2b) changing Job's perspective from an earthly one to a divine view of creation. In other words, God's speech shows Job the strangeness, beauty, and inscrutability of creation through the lens of the divine perspective.

\section{The Book of Job: A Trial That Enabled Job to See God}
\begin{quote}
\singlespacing
Then Job answered the Lord: ``I know that thou canst do all things, and that no purpose of thine can be thwarted. `Who is this that hides counsel without knowledge?' Therefore I have uttered what I did not understand, things too wonderful for me, which I did not know. `Hear, and I will speak; I will question you, and you declare to me.' I had heard of thee by the hearing of the ear, but now my eye sees thee; therefore I despise myself, and repent in dust and ashes.\footnote{Job 42:1-6} 
\end{quote}

This passage comes at the very end of the Book of Job and, for that reason, it summarizes the transformation of Job's relationship with God. He has gone from a relationship in which he only heard God with his ears to a relationship in which he sees God with his eyes. Without such a trial, it is not clear that Job would have had this transformation. He clearly knew God well before his trial, as he sacrificed on a regular basis and ``was blameless and upright, one who feared God and turned away from evil.''\footnote{Job 1:1.}. Similarly, he must have been familiar with the religious laws of the time in order to successfully maintain his innocence and righteousness in the face of criticism from his interlocutors. The success of his defense is even verified by God who says that Job's friends have not spoken correctly, as Job has.\footnote{Job 42:7-9.}

Yet the question emerges: if Job was correct in all that he had said of God, how could his suffering enable him to ``see'' God as opposed to ``hearing'' him? Perhaps the answer lies in an analysis of Job's potential error --- an underlying doubt in the goodness of God's creation --- an error that is related to Markel's inability to see God and Ivan's great doubt in the goodness of men.
\begin{quote}
\singlespacing
Let the day perish in which I was born, and the night that said, `A man-child is conceived.' Let that day be darkness! May God above not seek it, or light shine on it. Let gloom and deep darkness claim it. Let clouds settle upon it; let the blackness of the day terrify it.\footnote{\emph{Ibid.}, Job 3:3-5}
\end{quote} In the passage quoted above, Job has called into question the goodness of God's creation and he has moved into ingratitude at God's creation. As Kathryn Schifferdecker affirms in her study of the relationship between creation and Job, this ``lament is radically nihilistic, calling on forces of darkness, chaos, and death to negate light, life, conception, birth, and ultimately creation itself\ldots In the \emph{egocentrism of despair} [emphasis mine], Job closes in upon himself and wills creation, too, to collapse into darkness and chaos.''\footnote{Kathryn Schifferdecker. \emph{Out of the Whirlwind: Creation Theology in the Book of Job}. (Cambridge, Mass.: Harvard Theological Studies, Harvard Divinity School, 2008), 9.} This self-centeredness seems to be \emph{the result of doubt in God's goodness}. Without faith in God's goodness, man is left on his own terms in a world that he must master with his `will and his science', as Ivan's Devil says, and man cannot help but be self-centered.

 How, then, is Job's trust in God restored? The answer to this question comes from God's speech --- after which Job's trust in God is assured.

\begin{quote}
\onehalfspacing
Then the Lord answered Job out of the whirlwind: ``Who is this that darkens counsel by words without knowledge? Gird up your loins like a man, I will question you, and you shall declare to me. ``Where were you when I laid the foundation of the earth? Tell me, if you have understanding. Who determined its measurements --- surely you know! Or who stretched the line upon it? On what were its bases sunk, or who laid its cornerstone, when the morning stars sang together, and all the sons of God shouted for joy?\footnote{Job 42:1-6}
\end{quote}

Given the overpowering nature of God's response, some interpreters have claimed that the ``divine speeches reveal God as a capricious, jealous tyrant who abuses his power'' and who coerces Job into `trusting' him.\footnote{Schifferdecker, 9.} Others suggest that the God's questions are meant to show Job the order and goodness of creation: ``They enable him to realize anew that God establishes order in the cosmos\ldots This order visible in the universe leads Job to trust God even when he does not understand why he suffers.''\footnote{\emph{Ibid.}, 9.} This second line of interpretation is more consistent with Zosima's principles in that God's reference to creation is a way in which God can remind Job of his proper place in creation. The human place in creation is one of the primary themes in the Book of Genesis. It is for this reason that God's speech, in which the majority of his words have to do with creation, ought to be read in the context of Genesis.

\section{Genesis: The Lesson of Humanity's Place in Creation and Relationship With God}
In Genesis it is written that man is made in the `image and likeness of God'\footnote{Genesis 1:26} and this is the interpretative key to understanding man's proper place in creation. A collection of homilies by Pope Benedict XVI, in which he discusses Genesis, are helpful in understanding this concept. In his homilies, Pope Benedict highlights a few key themes which reveal the following: man is given special affection as God's creation; that as a created being, man is entirely dependent on God's goodness for his existence and, therefore, it is man's proper place to praise God; and, finally, Pope Benedict reflects on ``the fall'' and gives an interpretation of ``original sin'' and ``fallenness'' that is especially helpful in understanding how ``the fall'' changed man's relationship to God and to other human beings. 

Pope Benedict sees in the image of God's breathing life into man, as described in Genesis 2:7, a loving and creative God who pours Himself into His own creation. This image of God calls to mind Psalm 139, particularly lines 13-14: ``For it was you who formed my inward parts; you knit me together in my mother's womb. I praise you, for I am fearfully and wonderfully made.''\footnote{Psalm 139:13-14.} God, in this view, loves His creation and cares for each particular individual, so much so that He knits them each in their mother's wombs. The Psalmist continues to say that \emph{because ``I am fearfully and wonderfully made,'' I praise you.} The Psalmist is articulating a relationship with God that is laced with gratitude --- gratitude at having been created is the \emph{cause} of the Psalmist's praise of God. In fact, the relationship articulated here is a paradigmatic case in the biblical witness of what it means to be a created being, entirely dependent on God. Job, too, is aware of his dependence on the Lord: ``Naked I came from my mother's womb, and naked shall I return; the Lord gave, and the Lord has taken away; blessed be the name of the Lord.''\footnote{Job 1:21.} It is with this loving, creative God in mind that Zosima interprets Job.

But the question still remains: if God is such a loving God, why does Job suffer as he does and why does suffering exist in the world? Would such a loving God truly stand for such terrible suffering? This, of course, is the question that plagues Ivan and an answer can be found in the text of Genesis and its account of the ``fall''. If suffering is a product of an initial fracturing of the original relationship between human beings and God, and if after this fracture, suffering has become an inherent part of existence in this world, then suffering is understood in a completely different way than Ivan conceives it. 

In the context of the ``fall'', suffering does not necessarily have to be a means to an end --- a purification process, as Ivan's Devil suggests --- it might just be an unfixable aspect of human existence. In much the same way that a seed must ``fall into the ground and die''\footnote{John 12:24} so that it might bring forth fruit, so too has the world been changed in such a way that suffering is now an inevitable part of life. In this context, suffering is not a means to an end --- rather, suffering has been transformed so that it \emph{can be} fruitful. The ``fall'', as Pope Benedict conceives it, describes exactly a fracturing of human relation with God and with other humans in a `sin-damaged world' in which all human beings are born into suffering and cannot avoid it.\footnote{Ramsey, O.P., Boniface, and Pope Benedict XVI. \emph{In the Beginning: A Catholic Understanding of the Story of Creation and the Fall}, 73.}

\section{The Fall: Doubt Fractures Man's Relationship With God and Fellow Men}
Leading up to his discussion about original sin and the ``fall'', Pope Benedict highlights a very important detail about humans: they are relational and ``live in those whom they love and in those who love them and to whom they are `present'\thinspace''\footnote{\emph{Ibid.}, 72.} He continues to say that to ``be truly a human being means to be related in love, to be of and for\ldots''\footnote{\emph{Ibid.}, 72.} Sin destroys our relationship with God and with other humans. This is because sin is ``a rejection of relationality \ldots Consequently sin is always an offense that touches others, that alters the world and damages it\ldots every human being enters into a world that is marked by relational damage\ldots''\footnote{\emph{Ibid.}, 73.} Job, too, is born into this sin-damaged world. Similarly, Markel's guilt and sin is tied to improper relationality with God and with his nurse and mother. Markel's suffering made him aware of the fractures in his relationships. So too did Job's admonishment from God enable him to overcome his limited, fallen relationship with God and arrive at a higher, more proper relationship. 

This understanding of original sin also gives another way to comprehend suffering: suffering is the result of this sin-damaged world and it enables an escape the problem of \emph{necessary} suffering. Instead, suffering is unavoidable as all are born into the ``sin-damaged world''. This ``sin-damaged world'' is clearly present throughout Dostoevsky's novel in all of its characters and stories: in Zosima's relationship with Afanasy, Mitya's relationship with his father, and the acts of violence of the young boys. Even from a young age, grade school boys are capable of such terrible deeds as stabbing one another with pen-knives and pelting each other with rocks --- and all as a result of damaged relationships. The result of this ``sin-damaged'' world is one in which relationships with oneself are also damaged --- damaged in such a way that there is an all-consuming guilt that makes one ``not want to see paradise.''

Pope Benedict's explanation that sin is a ``rejection of relationality'' is particularly striking when taken together with Ivan's Devil's statement of men building a world for themselves. The Devil's own words bring to mind the original temptation from the mouth of a snake: ``Now the serpent was more crafty than any other wild animal that the Lord God had made. He said to the woman, ``Did God say, `You shall not eat from any tree in the garden'?''\footnote{Genesis 3:1.} In analyzing this line, Pope Benedict is quick to point out that the serpent's first words take the form of a question meant to induce doubt. The first step in sin, as seen in Genesis 3:1, ``is not the denial of God but rather doubt about his covenant\ldots''\footnote{Boniface Ramsey, O.P. and Pope Benedict XVI. \emph{In the Beginning: A Catholic Understanding of the Story of Creation and the Fall}, 66.} In this line of analysis, Job's ``nihilistic'' lamentations can be understood as sinful doubt that is indicative of a fractured relationship with God. Similarly, Ivan's relationship with God also can be understood as fractured in this ``doubtful'' sense --- especially since Ivan's primary doubt is in God's goodness and the goodness of creation. 

Only by restoring a proper relationship with God, restoring the ``sin-damaged world'', is Job able to see God and His glory, in the same way that Markel is able to see God's glory. While suffering remains an inherent part of the world, once this proper relationship has been restored, there is a mysterious way in which God is able to heal the wounds of the suffering. This is how Zosima concludes his analysis on Job: the old grief and pain ``gradually passes into quiet, tender joy; instead of young, ebullient blood comes a mild, serene old age: I bless the sun's rising each day and my heart sings to it as before, but now I love its setting even more, its long slanting rays, and with them quiet, mild, tender memories, dear images form the whole of a long and blessed life --- and over all is God's truth, moving, reconciling, all-forgiving!''\footnote{Dostoevsky, 292.} In accordance with the fact that ``active love'' is a way of life, and not simply a philosophical framework, Zosima does not try to understand how suffering is reconciled --- instead he embraces it as a ``great mystery of life.''\footnote{\emph{Ibid.}, 292.}

In conversation with Ivan, Alyosha is quick to remind Ivan that Christ suffered for the sins of humanity in order that humanity might be redeemed. In Christ, ``the passing earthly image and eternal truth here touched each other. In the face of earthly truth, the enacting of eternal truth is accomplished.''\footnote{\emph{Ibid.}, 292.} The truth in this phrase is that God became human in Christ and accepted death on a cross in order to redeem humanity. In Christ, the earthly truth of man's created nature and dependence on God touches the eternal truth of God's love for humanity. Christ is the personification of man's relationship with God the creator --- in that Christ \emph{is} the union of God's ``image and likeness'' with God. As such, Christ's sacrifice represents both the greatest act of love humanly possible and God's love for creation --- thus creating a path by which creation can be restored from its fallen state. It is important to note, however, that although Christ's sacrifice is indeed redemptive, it must work \emph{within the confines of the ``sin-damaged world'' in order to transcend it.}\footnote{This principle of working within the ``sin-damaged world'' is embodied in John 12:24.} In this way, Christ's sacrifice can be understood as fixing the relationships within the ``sin-damaged world''. 

It is no wonder, then, that Zosima's ``active love'' is a direct counter to Ivan's critique that men cannot love as Christ loved. This love of neighbor, represented in (2) proper perspective, is the act of \emph{restoring one's relationship with others} by means of narrowing the focus to a (2a) local focus and (2b) learning to operate with forgiveness, as opposed to earthly standards of justice.

\chapter{PROPER PERSPECTIVE: LOVING ONE'S NEIGHBOR AS CHRIST LOVED}
	The second principle of Zosima's ``active love'' is characterized, broadly, by (2) proper perspective. This broad perspective can be broken into two smaller sections: (2a) a focus on the local level towards one's own neighbors; and (2b) the ability to operate with forgiveness. With regards to (2a), Zosima gives the most explicit account of ``active love'' to the ``Lady of Little Faith'', Madame Khokhlavov.\footnote{The version of \emph{The Brothers Karamazov}} used in this thesis uses `Khokhlavov' as opposed to `Khokhlavova', which is used in the Norton Critical Edition.} A further account of this (2a) local perspective is given, in this same chapter, in the doctor who loved humanity but hated his neighbor. The second aspect of this proper perspective, (2b) operating with forgiveness, is illustrated in a great episode of Zosima's own life, in the story of the ``Mysterious Visitor''. This second principle, (2b), is closely tied to seeing that ``life is paradise'' and in bringing hope, as Markel's own words draw the explicit connection between paradise and forgiveness.
	
	Proper perspective is the answer to Ivan's great doubt that man can love as Christ loved. Through the practice of (2a) local perspective, Zosima is able to love the ``Mysterious Visitor'' --- and, in fact, a number of other people such as Afanasy and the Karamazov brothers --- as Christ loved, with genuine sincerity, selflessness and empathy. In a focus on the (2b) forgiveness, Zosima is able to encourage true repentance and redemption in the life of a man who lived with tremendous guilt, the guilt of a premeditated murder, for an entire lifetime. \emph{It is not coincidence that Ivan Karamazov is overwhelmed by a similar guilt and a similar inability to forgive himself}. Forgiveness is the key that enables \emph{transformation} as opposed to merely retribution. 
	
	In the development from an immature faith to a mature faith, if (1) gratitude is the first step in repairing the relationship with God, (2) proper perspective is the second step in trusting that man can indeed love as Christ loved by enabling the reparation of local, personal relationships between people. The final step in the maturation of this faith will be (3) in the acceptance of reality: that it is fallen, that it \emph{has been redeemed and transformed} and that, in spite of this, suffering is still an inherent part of earthly existence.
	 
\section{Madame Khokhlavov's ``Little Faith'' and the Search for Immediate Gratitude}
Madame Khokhlavov comes to Zosima seeking the healing of her daughter. Her conversation soon turns towards herself, however, and she explains to Zosima that she is suffering from a lack of faith. She wants to believe in God, but she cannot bring herself to it because her conception of faith is still operating with worldly standards of vindication and reward. Throughout the conversation, two things become apparent: Madame Khokhlavov seeks immediate repayment for her `charity' and she is focused on `humanity' in the abstract. In this way, she is operating with a worldly, broad and na\"{\i}ve faith that is similar to Ivan's. Zosima proposes an answer to Madame Khokhlavov's suffering, but this answer runs counter to her self-centeredness as well as her broad perspective. Zosima focuses on self-giving acts towards one's neighbor, irregardless of receiving gratitude, teaches one to overcome a self-centered faith and a broad, humanitarian, na\"{\i}ve faith.

Madame Khokhlavov pleads to Zosima: ``And yet happiness, happiness --- where is it? Who can call himself happy? \ldots I am suffering, forgive me, I am suffering!''\footnote{\emph{Ibid.}, 55.} He responds: ``From what precisely\ldots Lack of faith in God?''\footnote{\emph{Ibid.}, 55.} Madame Khokhlavov responds: ``Oh, no, no I dare not even think of that, but the life after death --- it's such a riddle! \ldots''\footnote{\emph{Ibid.}, 55.} 

In many ways, Madame Khokhlavov's lack of faith is similar to Ivan's: she dare not even think of doubting God --- (``I dare not even think of that'') and she shifts her focus to the question of the afterlife (like Ivan) and seems to doubt God's love. If she does not doubt God's existence, why else would she doubt the afterlife? She says: ``\ldots this thought about a future life after death troubles me to the point of suffering, terror and fright\ldots''\footnote{\emph{Ibid.}, 55.} Note: the alternative to the afterlife that she poses is that ``I die, and suddenly there's nothing''\footnote{\emph{Ibid.}, 56.}, so her primary concern is whether or not the afterlife exists --- not whether she is going to hell. In this way, she has doubted God's goodness through the lens of the existence of the afterlife. In much the same way, Ivan keeps the afterlife as merely the logical means by which to maintain morality.

 Furthermore, as Ivan is aware of his thirst for life and inability to resolve his torment, Madame Khokhlavov is aware of her thirst for immediate gratitude: ``if there's anything that would immediately cool my 'active' love for \emph{mankind} [emphasis mine], that one thing is ingratitude. In short, I work for pay and demand my pay at once, that is, praise and a return of love for my love. Otherwise I'm unable to love anyone!''\footnote{\emph{Ibid.}, 57.} This is the earthly desire for swift justice that is present in Ivan's na\"{\i}ve faith. Zosima's `answer' to Madame Khokhlavov points to a mature faith, one that endures suffering without earthly reward. The key to developing this sort of faith, as becomes clear in Zosima's `answer', is a continual focus on (2a) the needs of those who are immediately in front of her and (2b) the fact that her reward will not operate within earthly standards of justice.

\section{Zosima's Answer: Seeing Christ in all People}
Zosima replies to her anguish with great compassion, in the same way that he replies to Ivan's great doubt: ``I believe completely in the genuineness of your anguish.''\footnote{\emph{Ibid.}, 56.} This ability to embrace her anguish and her doubt is indicative of great sympathy. This sympathy comes, in part, from Zosima's focus: he is focused solely on Madame Khokhlavov. She has his full attention and feels that she can trust him completely. This is similar to the effect that Alyosha has on those who see him: ``There was something in [Alyosha] that told one, that convinced one \ldots that [Alyosha] did not want to be a judge of men, that he would not take judgment upon himself and would not condemn anyone for anything.''\footnote{\emph{Ibid.}, 19.} This particular characteristic seems to be dependent on Alyosha's ``love of people'' and his ``complete faith in people.''\footnote{\emph{Ibid.}, 19.} Zosima displays these same characteristics in his encounter with Madame Khokhlavov. In this way, Zosima practices the very (2a) local perspective of the ``active love'' that he preaches.

Madame Khokhlavov continues:
\begin{quote}
[Madame K.]: How, how can it [God's existence] be proved? \ldots It's devastating, devastating!

[Zosima]: No doubt it is devastating. One cannot prove anything here, but it is possible to be convinced.

[Madame K.]: How? By what?

[Zosima]: By the experience of active love. Try to love your neighbors actively and tirelessly. The more you succeed in loving, the more you'll be convinced of the existence of God and the immortality of your soul. And if you reach complete selflessness in the love of your neighbor, then undoubtedly you will believe, and no doubt will even be able to enter your soul. This has been tested. It is certain.\footnote{Dostoevsky, 56.}
\end{quote}

Madame Khokhlavov is seeking a simple answer to her doubts, almost as if she expects a single statement from Zosima to cure her. This sort of thinking operates within earthly and self-centered standards. It expects an argument, perhaps even a Euclidean argument, that God exists. As Job and Markel illustrated, however, coming to ``see'' God is not so easy --- and it often involves a great trial. Furthermore, Job and Genesis illustrate another great principle of faith: faith is interpersonal. Job does not plead to an abstract principle for help. No, he pleads to God. Genesis confirms this personal relationship between God and creation and it also reveals the ``fall'' as the source of a ``sin-damaged'' world. Restoring one's relationship with God is one key in the maturation of one's faith and this was seen in the section on gratitude. The second step in this development of faith lies in the restoration of one's relationship with the rest of humanity --- a direct counter to Ivan's great doubt in man's ability to love as Christ loved. 

The process by which one's \emph{faith in God} is restored through \emph{selfless love of neighbor} is somewhat of a mystery. This process is demystified, however, with the insight from Genesis, that man is made in God's image. If man is indeed made in God's image then the idea of `seeing Christ in all people' is no stretch of the imagination and has a biblical reference point in Matthew 25:36-40.\footnote{Matthew 25:36-40 reads: `I was naked and you gave me clothing, I was sick and you took care of me, I was in prison and you visited me.' \ldots `Truly I tell you, just as you did it to one of the least of these who are members of my family, you did it to me.'} The way of life embodied by Zosima's ``active love'', through the second element (2) proper perspective, enable one to see Christ in all people.

\section{Forgiveness: The Divine Justice That Enables Redemption}
This restorative love is not easy, however, and it is easily confused with a general love for humanity. Madame Khokhlavov explains to Zosima that she has a great love for humanity: ``You see, I love mankind so much that --- would you believe it? --- I sometimes dream of giving up all, all I have, of leaving Lise and going to become a sister of mercy.''\footnote{\emph{Ibid.}, 56.} Here Zosima says, not without a bit of humor, ``It's already a great deal and very well for you that you dream of that in your mind and not of something else. Once in a while, by chance, you may really do some good deed.''\footnote{\emph{Ibid.}, 56.} Zosima continues to explain to Madame Khokhlavov that she will make no progress in convincing herself if she merely speaks sincerely in order to be praised for her sincerity. Zosima is so astute in his analysis that she cries out, ``You've brought me back to myself, you've caught me out and explained me to myself!''\footnote{\emph{Ibid.}, 58.}

And here Zosima spells out the difficulty of active love --- it is not na\"{\i}ve, nor does it discount human suffering: ``\ldots active love is labor and perseverance, and for some people, perhaps, a whole science.''\footnote{\emph{Ibid.}, 58.} Instead, it embraces this suffering and accepts reality as it is, without ever losing hope: ``If you do not attain happiness, always remember that you are on a good path, and try not to leave it.''\footnote{\emph{Ibid.}, 58.} Where does this hope come from? Zosima reminds her that she is ``on a good path,'' that is, if she practices this love she will be on a path to heaven. But this is not mere eschatological hope --- a life of forgiveness, as Markel makes clear, \emph{is} paradise. Paradise is the source of this hope and it arises from aspect (2b) in which forgiveness allows for trust in God's goodness, in spite of suffering and without any earthly vindication. Christ's sacrifice enabled redemption --- but according to earthly standards of justice, Christ's victory was in no way a triumph in the worldly sense. This is borne out by the fact that he worked within the confines of the ``sin-damaged world'' in order to transcend it. 

As in Markel's story, an application of love (2a) at the local level is sustained by his hope (2b) trust in God's divine standards of justice, which often manifest themselves at the least expected moments: ``But I predict that even in that very moment when you see with horror that despite all your efforts, you not only have not come nearer your goal but seem to have gotten farther from it, at that very moment --- I predict this to you --- you will suddenly reach your goal and will clearly behold over you the wonder-working power of the Lord, who all the while has been loving you, and all the while has been mysteriously guiding you.''\footnote{\emph{Ibid.}, 58.} In other words, God's standards of justice tend to operate in mysterious ways --- such as the healing that took place in order for Job to love his new children ``when his former children are no more\ldots''.\footnote{\emph{Ibid.}, 292. The line continues: ``Remembering them, was it possible for him to be fully happy, as he had been before...?'' And Zosima answers in the affirmative: ``\ldots it is possible: the old grief, by a \emph{great mystery of human life}, gradually passes into quiet, tender joy\ldots''}

These great moments of grace, while apparently far-fetched, are scattered throughout the novel itself. Zosima is overpowered by grace the night before his duel. Alyosha's faith is restored at the depths of his despair by Grushenka. Mitya's redemptive revelation that he is guilty before all comes on the night of his interrogation. These moments of grace are similar to Job's encounter with God when he has suffered greatly and is at his wit's end in justifying himself. These great trials call to mind Christ's own words, quoted from Psalm 22: ``My God, my God, why have you forsaken me? Why are you so far from helping me, from the words of my groaning?''\footnote{Psalm 22:1-3.} In the midst of this great despair, the Psalmist later proclaims: ``From the horns of the wild oxen you have rescued[d] me. I will tell of your name to my brothers and sisters; in the midst of the congregation I will praise you\ldots''\footnote{Psalm 22:21-22.} Like Markel and like Christ, the Psalmist's great joy lies not in some earthly triumph but in the mysterious, eternal dominion of God. 

Thus, by incorporating (2) proper perspective, a substantial counter is posed to Ivan's doubt in man's ability to love as Christ loved. It is fitting that one of the greatest test cases of ``active love'' comes from Grushenka, in her genuine and sincere compassion for Alyosha. Such acts of love do not come only from the holiest people, but also the lowliest. For this reason, ``active love'' emphasizes humility and honesty with oneself. In an honest accounting of oneself, it is impossible \emph{to not} find flaws: and this is why Ivan, with his great desire for retribution, struggles so mightily with his guilt over his father's murder.

\chapter{ACCEPTING REALITY: MAN'S FALLEN NATURE AND REDEMPTION}
The ``sin damaged world'' is helpful in this analysis in that it reveals the nature of fallenness --- the damage is done by means of fractured relationships. The first is a fractured relationship with God --- as expressed in Ivan's distrust in God, the same distrust the mirrors Job's doubt in God's goodness. The second is a fractured relationship with others --- this is expressed in Ivan's doubt that man can love as Christ loved. The third is a fractured relationship with oneself --- as expressed in Ivan's struggle to embrace both the guilt over his father's murder and forgiveness. In this struggle, Ivan's great desire for retribution is at odds with his own guilt. The final obstacle for Ivan's faith in its maturation is his inability to accept the forgiveness for his sinfulness that was enabled by Christ's death.

Zosima's ``active love'' speaks to Ivan's final obstacle in two steps: first, one must accept man's fallen nature and become guilty before all; and second, one must also accept the forgiveness of Christ. This principle of (3) acceptance of reality is embodied in Zosima's beloved John 12:24: ``Very truly, I tell you, unless a grain of wheat falls into the earth and dies, it remains just a single grain; but if it dies, it bears much fruit.''\footnote{John 12:24.} As this passage, and Christ's death, suggests: the world is fallen in such a way that death is a necessary component in the development towards life. It is this principle that ``active love'' invokes as a \emph{redemptive and transformative} process.

\section{Guilty Before All: Markel's Message Lives Through Zosima}
The idea that one must be ``guilty before all'' is an idea that stuck with Zosima from the time of his brother's death. This idea is fleshed out more fully by Zosima in his comments during Ivan's first philosophical argument at the monastery. During this conversation, Zosima explains that true reformation and redemption lies in one's conscience becoming self-aware: ``\ldots real punishment, the only real, the only frightening and appeasing punishment \ldots lies in the acknowledgement of one's own conscience.''\footnote{Dostoevsky, 64.} He continues to explain the importance of forgiveness and true acknowledgement of sin: ``If anything protects society even in our time, and even reforms the criminal himself and transforms him into a different person, again it is Christ's law alone, which manifests itself in the acknowledgement of \emph{one's own conscience} [emphasis mine].''\footnote{\emph{Ibid.}, 64.} The process of acknowledging one's own guilt is what is meant in the phrase ``I am guilty before all'' --- as is shown in Zosima's belief that only when one acknowledges guilt internally will one publicly confess it to society.\footnote{\emph{Ibid.}, 64.}

As Zosima emphasizes in his conversation with Madame Khokhlavov, the key in this process is to be honest with oneself and to acknowledge one's guilt: ``Above all, avoid lies, all lies, especially the lie to yourself. \ldots do not even be very frightened by your own bad acts.''\footnote{\emph{Ibid.}, 58.} In Zosima's own life, the realism of not lying to himself comes in his recognition of his guilt in beating Afanasy: ``Indeed, how did I deserve that another man, just like me, the \emph{image and likeness of God} [emphasis mine], should serve me? This question pierced my mind for the first time in my life.''\footnote{\emph{Ibid.}, 298.} This acknowledgement of a personal sin leads Zosima to remember immediately Markel's proclamation of being ``guilty before all.'' This might seem to be an unusual jump. from acknowledging a particular sin to such a general sin. but in the context of a ``sin-damaged world'', this responsibility for all makes much more sense: because of man's fallen nature, he is continually sinning against God, against others, and against himself. This sin manifests itself in fractured relationships by means of: doubt, insincerity, hatred, irrational repugnance, personal pride and in a litany of other ways. Once this guilt is accepted, the next step in the process of redemption and transformation is the ability to accept forgiveness for this sin. This ability to accept forgiveness is fundamental to Markel's conception of paradise.

\section{Accepting Forgiveness: The Final Key in the Transformation}
Zosima tells Ivan that the problem will never be resolved. The fact that Ivan cannot resolve this dilemma himself reveals, to a certain extent, Ivan's need of an external force to resolve it. This external force is Christ and the forgiveness that his death brought --- ``he can forgive everything, forgive all \emph{and for all}.''\footnote{\emph{Ibid.}, 246.} Because of his desire for retribution and earthly justice, Ivan is unable to accept Christ and the consequent forgiveness as a resolving force for his dilemma. While it might seem odd that Ivan cannot accept such a force, there is precedent in the novel in the mysterious visitor who also cannot accept forgiveness.

The mysterious visitor is an example of a prideful conscience attempting to evade its own standards of justice. After committing the murder of his beloved, he attempts to soothe over his conscience by justifying it as an act of love, through philanthropy and in a fruitful marriage. Yet his conscience continually tugs at him and draws him to Zosima. After spending significant time with Zosima, he confesses to the murder and feels the need to confess publicly. The encounter culminates in the visitor resisting the urge to confess to Zosima (and even contemplating murdering Zosima) while Zosima reads two passages from scripture: John 12:24 and Hebrews 10:31, ``It is a fearful thing to fall into the hands of the living God.''\footnote{Zosima articulates ``active love'' with similar terminology: ``I am sorry that I cannot say anything more comforting, for active love is a harsh and fearful thing compared with love in dreams.'', 58.} The first passage, in alluding to Christ and his death, points to the need for external grace in forgiveness. The second passage alludes to the great difficulty in submitting one's will to God. Once the will submits, after it has been prepared and transformed through the practice of ``active love'', it will be ready for the great moment of grace. This moment for the mysterious visitor manifests itself in tender joy that overwhelms him after his public confession

The moment of grace for Job manifests itself in ``seeing'' God and growing in love for his new children. The moment for Zosima manifests itself in an overflowing joy that permeates his existence.\footnote{\emph{Ibid.}, 292.} This joy is a product of a life of ``active love'' which has enabled Zosima to restore his relationship with God, with others and himself. Furthermore, it has enabled him to acknowledge his guilt before all, to acknowledge the fallenness of human nature and the inherent suffering in life, while also acknowledging that Christ has bought eternal salvation and redeemed men. ``Life is paradise'' for Zosima precisely because he has acknowledged this guilt and forgiveness. In so doing, he has grown into a mature faith which adapts to reality, accepts reality and has his sight opened to the great goodness of God's creation (much the same way as Markel does). In this way, suffering is not a test of faith for Zosima so much as another reality of the ``sin-damaged world'' that he must adapt to and embrace. 

\section{Conclusion: Coming to Know a Quiet and Tender Joy}
After the mysterious visitor confesses his crime, he is swept up in a feverish ecstasy, he passes away and the town blames Zosima for his illness and his death. Yet this does not bother Zosima because he knows the truth of the great reconciliation: ``I was silent and indeed rejoiced at heart, for I saw plainly God's mercy to the man who had turned against himself and punished himself.''\footnote{\emph{Ibid.}, FIND PASSAGE IN BOOK.} This is \emph{precisely} Ivan's great torment and the cause of his insanity: he is punishing himself over the guilt of his father's murder.  In the end, despite his great facade of philosophical argument, Ivan cannot escape his own conscience. His love for life has prevailed over the meaning of life and he has begun his transformational quest in learning to forgive himself. If he truly believed his words and truly doubted the existence of God, morality would be a mere mirage --- but something deep within Ivan himself knows this not to be the case.

Faith that has matured through the process of ``active love'', once it has been stripped of its `triumphalistic expectation', is able to see God's glory moving imperceptibly throughout the world --- softening, healing and reconciling.\footnote{\emph{Ibid.}, FIND THIS PASSAGE IN TEXT.} The greatest representation of this faith is seen in Alyosha at Ilyushechka's funeral and in his speech to the young boys: he is hopeful and joyful, tender and saddened, but patiently laboring in the restoration of the ``sin-damaged'' world.

\pagebreak
