\tableofcontents
\chapter{INTRODUCTION}

\pagestyle{myheadings}
\markright{\hfill Griggs \hspace{1 mm}}

\label{introduction}

\section{Statement of Thesis and Layout of Work}

In Dostoyevsky's \emph{Brothers Karamazov}, one of the major themes throughout is a particular articulation of the problem of suffering. In the novel, it is articulated most acutely by Ivan Karamazov. In the chapters "Rebellion" and "The Grand Inquisitor", Ivan tells of the suffering of innocent children at the hands of Turkish soldiers and the death of a young boy at the hands of a vengeful general. Ivan sees the suffering of these innocent children as a fundamental injustice that can never be accounted for and this leads him to rebel against God. Ivan has a ''childlike conviction that the sufferings will be healed and smoothed over, that the whole offensive comedy of human contradictions will disappear like a pitiful mirage, a vile concoction of man's Euclidean mind ... [eternal justice] will suffice not only to make forgiveness possible, but also to justify everything that has happened with men.''\footnote{Dostoyevsky, Fyodor. Translated by Richard Pevear, and Larissa Volokhonsky. \emph{The Brothers Karamazov: A Novel in Four Parts with Epilogue}. New York: Farrar, Straus and Giroux, 2002, 235.} 

Yet, even with this conviction, he simply does not accept God's world. We learn further that Ivan's rebellion is the result of his belief that men were inevitability bound to bring injustice into the world because ''Christ's love for people is in its kind a miracle impossible on earth.''\footnote{ibid, 237.} This is the ultimate source of cynicism lying behind Ivan's rebellion. Like the Grand Inquisitor, he cannot understand why God gave men freedom when men were bound to bring such suffering into the world: ```Man was made a rebel; can rebels be happy?'''\footnote{ibid, 251.} Furthermore, he cannot understand why suffering of innocents seems to be an \emph{means to an end} --- that suffering is necessary in order to enable holiness. This concern is expressed most clearly in the Devil's description of Job: ``how many souls had to be destroyed ... in order to get juts one righteous Job ... ''\footnote{ibid, 648} Ultimately, Ivan asks, why not renounce a God who necessitates such suffering and depend on man with ``his will and his science no longer limited''\footnote{ibid, 649.} to replace his heavenly hope with an earthly paradise.

Thus, we have in Ivan's words an articulation of the problem of suffering based in the belief that men cannot love as Christ loved and that suffering is necessary for redemption. This is a belief that is diametrically opposed to the beliefs of Father Zosima who preaches a Christian way of life based in ``active love''. The teaching of ``active love'' has three particular characteristics: (1) gratitude and appreciation for God's creation; (2) a proper perspective that is characterized by (2a) looking at the local, interpersonal level instead of the global, generalized level and (2b) realizing that the ultimate end is in heaven and not earth; and (3) the ability to accept reality as it is, particularly man's fallen nature and the inherent suffering in life. The ultimate result of ``active love'' is a redemptive transformation that enables one to confront and transcend Ivan Karamazov's cynicism and rebellion. 

Each of the three characteristics has its own role to play in this transformative process. Ivan's rejection of God is not a purely rational problem, in fact, it is most accurately understood as an interpersonal problem. This being the case, (1) gratitude and appreciation for God's creation is essential because it enables one to restore a proper relationship with God himself. Zosima uses Job as the exemplar of this gratitude, in the belief that Job was actually blessed with the opportunity to glorify God through overcoming the temptation to rebel, brought on by his suffering. 

Another component of Ivan's rebellion is that it is based at an abstracted level from reality. It is not based so much in his own suffering as it is in the suffering of the children that he has read about in newspapers. In order for suffering to enable a transformation, ``active love'' teaches (2) proper perspective by looking at the (2a) local level and by realizing that (2b) heaven is the ultimate end. When one looks at the local level, Zosima teaches, there is a way in which one learns to love as Christ loved. Instead of loving humanity, ``active love'' teaches one to love the people as individuals. 

Finally, Ivan's rebellion reflects an immature faith in that it cannot come to grips with the fact that there is suffering in the world and that man's nature is fallen. Zosima's ``active love'' wholeheartedly (3) accepts that these are realities and it does not try to avoid them or work around them. Instead, it embraces these realities and transforms men to cope with these realities through divine grace --- in much the same way that Christ embraced death on a cross in order that men could be redeemed. By wholeheartedly embracing these harsh realities, ``active love'' enables one to be transformed by the harsh realities in such a way that life becomes joyful.

These are lessons that are learned in the lives of the main characters of the novel: Zosima, Ivan, Alyosha and Dmitry. Each, in their own way, has a great encounter with suffering. Zosima suffers the loss of his beloved brother. Ivan struggles with the suffering of innocents and is driven to hallucination by the guilt of his father's murder. Alyosha rebels against his God for a brief moment at the `embarrassment' of Zosima's bodily corruption and Mitya is agonized by his actions and his false sentence. Each character has a particular response to suffering and these responses are characterized by a development from an immature faith to a mature faith. The principles of ``active love'' can be seen in the transformation of each of the characters, whether it be Alyosha's movement away from a faith that seeks swift earthly justice or Ivan's struggle with his own fallenness and guilt. The end product of ``active love'' is that despite the great depths of man's depravity, life is joyful and must be joyful. In the words of Dmitry, describing life in the mines of Siberia: ``we'll be in chains, and there will be no freedom, but then, in our great grief, we will arise once more into joy, without which it's not possible for man to live, or for God to be, for God gives joy, it's his prerogative, a great one.''\footnote{ibid, 592.}

From where do these principles of ``active love'' arise? Alyosha's biographical account begins with an account of Zosima's older brother, Markel, who at a young age was turned to atheism and then reverted back to belief on his deathbed. Markel's death and his final words laid the seed in the teaching of ``active love'' that Zosima makes explicit in his life as a monk.

	\section{Zosima's Brother: The Beginning of ``Active Love''}
	
	In his biographical account of Father Zosima's life, Alyosha begins with the story of Zosima's brother, Markel. Markel befriends a young philosophy student who has been banished from Moscow for ``freethinking.''\footnote{ibid, 287.} After this encounter with the young ``freethinker'', there is a noted change in Markel in that he does not fast and he says that ``there isn't any God.''\footnote{ibid, 287.} Not long after this, however, in the sixth week of Lent, Markel became terminally sick and soon realized the nature of his sickness. Soon, a spiritual change took place in Markel. His faith was restored and even strengthened. He began speaking strangely, saying things such as: ``life is paradise, and we are all in paradise, but \emph{we do not want to know it} [emphasis mine]'' and ``that verily each of us is guilty before everyone, for everyone and everything.''\footnote{ibid, 288-289.} These sayings culminate in Markel's grandest admission of sin and gratitude: ``there was so much of God's glory around me: birds, trees, meadows, sky, and I alone lived in shame, I alone dishonored everything, and did not notice the beauty and glory of it all.''\footnote{ibid, 289.} The last words mentioned in this section that are attributed to Merkel show a theological definition of heaven that is rooted in forgiveness and love that overcomes Merkel's deep sense of guilt: ``Let me be sinful before everyone, but so that everyone will forgive me, and that is paradise.''\footnote{ibid, 290.}.
	
	Markel has been transformed by his terminal illness in such a way that he is able to see that ``life is paradise''. This transformation is evident when he says that ``we do not want to know it'' --- if not for his own illness, he would not be able know, nor would he want to know, this paradise. Without his own illness and imminent death, he would not be able to notice the ``glory of God'' all around him. Markel is so thankful for the creation around him. This, then, reveals that suffering can be transformative in such a way that it enables one to see God's glory and joy. This is (1) gratitude and it is the first principle of ``active love'' that will be explored throughout this thesis.
	
	Markel's suffering is transformative in another sense, too: it leads him to a great sense of awareness of his guilt and responsibility towards others: ``each of us is guilty before everyone.'' This responsibility is seen in the very personal and local interaction that he has with his nurse who is lighting a candle in front of an icon in his room. Previously, Markel had treated her harshly in refusing to let her pray and light the candle in front of an icon. His illness, however, causes him to change his ways: ``Light it, light it, dear, I was a wretch to have prevented your doing it.''\footnote{249, garnett} Markel has a similar transformation in his treatment of his mother. His statements of being responsible ``for all'', then, should be understood as having a local and particular application. Markel is not talking of a broad notion of a ``love of humanity''. Instead, he is talking of a love for humanity that is expressed through encounters with the people right in front of him. Furthermore, notice that Markel's perspective is focused on heaven, the ultimate end, through the lens of this local application of love. He recognizes that he has sinned against those around him, particularly his mother and his nurse, and they have forgiven him. This leads him to ask: ``Am I not in heaven now?'' This is (2) proper perspective that is focused (2a) locally and (2b) on heaven, the second element of ``active love''.
	
	Finally, Markel is struck with an overwhelming sense of guilt, a feeling that he has sinned against everyone, including God and His creation. There is a deeper, more personal way in which this guilt shows itself, a very sickly way in which guilt causes men to bring suffering upon themselves. They feel so guilty that they \emph{``do not want to know'' the paradise before their very eyes}. Markel's acceptance of this guilt and his `sins' is, perhaps, a recognition of the fallenness of man's nature. Man is guilty before God and he dishonors God's creation because he is fallen.  Similarly, man is guilty before all because he is fallen and focused solely on himself --- as opposed to focusing on his neighbors, as the Bible instructs and ``active love'' teaches.  Markel's sickness leads him to focus on his mother and his nurse, to be sensitive to their suffering and his poor treatment of them. This the third and final element of ``active love'', (3) embracing reality from the depths of the depravity of man to the heights of Christ's forgiveness and transformative sacrifice. 
	
	Once Markel accepts this guilt and \emph{accepts} forgiveness, he becomes joyful --- and this is the great mystery behind ``active love'', the joy that follows and comes often at the darkest moments in one's existence. The joy that overwhelms Markel is the interpretative key that Zosima will use in his teaching of the book of Job. Perhaps it is not coincidence that Zosima's teachings on Job are contained in the very next section after the story of Markel.
	
	
	\chapter{GRATITUDE: AFFIRMING THE GOODNESS OF CREATION}
	Alyosha's biographical sketch continues with Zosima's teachings on the Book of Job, a Biblical text that is central to his teaching of ``active love''. We learn that Job was the first Biblical text that Zosima felt that he clearly understood from a young age. In Alyosha's words, a brief paraphrase of Job is given in which it becomes clear that Zosima sees Job as God's beloved servant: ``And have you seen my servant Job?'' God asks [Satan].\footnote{ibid, 291.} Further in the account, Zosima speaks of ``scoffers and blasphemers'' who asked why the Lord would allow His saint to suffer so greatly.\footnote{ibid, 292.} In a tone that foreshadows Ivan's devil, these ``scoffers'' go so far as to mock God as saying: ``See what my saint can suffer for my sake!''\footnote{ibid,292.} As if to ask, as Ivan does, is this suffering really necessary to prove Job's righteousness?
	
	Zosima answers this question by saying that what \emph{is actually} happening in the story of Job is that the trials of Job are not so much to prove Job's righteousness as they are an opportunity for Job to affirm the goodness of God's creation. God looks praises his creation saying ``That which I have created is good.''\footnote{ibid, 292.} And, Zosima says that Job is praising God and that he serves not only God, but also serves all of creation and all generations.\footnote{ibid, 292.} Furthermore, for Zosima and his teaching of ``active love'', Job illustrates the mystery at the heart of ``active love'' that is the way in which suffering and grief gradually pass into quiet, tender joy.\footnote{ibid, 292.}
	
	Markel's gratitude was a product of being able to see the goodness of God's creation, the ``glory all around him''. As Zosima tells it, one of the fundamental lessons of Job is that Job is given the opportunity to re-affirm the goodness of God's creation by enduring his trials. Markel was unable to see the goodness of creation until he underwent his own trial. Job, too, seems to suffer from an inability to fully comprehend the goodness of creation and, furthermore, his own faith undergoes a period of growth through his suffering and his encounter with God.
	
	\section{The Book of Job: A Trial That Enabled Job to See God}
	\begin{quote}
	\onehalfspacing
	Then Job answered the Lord: ``I know that thou canst do all things, and that no purpose of thine can be thwarted. `Who is this that hides counsel without knowledge?' Therefore I have uttered what I did not understand, things too wonderful for me, which I did not know. `Hear, and I will speak; I will question you, and you declare to me.' I had heard of thee by the hearing of the ear, but now my eye sees thee; therefore I despise myself, and repent in dust and ashes.\footnote{Edited by May, Herbert G., and Bruce Manning Metzger. ''The Book of Job.'' In The \emph{New Oxford Annotated Bible with the Apocrypha: Revised standard version, containing the second edition of the New Testament and an expanded edition of the Apocrypha}. New York: Oxford University Press, 1977. 613-655, Job 42:1-6} 
	\end{quote}

	The words that reveal most clearly come in the passage above, namely Job's statement that he no longer only hears God, but he also sees God. In this passage, too, Job reveals his own ignorance as well as his own pride in `uttering things that he did not understand'. Perhaps Job's greatest `sin' is that he comes to doubt God's goodness --- and consequently the goodness of God's creation. 
	
	\begin{quote}
	\onehalfspacing
	``Let the day perish in which I was born, and the night that said, `A man-child is conceived.' Let that day be darkness! May God above not seek it, or light shine on it. Let gloom and deep darkness claim it. Let clouds settle upon it; let the blackness of the day terrify it.''\footnote{ibid, Job 3:3-5}
	\end{quote}
	
	In the passage quoted above, it is certainly plausible that Job has called into question the goodness of God's creation. Furthermore, his lamentation reveals a self-centered focus --- the exact of the externally-looking gratitude that Markel exhibits. This ``lament is radically nihilistic, calling on forces of darkness, chaos, and death to negate light, life, conception, birth, and ultimately creation itself... In the egocentrism of despair, Job closes in upon himself and wills creation, too, to collapse into darkness and chaos.''\footnote{Schifferdecker, Kathryn. \emph{Out of the Whirlwind: Creation Theology in the Book of Job}. Cambridge, Mass.: Harvard Theological Studies, Harvard Divinity School :, 2008, 9.} This self-centeredness seems to be \emph{the result of doubt in God's goodness}. Without faith in God's goodness, man is left on his own terms in a world that he must master with his `will and his science', as Ivan's Devil says, and man cannot help but be self-centered. This, however, clashes with Mitya's great, ecstatic proclamation that ``man cannot live without joy''. And, if Mitya's statement is true, gratitude for God's creation is necessary for joy.
	
	This next part will talk about how God convinces Job that there is order rand creation is good. I have some more schifferdecker for this part.
	
	
