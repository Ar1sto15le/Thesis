\chapter{INTRODUCTION}

% \pagestyle{plain}

\label{introduction}

\section{What is the Problem of Suffering?}
The problem of suffering is essentially as follows: if there is a good and omnipotent God he would not allow suffering in the world; there is suffering in the world; therefore, there is no God. This, at least, is the philosophical argument in its proper form. It is a common argument against God's existence and, consequently, it also has a philosophical counterpart in the free will defense. Alvin Plantinga has a particularly strong free will defense which can be summarized as such: a world with human free will is most desirable; God cannot cause humans to do what is right without violating free will; therefore, it is logically possible that evil can exist in the world while at the same time God exists. Purely from a logical point of view, both arguments are worth analyzing. What is important to note, however, is the particular emphasis on suffering which is often equated with evil, as if this necessarily follows --- but, in fact, it does not necessarily follow that suffering --- even of innocent children --- is evil.

This impulse to equate suffering of the innocent with evil is, more than anything, a visceral response that, at a glance, seems to be a product of more modern times. Using the book of Job as a proxy for their takes on the problem of suffering, something that becomes apparent is that Gregory the Great and Aquinas, two great representatives of Christianity for their times periods, viewed suffering in a redemptive light. Suffering, as it is described in their commentaries on Job, is meant to purify the soul and bring the soul closer to true happiness, which for Aquinas is actually defined by contemplation of God --- not material wealth, earthly joy or even health, but contemplation of God. This redemptive view of suffering is one that clearly depends on belief in the afterlife to be convincing, but assuming this, it is a strong and logically coherent argument.

The tendency to equate innocent suffering with moral evil is such a visceral response that it is worth asking: is the logical argument really the problem? Further, is God's existence really what's at stake? Or, to put it bluntly, is it simply a matter of not accepting the world as God has made it? This might be a more accurate portrayal of the situation --- one in which we simply cannot accept that God would allow suffering of innocent children. This is a view that is expressed concretely in the character of Ivan Karamazov in Dostoevsky's \emph{Brothers Karamazov} who simply ``cannot accept” God's world --- despite an unwavering belief in God's existence. Another illuminating aspect of Ivan Karamazov is that he realizes that, logically, God is the reason that ties together morality and that God, in fact, is the very source of his great compassion for the suffering of the innocent.

The suffering of the innocent evokes a strong emotional response and it can certainly be a tremendous obstacle to adopting belief in the Christian God. For this very reason, however, it is one of the most important questions to overcome in order to establish a strong and mature faith. Perhaps it is for this reason that it is featured so prominently in Dostoevsky's novel --- ``These fools could not even conceive so strong a denial of God as the one to which I gave expression.”\footnote{Osborn, Ronald. ''Beauty Will Save the World: Metaphysical Rebellion and the Problem of Theodicy in Dostoevsky's Brothers Karamazov.'' Modern Age 54, no. 1-4 (2012). \url{http://www.academia.edu/2512653/Beauty_Will_Save_the_World_Metaphysical_Rebellion_and_the_Problem_of_Theodicy_in_Dostoevskys_Brothers_Karamazov} (accessed December 7, 2013), endnote 13.} This is something that he wrote in his letters as he was writing the novel. But the \emph{Brothers Karamazov} is an expression of Christianity attempting to overcome this problem. It does this through the story that unfolds and in the presentation of the novel an `argument' emerges for Christianity and for Zosima's active love. I say `argument' because it is not really an argument so much as an expression of a way of life that touches the whole human person.

The Christian narrative, as it is expressed in the Gospels, and the accompanying Christology is particularly poignant on this point: Christ became fully man and he suffered death on a cross. He, too, cried at the death of loved ones as he did at the tomb of Lazarus and he, himself, felt the pains of death. What greater illustration of our inability to avoid suffering is there than this simple fact: Christ could not avoid death on a cross. This fact alone seems to prove this point: suffering is now a permanent part of our worldly existence. It may not have been necessary to human existence before the `fall', but it is now something that we cannot avoid --- even Christ could not avoid it. Thus, suffering has a very integral part in the Christian narrative. Suffering plays a part in our understanding of ourselves in relation to God, in understanding our purpose in this world and in our daily existence. It is a problem that touches everyone in some form or another and, for this reason, it is an issue that is worth discussing.

\section{Redemptive Suffering: Aquinas' Answer to the Problem of Suffering}
Aquinas' commentary on Job begins with his acknowledgement of Gregory the Great's moral commentary on Job. In this commentary, Gregory starts from the perspective that the ``lord in loving-kindness permitted that [evil works] to be done [to Job]... for when the enemy had got leave to have him with the purpose of destroying him, by his temptations he effected that his merits were augmented.''\footnote{Gregory the Great. \emph{Morals on the Book of Job}. pp. 7.} In other words, Job is a book of moral teachings that will enable us to grow closer to God and the inflictions of this life are both tests and opportunities for spiritual growth. For this reason, Gregory operates with a mystical reading of Job. We see a similar perspective in Aquinas who understands that the ``whole intention of this book [Job] is directed to this: to show that human affairs are ruled by divine providence using probable arguments.''\footnote{Aquinas, Thomas. \emph{Commentary on the Book of Job}. Prologue, pp. 3.}

Both of these commentaries, taken together, offer quite a different starting perspective than that of Ivan, whose modern view is predominantly concerned with a higher justice being found on earth. Aquinas and Gregory advocate the complete opposite initial interpretative assumption: that the ultimate end of man is to be found in heaven, not on earth. Aquinas, in particular, articulates this view to such an extent that the modern question of whether or not evil/suffering contradicts the existence of God does not even come across in his work. Eleonore Stump explains in her essay, Aquinas on the Sufferings of Job. that: ``... Aquinas sees the problem in the book of Job differently. He seems not to recognize that suffering in the world, of the quantity and quality of Job's, calls into question God's goodness, let alone God's existence. Instead Aquinas understands the book as an attempt to come to grips with the nature and operations of divine providence.'' The question for Aquinas is not whether suffering/evil in the world means that God cannot exist. Instead the question is: how does suffering/evil fit into God's providential work? 

\section{The Inadequacy of Analytic Philosophy in Solving Suffering}
We see in Ivan and Job a complex and nuanced faith. Both have the theoretical and philosophical strength to defend themselves and their arguments. Job eventually comes to realize that this is not enough --- he must come to see God, not only hear him. Similarly, despite Ivan's greatest attempts to be purely rational, he cannot escape the innate desire to live and he acknowledges this. Acknowledging this innate, irrational desire, implies that Ivan is recognizing another form of wisdom outside of reason alone. Yet, Ivan still fancies himself to be like a Euclidean geometrician --- something, perhaps, akin to an analytic philosopher.

The philosopher Eleonore Stump, in her work \emph{Wandering in Darkness: Narrative and the Problem of Suffering}, discusses the analytic Anglo-American tradition of philosophy. She brings out two pertinent points to our discussion in her Philosophy and Narrative chapter: that analytic philosophy does not do justice to interpersonal, right-brained modes of thought and that the problem of suffering is inherently an interpersonal problem. ``In particular, in its emphasis on left-brain mediated pattern-processing, philosophy in the Anglo-American tradition has tended to leave to one side the messy and complicated issues involved in relations among persons.''\footnote{Stump, Eleonore. ''Philosophy and Narrative.'' \emph{In \emph{Wandering in Darkness: Narrative and the Problem of Suffering}}. Oxford: Oxford University Press, 2010. Oxford Scholarship Online, 2010. doi: \url{10.1093/acprof:oso/9780199277421.003.0002}, 3.} She continues this discussion by pointing to the fact that analytic philosophers, in dealing with interpersonal issues, do not draw their cases from complex real life issues nor ``the world's great literature.''\footnote{ibid, 3} She continues: ``Personal relations, however, are at the heart of certain philosophical problems. Central to the problem of suffering in all its forms, for example, is a question to which a consideration of interpersonal relations is maximally relevant: could a person who is omnipotent, omniscient, and perfectly good allow human persons to suffer as they do?''\footnote{ibid,4.}

It is for this very reason that Stump suggests a marriage between the two intellectual disciplines of literature and philosophy. This marriage, she suggests, would address the ``shortcomings of analytic philosophy while preserving its characteristic excellences.''\footnote{ibid,4} Furthermore, and most importantly, Stump presents a strong argument that ``there are things to know which can be known through narrative but which cannot be known as well, if at all, through the methods of analytic philosophy.''\footnote{ibid, 4} Note that analytic philosophy is characterized by: being ``in the habit of categorizing by means of sets of abstract properties, and the labels it uses in categorizing are designations of summarizing those properties. So, for example, Descartes' theory of knowledge shares with various contemporary theories of knowledge the properties of taking knowledge to be built up on a foundation, of supposing the foundation to be basic beliefs...''\footnote{Stump, Eleonore. ''Narrative as a Means of Knowledge: Francis and Dominic.'' \emph{\emph{Wandering in Darkness: Narrative and the Problem of Suffering}}. Oxford: Oxford University Press, 2010. Oxford Scholarship Online, 2010. doi: \url{10.1093/acprof:oso/9780199277421.003.0003}, 4.} This description and its comparison to Descartes sounds strikingly familiar to the character of Ivan Karamazov and his desire to build upon Euclidean logic. It is not surprise, then, that Descartes' criterion for certainty is the very same criterion of ``clear and distinct'' that is present in Euclid's geometry.

\section{The Nature of Faith}
Faith is a nuanced concept in that it encompasses both rational understanding, something similar to philosophical assent, and a sense of giving oneself to, or trusting in, the person of Christ. This nuance becomes clear in the translation of Isaiah 7:9, as Pope Emeritus Benedict explains in his \emph{Introduction to Christianity}: ``A more literal translation would be, ``If you do not believe [if you do not hold firm to Yahweh], then you will have no foothold (Is 7:9). The one root word '….mn (amen) embraces a variety of meanings whose interplay and differentiation go to make up the subtle grandeur of this sentence. It includes the meanings truth, firmness, firm ground, ground, and furthermore the meanings of loyalty, to trust, entrust oneself, take one's stand on something, believe in something; thus faith in God appears as a holding on to God through which man gains a firm foothold for his life. Faith is thereby defined as taking up a position, as taking a stand trustfully on the ground of the word of God. The Greek translation of the Old Testament ... transferred the above-mentioned sentence onto Greek soil not only linguistically, but also conceptually by formulating it as ''If you do not believe, then you do not understand, either.''\footnote{XVI, Pope Benedict. \emph{Introduction to Christianity}. San Francisco: Ignatius Press, 2004,69.}

We now have a helpful paradigm for understanding Job's second response which highlights, perhaps, a greater sense of trusting in God and not only intellectually understanding God, as the Greek translation highlights. In Job's belief, we can see a model of a `system of meaning'\footnote{Williams, Rowan. \emph{Dostoevsky:Language, Faith, and Fiction}. London: Continuum, 2009, vii.} that encompasses more than a single methodology of thought. A helpful way of understanding this sort of belief is through Ratzinger's answer to the question, ``What is belief really?.'' He says that it ``is a human way of taking up a stand in the totality of reality, a way that cannot be reduced to knowledge and is incommensurable with knowledge; it is the bestowal of meaning without which the totality of man would remain homeless, on which man's calculations and actions are based, and without which, in the last resort he could not calculate and act, because he can only do this in the context of a meaning that bears him up.''\footnote{XVI, Pope Benedict. \emph{Introduction to Christianity}, 72.} Can a child's belief in the love of her mother, or a husband's love for his wife, really be described with philosophical framework alone?

A philosophical solution to theodicy is not going to have the completeness of a solution which encompasses more of the human experience and the human imagination. In discussing the inadequacy of understanding the world through one means only, Rowan Williams says that ``... no system of perceiving and receiving the world can fail to depend upon imagination, the capacity to see and speak into and out of a world that defies any final settlement as to how it shall be described.''\footnote{Williams, Rowan. \emph{Dostoevsky: Language, Faith and Fiction}, viii.} Thus, we turn to scripture --- which, in the Catholic interpretation, is overflowing with meaning that is understand through philosophy, but also poetically --- and we are also justified in turning to films and novels to better understand Job's problem, theodicy and God's response to Job.

\section{The Necessity of Narrative}
Dostoevsky's Brothers Karamazov is one novel that is especially thorough in its portrayal of theodicy and in its `answer'. In this novel, we have ``Dostoevsky's most compelling and sympathetic rebel against faith'' in Ivan Karamazov.\footnote{Osborn, Ronald. ''Beauty Will Save the World: Metaphysical Rebellion and the Problem of Theodicy in Dostoevsky's Brothers Karamazov.'', 101.} Ivan's argument with God is that by ``giving humans freedom, God has become complicit in the senseless, unending, unredeemable suffering of the innocent.''\footnote{ibid, 102.} Believing this to be the case, Ivan openly refuses to accept God. This will be discussed later, but it is important to note that Ivan is the intellectual amongst the Karamazov brothers and that his arguments are purely rational. Furthermore, it is even questionable as to whether or not Ivan has convinced himself with his logic: 
``I want to live, and I do live, even if it be against logic. Though I do not believe in the order of things, still the sticky little leaves that come out in the spring are dear to me, the blue sky is dear to me, some people are dear to me, whom one loves sometimes, would you believe it, without knowing why.''\footnote{Dostoyevsky, Fyodor. Translated by Richard Pevear, and Larissa Volokhonsky. \emph{The Brothers Karamazov: A Novel in Four Parts with Epilogue}. New York: Farrar, Straus and Giroux, 2002, 230.}

What we see here is that even Ivan, with his emphasis on `Euclidean Logic', has not been able to live his life purely through logic. Human life encompasses more than logic and this is precisely the vein that Dostoevsky uses to build his case for faith. It is important to note that Dostoevsky does not refute Ivan directly, or even with `Euclidean Logic'. Rather, his ``reply to Ivan's materialism is based upon a poetic revelation of the existential results of his ideas as manifest in his life and the lives of those around him...''\footnote{Osborn, Ronald. ``Beauty Will Save the World: Metaphysical Rebellion and the Problem of Theodicy in Dostoevsky's Brothers Karamazov.'', 102.} These existential results show us that Ivan cannot continue leading his life the way that he does, that it is simply not a sustainable `philosophy of life'. We even begin to see this in Ivan's admission that he cannot explain his desire to live nor his desire to love. Ultimately, the argument for belief in God, despite the overwhelming suffering of the innocent, lies in embracing gratitude as a created being. This is the `answer' to theodicy that is put forth not only in Job, but also in Malik's Tree of Life and Dostoevsky's Brothers Karamazov. God's response to Job leads us back to creation and in Genesis we learn that God has created us, in His own image, out of love.

We already see this answer forming in Ivan's appreciation of life and nature. We see this answer forming in Brad Pitt's reflection in the Tree of Life, ``I'm nothing. Look at the glory around us. trees and birds. I lived in shame. I dishonored it all and didn't notice the glory. I'm a foolish man.'' Structurally, then, the arc that we can follow will be this: starting with the text of Job itself, we understand that God's response leads Job to Genesis; an analysis of Genesis, will lead us to a number of important themes, the primary one being that humans are created by God out of love; at this point, we will look at two pieces of human imagination that grapple with theodicy: The Tree of Life and Brothers Karamazov. Both of these works speak to an answer to the theodicy that is not only theoretical, but that is tested against reality. After all, the question of theodicy is not merely a theoretical mind-game. It is a question of how to live in the world. In fact, we see in Ivan a tenderness --- in his disgust with suffering --- that is absent from the faith. This is ultimately what we see in Ivan's ''if there is no God, everything is permitted.'' Perhaps it is the acknowledgement of this fact that does not allow Ivan to completely embrace his philosophical rebellion. Perhaps, too, Ivan's hesitation to fully live out his logical conclusion reveals another aspect of the question of theodicy, that it is a question not so much about whether or not God exists, but whether or not we can accept the terms of the world God has created. It is these very terms, manifest in the physical decay of Zosima's body, that cause even Alyosha to doubt God. Thus, as Rowan Williams alludes to in his introduction to \emph{Dostoevsky: Language, Faith and Fiction}, the ``crisis is not so much, then, about whether God exists, but about what the nature is of God's relation with the world, and most of all with the human world.''\footnote{Williams, Rowan. \emph{Dostoevsky:Language, Faith, and Fiction}, viii.} This crisis is the embodiment of a `serious spiritual perception' which states that ``God exists but I'm not sure whether I believe in him.''\footnote{ibid,8.}
This, ultimately, then is a question of belief in objective meaning. And, as Ratzinger tells us in his \emph{Introduction to Christianity}: the ``Christian faith lives on the discovery that not only is there such a thing as objective meaning but that this meaning knows me and loves me, that I can entrust myself to it like the child who knows that everything he may be wondering about is safe in the ``you'' of his mother.''\footnote{XVI, Pope Benedict. \emph{Introduction to Christianity}, 80.} This, ultimately, seems to be the answer that Job arrives at as well.
In order to make the argument for her claim that there are things to know which can only be known through narrative, Stump categorizes two modes of thought in a typological manner: the Dominican system and the Franciscan. ``Categorizing on the basis of sets of abstract properties and abstract designations can itself be thought of as Dominican; categorizing on the basis of typology, which requires acquaintance with stories and persons, can be taken as Franciscan.''\footnote{Stump, Eleonore. ``Narrative as a Means of Knowledge: Francis and Dominic.'' In \emph{\emph{Wandering in Darkness: Narrative and the Problem of Suffering}}, 4.} After briefly introducing these typologies, Stump explains that in ``case the thing being characterized is not amenable to crisp definition and precision, then, paradoxically, the vague but intuitive Franciscan approach will be more accurate than the Dominican approach, whose search for an unavailable accuracy will result in carefully patterned mischaracterization.''\footnote{ibid, 4.} 
She continues to explain that while Dominic's ministry is grounded in argumentation, Francis' ministry ``is grounded in his personal response to a personal call from a suffering incarnate deity. The rebuilding of Christ's church that Francis was called to do is based on Francis' personal relationship to the person of Christ...''\footnote{ibid, 9.} The very nature of the typological Franciscan understanding is a personal relationship --- and the ``knowledge of the ultimate foundation of reality, knowledge of morality, and knowledge of the good life are all best understood as knowledge of persons.''\footnote{ibid, 11.} This is strikingly similar to the change that occurs in Job as well as the proposal of active love, which is a love that entirely focused on personal relationships.

The argument continues with a contrast between two types of knowledge. Knowledge that, in analytic philosophy ``knowledge is a matter of having an attitude toward a proposition, of knowing that...''\footnote{ibid, 14.} In this view, even statements such as `I believe in God' are transformed into statements of `I believe p because you tell me that p', giving us a statement such as, `I believe that God exists and is good and trustworthy'.\footnote{ibid, 13.} Yet, the argument continues, this is clearly not the only way in which we can know. In contrast, ``some analytic philosophers, most famously Bertrand Russell, have contrasted knowing that with knowing by acquaintance.''\footnote{ibid, 15.} Stump is quick to point out that the theory of knowing by acquaintance is less developed than the theory of knowing that. This, perhaps, is to be expected as knowing by acquaintance does not itself lead to rigorous analytical analysis. Despite this, there Stump provides a thought experiment that makes the point clear:
The thought experiment involves a neuroscientist Mary who knows everything there is to know about the brain, including everything there is to know about the way in which the brain processes color, but who has had no perceptual experience of color of any sort because from birth she has been isolated (by some suitable villain) in a black-and-white environment. Now suppose that Mary is finally rescued from her imprisonment and perceives color for the first time... it seems clear that she comes to know something new when she first sees colors. But the problem is not just that no neuroscience books could give her the knowledge she acquires when she first sees red. No books could giver her this knowledge. There are certain first-person knowledge states that cannot be adequately acquired or expressed by means of propositional attitudes. That is why, when Mary first knows redness in virtue of perceiving it, it is hard to express what she knows adequately in terms of knowing that.\footnote{ibid, 15-16.}

This experiment is a key illustration of the sort of knowledge that we are trying to identify throughout this thesis. Furthermore, it is a key illustration of the necessity of incorporating other forms of knowledge --- such as narrative and literature --- into our discussion. Ultimately, the `answer' to the problem of suffering is grounded in a personal relationship with God --- to use Stump's terminology, it is a Franciscan answer. Thus, it will be vague by definition. But we can note certain key aspects of the answer: it is based in a relationship with God, a relationship with other created beings (Zosima's active love) and it defies reduction to a rational principle in the same way that Ivan cannot explain his desire to live.

\section{Statement of Thesis and Layout of Work}

This thesis proposes that the problem of suffering, an inherently interpersonal problem, must be solved in an all-inclusive way of life that is much more than a `philosophy of life'. The exact way of life that is proposed is Zosima's active love that is displayed in Dostoevsky's Brothers Karamazov. Active love is a way of being that has three particular characteristics: it is rooted in gratitude and a proper understanding of man's relationship with God (i); it is characterized by proper perspective (ii) in two ways: it recognizes that suffering is not always going to be understandable (iii) and it focuses on particulars instead of broad and complicated generalities (it deals with a particular person instead of `humanity') (iv); it accepts reality as it is, whether that be man's fallenness or in a sort of `higher justice or purpose' of suffering manifesting itself in this life (v). In these ways, active love is firmly compatible with Christianity and, in fact, is a clear `way of life' that is rooted in Christianity. Gratitude necessarily follows from the Christian belief that we are created beings entirely dependent on God --- the very God who made man in his image and breathed life into him. Just as Christ told us that we serve him by serving the poor, so too does active love focus on seeing Christ in all people. Just as Christ could not avoid death on a cross, so too man cannot avoid suffering. Christ became fully human, accepting man on man's own terms of fallenness, and suffered despite his innocence. Yet, precisely the point that was darkest in the Christian narrative --- Christ's complete gift of himself --- was also its great triumph. This is exactly the message that Zosima teaches: when one has given themselves entirely, when they are at their breaking point, precisely at that point will God's grace restore them and invigorate them.

This paper will work through these aspects of active love in the order they have been presented. Gratitude and man's relationship with God (i) and man's proper relationship with God is the focus of the first section. This section makes extensive use of the text of Job and, because God's speeches to Job are direct references to creation, the text of Genesis. The primary point of emphasis in this section is that God directs Job's attention to creation --- signalling that creation (and its lessons) marks the turning point for Job, leading him into a more complete relationship with God. Benedict XVI's, then Cardinal Ratzinger, issued a number of relevant homilies which will be a primary source of information on Genesis. Proper perspective (ii) will be discussed in the following section, with the intention of elucidating two points: that Ivan Karamazov's dilemma will never be completely resolved (iii), which has the Brothers Karamazov as its primary material, and that the key to active love is seeing Christ in all people (iv) --- and focusing on each individual person --- which will also have Dostoevsky's novel as its main source. The final section of the paper will focus on active love's emphasis on accepting life as it is (v). This final section will primarily focus on Ivan Karamazov's rebellion, his grand inquisitor and, accordingly, Zosima's counter-responses within his active love.
