\documentclass[12pt]{letter}
\usepackage{fontspec}
\usepackage[top=1in, bottom=1in, left=1in, right=1in]{geometry}
\usepackage{setspace}
%\doublespacing
\usepackage{parskip}
% note to future self: use this command to find the proper name of fonts 
% fc-list : family file | grep -i fontin or look at font manager!
% \fontspec[Ligatures=TeX,BoldFont = BaskervBookExpBQ-MediumOsF, ItalicFont = BaskervBookExpBQ-ItalicOsF, BoldItalicFont = BaskervBookExpBQ-MediumItalicOsF]{BaskervBookExpBQ-RegularOsF}
\pagestyle{plain}
\setlength{\parindent}{1cm}
\begin{document}
BUCK V. BELL

274 U.S. 200

Buck v. Bell (No. 292)

Argued: April 22, 1927

Decided: May 2, 1927

143 Va. 310, affirmed.

Syllabus
Opinion, Holmes
Syllabus

1. The Virginia statute providing for the sexual sterilization of inmates of institutions supported by the State who shall be found to be afflicted with an hereditary form of insanity or imbecility, is within the power of the State under the Fourteenth Amendment. P. 207.

2. Failure to extend the provision to persons outside the institutions named does not render it obnoxious to the Equal Protection Clause. P. 208.

ERROR to a judgment of the Supreme Court of Appeals of the State of Virginia which affirmed a judgment ordering [p201] the Superintendent of the State Colony of Epileptics and Feeble Minded to perform the operation of salpingectomy on Carrie Buck, the plaintiff in error. [p205]

TOP
Opinion

HOLMES, J., Opinion of the Court

Mr. JUSTICE HOLMES delivered the opinion of the Court.

This is a writ of error to review a judgment of the Supreme Court of Appeals of the State of Virginia affirming a judgment of the Circuit Court of Amherst County by which the defendant in error, the superintendent of the State Colony for Epileptics and Feeble Minded, was ordered to perform the operation of salpingectomy upon Carrie Buck, the plaintiff in error, for the purpose of making her sterile. 143 Va. 310. The case comes here upon the contention that the statute authorizing the judgment is void under the Fourteenth Amendment as denying to the plaintiff in error due process of law and the equal protection of the laws.

Carrie Buck is a feeble minded white woman who was committed to the State Colony above mentioned in due form. She is the daughter of a feeble minded mother in the same institution, and the mother of an illegitimate feeble minded child. She was eighteen years old at the time of the trial of her case in the Circuit Court, in the latter part of 1924. An Act of Virginia, approved March 20, 1924, recites that the health of the patient and the welfare of society may be promoted in certain cases by the sterilization of mental defectives, under careful safeguard, etc.; that the sterilization may be effected in males by vasectomy and in females by salpingectomy, without serious pain or substantial danger to life; that the Commonwealth is supporting in various institutions many defective persons who, if now discharged, would become [p206] a menace, but, if incapable of procreating, might be discharged with safety and become self-supporting with benefit to themselves and to society, and that experience has shown that heredity plays an important part in the transmission of insanity, imbecility, c. The statute then enacts that, whenever the superintendent of certain institutions, including the above-named State Colony, shall be of opinion that it is for the best interests of the patients and of society that an inmate under his care should be sexually sterilized, he may have the operation performed upon any patient afflicted with hereditary forms of insanity, imbecility, c., on complying with the very careful provisions by which the act protects the patients from possible abuse.

The superintendent first presents a petition to the special board of directors of his hospital or colony, stating the facts and the grounds for his opinion, verified by affidavit. Notice of the petition and of the time and place of the hearing in the institution is to be served upon the inmate, and also upon his guardian, and if there is no guardian, the superintendent is to apply to the Circuit Court of the County to appoint one. If the inmate is a minor, notice also is to be given to his parents, if any, with a copy of the petition. The board is to see to it that the inmate may attend the hearings if desired by him or his guardian. The evidence is all to be reduced to writing, and, after the board has made its order for or against the operation, the superintendent, or the inmate, or his guardian, may appeal to the Circuit Court of the County. The Circuit Court may consider the record of the board and the evidence before it and such other admissible evidence as may be offered, and may affirm, revise, or reverse the order of the board and enter such order as it deems just. Finally any party may apply to the Supreme Court of Appeals, which, if it grants the appeal, is to hear the case upon the record of the trial [p207] in the Circuit Court, and may enter such order as it thinks the Circuit Court should have entered. There can be no doubt that, so far as procedure is concerned, the rights of the patient are most carefully considered, and, as every step in this case was taken in scrupulous compliance with the statute and after months of observation, there is no doubt that, in that respect, the plaintiff in error has had due process of law.

The attack is not upon the procedure, but upon the substantive law. It seems to be contended that in no circumstances could such an order be justified. It certainly is contended that the order cannot be justified upon the existing grounds. The judgment finds the facts that have been recited, and that Carrie Buck is the probable potential parent of socially inadequate offspring\footnote{socially inadequate? woah... the question is when and where do we start drawing a line? is there any room for objective moral standards?}, likewise afflicted, that she may be sexually sterilized without detriment to her general health, and that her welfare and that of society will be promoted by her sterilization, and thereupon makes the order. In view of the general declarations of the legislature and the specific findings of the Court, obviously we cannot say as matter of law that the grounds do not exist, and, if they exist, they justify the result. We have seen more than once that the public welfare may call upon the best citizens for their lives. It would be strange if it could not call upon those who already sap the strength of the State for these lesser sacrifices, often not felt to be such by those concerned, in order to prevent our being swamped with incompetence. It is better for all the world if, instead of waiting to execute degenerate offspring for crime or to let them starve for their imbecility, society can prevent those who are manifestly unfit from continuing their kind. The principle that sustains compulsory vaccination is broad enough to cover cutting the Fallopian tubes. Jacobson v. Massachusetts, 197 U.S. 11. Three generations of imbeciles are enough. [p208]

But, it is said, however it might be if this reasoning were applied generally, it fails when it is confined to the small number who are in the institutions named and is not applied to the multitudes outside. It is the usual last resort of constitutional arguments to point out shortcomings of this sort. But the answer is that the law does all that is needed when it does all that it can, indicates a policy, applies it to all within the lines, and seeks to bring within the lines all similarly situated so far and so fast as its means allow. Of course, so far as the operations enable those who otherwise must be kept confined to be returned to the world, and thus open the asylum to others, the equality aimed at will be more nearly reached.

Judgment affirmed.

MR. JUSTICE BUTLER dissents.

-------
Carrie Buck’s Daughter
Stephen Jay Gould
(originally published in Natural History magazine, July 1984)
The Lord really put it on the line in his preface to that prototype of all prescription, the
Ten Commandments:

for I, the Lord thy God, am a jealous God, visiting the iniquity of the fathers
upon the children unto the third and fourth generation of them that hate me
(Exod. 20:5).

The terror of this statement lies in its patent unfairness—its promise to punish guiltless
offspring for the misdeeds of their distant forebears.

A different form of guilt by genealogical association attempts to remove this stigma of
injustice by denying a cherished premise of Western thought—human free will. If
offspring are tainted not simply by the deeds of their parents but by a material form of
evil transferred directly by biological inheritance, then “the iniquity of the fathers”
becomes a signal or warning for probable misbehavior of their sons. Thus Plato, while
denying that children should suffer directly for the crimes of their parents, nonetheless
defended the banishment of a man whose father, grandfather, and great-grandfather had
all been condemned to death.\footnote{where does Plato do this?}

It is, perhaps, merely coincidental that both Jehovah and Plato chose three generations as
their criterion for establishing different forms of guilt by association. Yet we have a
strong folk, or vernacular, tradition for viewing triple occurrences as minimal evidence of
regularity We are told that bad things come in threes. Two may be an accidental
association; three is a pattern. Perhaps, then, we should not wonder that our own
century’s most famous pronouncement of blood guilt employed the same criterion—
Oliver Wendell Holmes’s defense of compulsory sterilization in Virginia (Supreme Court
decision of 1927 in Buck v. Bell): “three generations of imbeciles are enough.”
Restrictions upon immigration, with national quotas set to discriminate against those
deemed mentally unfit by early versions of IQ testing, marked the greatest triumph of the
American eugenics movement—the flawed hereditarian doctrine, so popular earlier in
our century and by no means extinct today (see my column on Singapore’s “great
marriage debate,” May 1984), that attempted to “improve” our human stock by
preventing the propagation of those deemed biologically unfit and encouraging
procreation among the supposedly worthy. But the movement to enact and enforce laws
for compulsory “eugenic” sterilization had an impact and success scarcely less
pronounced. If we could debar the shiftless and the stupid from our shores, we might also
prevent the propagation of those similarly afflicted but already here.

The movement for compulsory sterilization began in earnest during the 1890s, abetted by
two major factors—the rise of eugenics as an influential political movement and the
perfection of safe and simple operations (vasectomy for men and salpingectomy, the
cutting and tying of Fallopian tubes, for women) to replace castration and other obvious
mutilation. Indiana passed the first sterilization act based on eugenic principles in 1907\footnote{INDIANA!} (a
few states had previously mandated castration as a punitive measure for certain sexual
crimes, although such laws were rarely enforced and usually overturned by judicial
review). Like so many others to follow, it provided for sterilization of afflicted people
residing in the state’s “care,” either as inmates of mental hospitals and homes for the
feebleminded or as inhabitants of prisons. Sterilization could be imposed upon those
judged insane, idiotic, imbecilic, or moronic, and upon convicted rapists or criminals
when recommended by a board of experts.

By the 1930s, more than thirty states had passed similar laws, often with an expanded list
of so-called hereditary defects, including alcoholism and drug addiction in some states,
and even blindness and deafness in others.\footnote{wow. that is really extensive....} It must be said that these laws were continually challenged and rarely enforced in most states; only California and Virginia
applied them zealously. By January 1935, some 20,000 forced “eugenic” sterilizations
had been performed in the United States, nearly half in California.

No organization crusaded more vociferously and successfully for these laws than the
Eugenics Record Office, the semiofficial arm and repository of data for the eugenics
movement in America. Harry Laughlin, superintendent of the Eugenics Record Office,
dedicated most of his career to a tireless campaign of writing and lobbying for eugenic
sterilization. He hoped, thereby, to eliminate in two generations the genes of what he
called the “submerged tenth”—“the most worthless one-tenth of our present population.”
He proposed a “model sterilization law” in 1922, designed to prevent the procreation of persons socially inadequate from defective inheritance, by authorizing and providing for eugenical sterilization of certain
potential parents carrying degenerate hereditary qualities.

This model bill became the prototype for most laws passed in America, although few
states cast their net as widely as Laughlin advised. (Laughlin’s categories encompassed
“blind, including those with seriously impaired vision; deaf, including those with
seriously impaired hearing; and dependent, including orphans, ne’er-do-wells, the
homeless, tramps, and paupers.”) Laughlin’s suggestions were better heeded in Nazi
Germany, where his model act served as a basis for the infamous and stringently enforced
Erbgesundhetsrecht, leading by the eve of World War 11 to the sterilization of some
375,000 people, most for “congenital feeblemindedness,” but including nearly 4,000 for
blindness and deafness.\footnote{wow.}

The campaign for forced eugenic sterilization in America reached its climax and height
of respectability in 1927, when the Supreme Court, by an 8-1 vote, upheld the Virginia
sterilization bill in the case of Buck v. Bell. Oliver Wendell Holmes, then in his mid-
eighties and the most celebrated jurist in America, wrote the majority opinion with his
customary verve and power of style. It included the notorious paragraph, with its chilling
tag line, cited ever since as the quintessential statement of eugenic principles.
Remembering with pride his own distant experiences as an infantryman in the Civil War,
Holmes wrote:

We have seen more than once that the public welfare may call upon the best
citizens for their lives. It would be strange if it could not call upon those who
already sap the strength of the state for these lesser sacrifices. . . . It is better
for all the world, if instead of waiting to execute degenerate offspring for
crime, or to let them starve for their imbecility, society can prevent those who
are manifestly unfit from continuing their kind. The principle that sustains
compulsory vaccination is broad enough to cover cutting the Fallopian tubes.
Three generations of imbeciles are enough.

Who, then, were the famous “three generations of imbeciles,” and why should they still
compel our interest?

When the state of Virginia passed its compulsory sterilization law in 1924, Carrie Buck,
an eighteen-year-old white woman, was an involuntary resident at the State Colony for
Epileptics and Feeble-Minded. As the first person selected for sterilization under the new
act, Carrie Buck became the focus for a constitutional challenge launched, in part, by
conservative Virginia Christians who held, according to eugenical “modernists,”
antiquated views about individual preferences and “benevolent” state power. (Simplistic
political labels do not apply in this case, and rarely do in general. We usually regard
eugenics as a conservative movement and its most vocal critics as members of the left.
This alignment has generally held in our own decade. But eugenics, touted in its day as
the latest in scientific modernism, attracted many liberals and numbered among its most
vociferous critics groups often labeled as reactionary and antiscientific. If any political
lesson emerges from these shifting allegiances, we might consider the true inalienability of certain human rights.)

But why was Carrie Buck in the State Colony, and why was she selected? Oliver Wendell
Holmes upheld her choice as judicious in the opening lines of his 1927 opinion:
Carrie Buck is a feeble-minded white woman who was committed to the State
Colony. . . . She is the daughter of a feeble-minded mother in the same
institution, and the mother of an illegitimate feeble-minded child.
In short, inheritance stood as the crucial issue (indeed as the driving force behind all
eugenics). For if measured mental deficiency arose from malnourishment, either of body
or mind, and not from tainted genes, then how could sterilization be justified? If decent
food, upbringing, medical care, and education might make a worthy citizen of Carrie
Buck’s daughter, how could the State of Virginia justify the severing of Carries Fallopian
tubes against her will? (Some forms of mental deficiency are passed by inheritance in
family line, but most are not—a scarcely surprising conclusion when we consider the
thousand shocks that beset fragile humans during their lives, from difficulties in
embryonic growth to traumas of birth, malnourishment, rejection, and poverty. In any
case, no fair-minded person today would credit Laughlin’s social criteria for the
identification of hereditary deficiency—ne’er-do-wells, the homeless, tramps, and
paupers—although we shall soon see that Carrie Buck was committed on these grounds.\footnote{It would be great to see the empirical data on the causes of `feeblemindedness'})

When Carrie Buck’s case emerged as the crucial test of Virginia’s law, the chief honchos
of eugenics knew that the time had come to put up or shut up on the crucial issue of
inheritance. Thus, the Eugenics Record Office sent Arthur H. Estabrook, their crack
fieldworker, to Virginia for a “scientific” study of the case. Harry Laughlin himself
provided a deposition, and his brief for inheritance was presented at the local trial that
affirmed Virginia’s law and later worked its way to the Supreme Court as Buck v. Bell.
Laughlin made two major points to the court. First, that Carrie Buck and her mother,
Emma Buck, were feeble-minded by the Stanford-Binet test of IQ, then in its own
infancy. Carrie scored a mental age of nine years, Emma of seven years and eleven
months. (These figures ranked them technically as “imbeciles” by definitions of the day,
hence Holmes’s later choice of words. Imbeciles displayed a mental age of six to nine
years; idiots performed worse, morons better, to round out the old nomenclature of
mental deficiency.) Second, that most feeblemindedness is inherited, and Carrie Buck
surely belonged with this majority. Laughlin reported:

Generally feeble-mindedness is caused by the inheritance of degenerate
qualities; but sometimes it might be caused by environmental factors which
are not hereditary. In the case given, the evidence points strongly toward the
feeble-mindedness and moral delinquency of Carrie Buck being due,
primarily, to inheritance and not to environment.

Carrie Buck’s daughter was then, and has always been, the pivotal figure of this painful
case. As I stated before, we tend (often at our peril) to regard two as potential accident
and three as an established pattern. The supposed imbecility of Emma and Carrie might
have been coincidental, but the diagnosis of similar deficiency for Vivian Buck (made by
a social worker, as we shall see, when Vivian was but six months old) tipped the balance
in Laughlin’s favor and led Holmes to declare the Buck lineage inherently corrupt by
deficient heredity Vivian sealed the pattern—three generations of imbeciles are enough.
Besides, had Carrie not given illegitimate birth to Vivian, the issue (in both senses) would
never have emerged.

Oliver Wendell Holmes viewed his work with pride. The man so renowned for his
principle of judicial restraint, who had proclaimed that freedom must not be curtailed
without “clear and present danger”—without the equivalent of falsely yelling “fire” in a
crowded theater—wrote of his judgment in Buck v. Bell: “I felt that I was getting near the
first principle of real reform.”\footnote{this is very scary.... if real reform consists in this, where are we going to stop?}

And so the case of Buck v. Bell remained for fifty years, a footnote to a moment of
American history perhaps best forgotten. And then, in 1980, it reemerged to prick our
collective conscience, when Dr. K. Ray Nelson, then director of the Lynchburg Hospital
where Carrie Buck was sterilized, researched the records of his institution and discovered
that more than 4,000 sterilizations had been performed, the last as late as 1972. He also
found Carrie Buck, alive and well near Charlottesville, and her sister Doris, covertly
sterilized under the same law (she was told that her operation was for appendicitis), and
now, with fierce dignity, dejected and bitter because she had wanted a child more than
anything else in her life and had finally, in her old age, learned why she had never
conceived.

As scholars and reporters visited Carrie Buck and her sister, what a few experts had
known all along became abundantly clear to everyone. Carrie Buck was a woman of
obviously normal intelligence.\footnote{wow.} For example, Paul A. Lombardo of the School of Law at
the University of Virginia, and a leading scholar of the Buck v. Bell case, wrote in a letter
to me:

As for Carrie, when I met her she was reading newspapers daily and joining a
more literate friend to assist at regular bouts with the crossword puzzles. She
was not a sophisticated woman, and lacked social graces, but mental health
professionals who examined her in later life confirmed my impressions that
she was neither mentally ill nor retarded.

On what evidence, then, was Carrie Buck consigned to the State Colony for Epileptics
and Feeble-Minded on January 23, 1924? I have seen the text of her commitment
hearing; it is, to say the least, cursory and contradictory. Beyond the simple and
undocumented say-so of her foster parents, and her own brief appearance before a
commission of two doctors and a justice of the peace, no evidence was presented. Even
the crude and early Stanford-Binet test, so fatally flawed as a measure of innate worth
(see my book The Mismeasure of Man, although the evidence of Carrie’s own case
suffices) but at least clothed with the aura of quantitative respectability, had not yet been
applied.

When we understand why Carrie Buck was committed in January 1924, we can finally
comprehend the hidden meaning of her case and its message for us today. The silent key,
again and as always, is her daughter Vivian, born on March 28, 1924, and then but an
evident bump on her belly. Carrie Buck was one of several illegitimate children borne by
her mother, Emma. She grew up with foster parents, J.T. and Alice Dobbs, and continued
to live with them, helping out with chores around the house. She was apparently raped by
a relative of her foster parents, then blamed for her resultant pregnancy. Almost surely,
she was (as they used to say) committed to hide her shame (and her rapist’s identity), not
because enlightened science had just discovered her true mental status. In short, she was
sent away to have her baby. Her case never was about mental deficiency; it was always a
matter of sexual morality and social deviance.\footnote{it always is...} The annals of her trial and hearing reek
with the contempt of the well-off and well-bred for poor people of “loose morals.” Who
really cared whether Vivian was a baby of normal intelligence; she was the illegitimate
child of an illegitimate woman. Two generations of bastards are enough. Harry Laughlin
began his “family history” of the Bucks by writing: “These people belong to the shiftless,
ignorant and worthless class of anti-social whites of the South.”

We know little of Emma Buck and her life, but we have no more reason to suspect her
than her daughter Carrie of true mental deficiency. Their deviance was social and sexual;
the charge of imbecility was a cover-up, Mr. Justice Holmes notwithstanding.

We come then to the crux of the case, Carrie’s daughter, Vivian. What evidence was ever
adduced for her mental deficiency? This and only this: At the original trial in late 1924,
when Vivian Buck was seven months old, a Miss Wilhelm, social worker for the Red
Cross, appeared before the court. She began by stating honestly the true reason for Carrie
Buck’s commitment:

Mr. Dobbs, who had charge of the girl, had taken her when a small child, had
reported to Miss Duke [the temporary secretary of Public Welfare for
Albemarle County] that the girl was pregnant and that he wanted to have her
committed somewhere—to have her sent to some institution.

Miss Wilhelm then rendered her judgment of Vivian Buck by comparing her with the
normal granddaughter of Mrs. Dobbs, born just three days earlier:
It is difficult to judge probabilities of a child as young as that, but it seems to
me not quite a normal baby. In its appearance—I should say that perhaps my
knowledge of the mother may prejudice me in that regard, but I saw the child
at the same time as Mrs. Dobbs’ daughter’s baby, which is only three days
older than this one, and there is a very decided difference in the development
of the babies.

That was about two weeks ago. There is a look about it that is not quite
normal, but just what it is, I can’t tell.

This short testimony, and nothing else, formed all the evidence for the crucial third
generation of imbeciles. Cross-examination revealed that neither Vivian nor the Dobbs
grandchild could walk or talk, and that “Mrs. Dobbs’ daughter’s baby is a very
responsive baby. When you play with it or try to attract its attention—it is a baby that you
can play with. The other baby is not. It seems very apathetic and not responsive.” Miss
Wilhelm then urged Carrie Buck’s sterilization: “I think,” she said, “it would at least
prevent the propagation of her kind.” Several years later, Miss Wilhelm denied that she
had ever examined Vivian or deemed the child feebleminded.

Unfortunately, Vivian died at age eight of “enteric colitis” (as recorded on her death
certificate), an ambiguous diagnosis that could mean many things but may well indicate
that she fell victim to one of the preventable childhood diseases of poverty (a grim
reminder of the real subject in Buck v. Bell).\footnote{poverty will always be with us... there will always be poor people, so getting rid of imbeciles will do nothing!} She is therefore mute as a witness in our
reassessment of her famous case.

When Buck v. Bell resurfaced in 1980, it immediately struck me that Vivian’s case was
crucial and that evidence for the mental status of a child who died at age eight might best
be found in report cards. I have therefore been trying to track down Vivian Buck’s school
records for the past four years and have finally succeeded. (They were supplied to me by
Dr. Paul A. Lombardo, who also sent other documents, including Miss Wilhelm’s
testimony, and spent several hours answering my questions by mail and Lord knows how
much time playing successful detective in re Vivian’s school records. I have never met
Dr. Lombardo; he did all this work for kindness, collegiality, and love of the game of
knowledge, not for expected reward or even requested acknowledgment. In a
profession—academics—so often marked by pettiness and silly squabbling over
meaningless priorities, this generosity must be recorded and celebrated as a sign of how
things can and should be.)

Vivian Buck was adopted by the Dobbs family, who had raised (but later sent away) her
mother, Carrie. As Vivian Alice Elaine Dobbs, she attended the Venable Public
Elementary School of Charlottesville for four terms, from September 1930 until May
1932, a month before her death. She was a perfectly normal, quite average student,
neither particularly outstanding nor much troubled. In those days before grade inflation,
when C mean “good, 81-87” (as defined on her report card) rather than barely scraping
by, Vivian Dobbs received A’s and B’s for deportment and C’s for all academic subjects
but mathematics (which was always difficult for her, and where she scored D) during her
first term in Grade 1A, from September 1930 to January 1931. She improved during her
second term in lB, meriting an A in deportment, C in mathematics, and B in all other
academic subjects; she was on the honor roll in April 1931. Promoted to 2A, she had
trouble during the fall term of 1931, failing mathematics and spelling but receiving A in
deportment, B in reading, and C in writing and English. She was “retained in 2A” for the
next term—or “left back” as we used to say, and scarcely a sign of imbecility as I
remember all my buddies who suffered a similar fate. In any case, she again did well in
her final term, with B in deportment, reading, and spelling, and C in writing, English, and
mathematics during her last month in school. This offspring of “lewd and immoral”
women excelled in deportment and performed adequately, although not brilliantly, in her
academic subjects.

In short, we can only agree with the conclusion that’Dr. Lombardo has reached in his
research on Buck v. Bell—there were no imbeciles, not a one, among the three
generations of Bucks. I don’t know that such correction of cruel but forgotten errors of
history counts for much, but it is at least satisfying to learn that forced eugenic
sterilization, a procedure of such dubious morality, earned its official justification (and
won its most quoted line of rhetoric) on a patent falsehood.

Carrie Buck died last year. By a quirk of fate, and not by memory or design, she was
buried just a few steps from her only daughter’s grave. In the umpteenth and ultimate
verse of a favorite old ballad, a rose and a brier—the sweet and the bitter—emerge from
the tombs of Barbara Allen and her lover, twining about each other in the union of death.
May Carrie and Vivian, victims in different ways and in the flower of youth, rest together
in peace.


--------

Darwin and human progress: The seeds of sociobiology

What is most remarkable about The Descent of Man is its thoroughgoing biological approach to the problem of human behavior. Where Darwin might have limited himself to a discussion of the anatomical and physiological affinities between humans and animals, he ranged much further and subjected every aspect of human life to scrutiny from the standpoint of natural history. The most important obstacle to the acceptance of Darwin’s theory was the


SHARON KINGSLAND 181
belief that man's moral sense could not have originated by natural selection.\footnote{Yeah, this seems to be a pretty important point... perhaps the point is similar to that of the creation of the soul at conception?} Darwin made no exceptions in his efforts to find a connection between mans and their animal origins, down to the last psychological detail: aesthetic sense, religious feeling, and conscience were all to be found, in rudimentary form, in the lower world. In this respect Darwin's approach pushed the biological view of man to its limit; here we have the origin of what is now called sociobiology.\footnote{is it possible to answer questions of truth from an empirical science?}

One of the most difficult problems was to account for altruism in humans\footnote{altruism! might be accounted for without moral means, though}, which Darwin considered to be the basis of the moral sense. He connected altruism to the social instincts, including parental and filial affections, which were found in the animal world, but which in humans had been perfected under the influence of natural selection. To explain how selection could create altruism, which after all did not benefit the altruistic individual, DarWin introduced the idea of community selection (Richards, 1987). Natural selection, he suggested, could act at the level of the community, preserving characteristics that would give the community as a Whole an advantage over another community, even though they did not seem to benefit the individual directly. Darwin had used this idea in the Origin to explain the evolution of sterile worker castes in insect societies: their sterility did not benefit the workers directly, but they were specialized to help their queen produce more workers efficiently, and the community flourished under their tireless efforts.\footnote{importance of insects!} In the Descent he turned the idea to the human community arguing that altruistic traits would be crucial advantages in competition between different tribes: "There can be no doubt that a tribe including many members Who, from possessing in a high degree the spirit of patriotism, fidelity, obedience, cour— age, and sympathy, were always ready to give aid to each other and to sacrifice themselves for the common good, would be victorious over most other tribes; and this would be natural selection" (quoted in Richards, 1987, p. 215).

DarWin’s idea that moral advance would give one tribe a competitive advantage over another was similar to Wallace's original argument for natural selection on the cultural level, but Wallace had dropped this argument in his later turn to spiritual causes. The main problem with the theory of community selection is that the human community is not as closely related as the insect society, so that it is not clear how altruistic behavior would become established within the population. Darwin was aware of this problem, but thought that in early societies members of tribes would be fairly closely related. He believed strongly that the basic social instinct, the feeling of sympathy that served as a social bond, was inheritable like any other instinct and had developed through natural selection. Human nature was, at its roots, altruistic. His explanation differed from Spencer's, but both concluded that morality was innate, a product of evolution.\footnote{morality was a product of evolution! morality was innate. this does not mutually exclude morality/christianity, though. It might be that it developed through evolution....}

Darwin’s interpretation was a biological version of the moral philosophy of James Mackintosh, a leading Whig politician in the two decades before his death in 1832. He was also connected to the Darwin family, being the


182 EVOLUTION AND HUMAN PROGRESS
brother-in-law of Darwin’s uncle Josiah Wedgwood. Mackintosh argued that humans had an instinctive sense of right and wrong, a kind of moral impulse that guided behavior, quite apart from any rational motive such as the anticipation of pleasure, with which such behavior might be associated.\footnote{in what way is this different than the idea of the law being written on one's heart?} Darwin gave this innate moral sense a biological history. Like Mackintosh, he intended the argument that altruism was innate to be a response to the British Utilitarian theorists of the time, who argued that human actions were guided by a search for pleasure and avoidance of pain, in other words by selfish motives. To the contrary, Darwin thought that humans often seemed to act unconsciously, as if from instinct or long habit.\footnote{it is interesting to note that Darwin was not a utilitarian, as we might expect this to be the position of a struggle for existence...} Far from being guided by a search for happiness or pleasure, people seemed to be impelled by a deeply planted social instinct. Morality was founded not on selfishness, but on an altruistic impulse, which had a biological basis and could be seen in undeveloped form in the lower animals. The goal of altruistic behavior was not the general happiness, but the general good of the community; defined as the maximum production of healthy, vigorous offspring having all their faculties intact (Darwin, 1896 ; Richards, 1987, pp. 115—1 16).

Darwin did make one small concession in response to the criticism that natural selection could not account for the adaptive value of every human characteristic.\footnote{note this! sexual selection is sort of the grey area for darwin, allowing for characateristics that he could not account for with natural selection alone...} This exception was developed in his theory of "sexual selection," which had been introduced in the Origin and was greatly expanded in the Descent. Sexual selection was based on the observation that some characteristics did not directly affect an individual’s own survival, but did improve its chances of reproducing by helping it to attract a suitable mate 0r ward off rivals. Such animal features as colorful ornamentation, spurs, or antlers, or the power of song, appeared to evolve as a result of the direct competition of males for the available females. Darwin referred to this process as sexual selection, which he considered to be different from natural selection because the organism’s survival did not hinge on the possession of the trait. Wallace disagreed with Darwin that sexual selection was different in kind from natural selection, for both types of selection involved competition within the population and both affected reproductive success. The fact that Darwin thought of sexual selection as different indicates that he interpreted natural selection as meaning mainly differential survival, Whereas sexual selection meant only differential reproduction. His discussion of sexual selection reflected his awareness that success in leaving progeny was a crucial part of the struggle for existence.

Sexual selection implied that females had the power to discriminate among males and to select the more desirable ones for mating. As we have seen, Wallace used this idea in arguing for a socialist utopia, although otherwise he disagreed with many of Darwin’s specific uses of sexual selection to explain adaptations. Darwin himself used sexual selection to explain differences between the races of man and the differences between the sexes, for instance the perceived greater tenderness of women and the greater inven


SHARON KINGSLAND 183
tiveness and aggressiveness of men. As with his explanation of the moral sense, these adaptations were considered to be based on innate differences; they were parts of human nature. They projected onto nature Darwin’s image of Victorian men and women; the intellectual differences between the sexes were not seen to be products of differential learning or opportunity, but to be biologically evolved and adaptive differences.\footnote{differences are not due to environment, but to biological evolution!}

Darwin was mainly interested in past evolution, but in the Descent he allowed himself a brief speculation about the future. He noted with regret that the struggle for existence did not always lead to progress, for the reckless often reproduced faster than the virtuous.\footnote{Darwin's evolution is not tied, necessarily, with progress as Spencer's is..} The Scottish moralist Greg, as noted above, had already developed the argument that selection could have a deleterious effect on society, and Darwin quoted Greg's example in the Descent: "The careless, squalid, unaspiring Irishman multiplies like rabbits: the frugal, foreseeing self-respecting, ambitious Scot, stern in his morality, spiritual in his faith, sagacious and disciplined in his intelligence, passes his best years in struggle and in celibacy, marries late, and leaves few behind him” (Darwin, 1896, p. 138). In such a society, the “less favoured” race would prevail.

But Darwin did not despair. He believed that the checks to population increase, such as disease or famine, operated more harshly on the inferior classes of society, so that in the long run their growth would be checked more than the growth of the middle class. Moreover, he felt that continued struggle was a prerequisite for future advance, for otherwise men would sink into indolence. “Hence our natural rate of increase, though leading to many and obvious evils, must not be greatly diminished by any means,” he concluded (ibid., p. 618). Malthus, one might imagine, would have nodded vigorous agreement. But Darwin added that natural selection was not the sole means toward progress, even though it had produced the basic social instincts from which civilization grew. Learning and reasoning were even more important for the development of man's moral nature and would continue to be important for man’s future progress, far outweighing the struggle for existence.\footnote{how is it that learning and reasoning could be so important for Darwin's framework? is this a concession that man has gotten outside of the struggle for existence?}

This verdict on human progress was echoed by Thomas Henry Huxley, who in two essays on evolution and ethics written in the 18905 developed the idea that the cosmic process (which was governed by natural selection) and human civilization were destined to oppose each other (Huxley, 1894). Like Darwin, he saw that the moral sense was built on our natural capacity to feel sympathy. Although humans had been at one time subject to natural selection, their evolution from the primitive feeling of sympathy to the fully developed moral sense entailed a repudiation of the struggle for existence—— of man’s aggressive and competitive nature—in favor of cooperation. The gradual strengthening of the social bond arrested the struggle and put an end to the kind of evolution that occurred in nature.

Cultural advance happened only under the protection of the artificial environment of society, just as the beauty of the garden compared with raw


184 EVOLUTION AND HUMAN PROGRESS
nature was the product of a protected, cultivated enclave. But this Garden of Eden had its serpent, in the form of the Malthusian law of population increase. Huxley rejected the idea of planned selection Within society as impractical; but, if population increase could not be limited, and if selection could not be controlled, then the perfectibility of society was always limited. Huxley and Darwin agreed that the Malthusian law could not be evaded, and like Malthus both spurned utopian visions, while allowing for some progress in the future. Huxley, however, went further than Darwin in depicting the struggle for existence as fundamentally opposed to civilized progress. His target was not Darwin but Spencer, in particular What he saw as Spencer’s "fanatical individualism,” Which tried to apply the analogy of cosmic nature to society (Richards, 1987, p. 316).\footnote{anti-spencer's individualism...}

All of these discussions of human progress and natural selection were inevitably bound to arguments about moral advancement.\footnote{moral advancement} Darwin and his contemporaries could not avoid the habit of equating human fitness With advanced intellectual ability and moral superiority. The questions being debated were Whether natural selection had had any role in creating these qualities, and Whether it could account for further progress in this direction. It might seem as though these evolutionists muddled the issue by not adopting a value-free definition of fitness as survival and reproductive success. Were they not simply projecting onto nature their ideas of the superior moral qual— ities of the middle-class Victorian by arguing that these qualities were innate? They were indeed defining human nature by looking in a mirror, yet as Richards (1987, p. 175) points out, their discussions show them to have been aware that survival was merely a criterion of fitness, not a definition of fitness in itself. It was important to recognize that there should be a criterion of fitness independent of survival, for otherwise one fell into a circular definition of natural selection, namely: "natural selection means that organisms that cannot live, die." To avoid the tautology, one must be able to judge What traits give the "fit" individuals an advantage in the struggle for life, apart from their mere survival. For the Victorians, these traits in humans included a welldeveloped moral sense.

Yet the problem of defining fitness was an important one, and it became even more so When social engineers in the twentieth century started to argue that human breeding should be controlled in order to perfect society. The danger in this interventionist approach to progress was that any form of social deviance could be targeted as "unfit." The engineering mentality, satirized by Aldous Huxley in Brave New World, which did not greatly exaggerate the ideas of the 19205 (Haldane, 1924), was a dominant ideology by the turn of the century. In combination with the ideals of Social Darwinism and with advances in the science of genetics, it produced a scientific and political movement devoted to human breeding. The movement was spearheaded by Darwin's cousin Francis Galton.


SHARON KINGSLAND 185
Engineering human progress
Darwin was aware of the possibility that natural selection might have a deleterious effect in modern society, but he believed that in the long run selection had the power to limit reproduction among the criminal classes and among those with weak constitutions. Others were not so optimistic; they felt that harmful though the rapid reproduction of the lower classes might be in any culture, it would be suicidal in a democratic society. Francis Galton responded to this perceived threat to British society With the suggestion that progress required scientific control over human reproduction. He imagined a perfect society in which human breeding would be taken as seriously as the breeding of domestic animals, a society in which the ideal of “race improvement” would become the basis for a new ethics that would supplant Christianity. He conceived the idea for a program of selective breeding in 1865 and in 1883 coined the word “eugenics” to describe it (Kevles, 1985). Galton dismissed the common idea, argued by Spencer, of an inverse relation between intelligence and fertility (Cowan, 1977). Instead, he believed, the "fittest" members of society could be encouraged to marry and reproduce if given the right incentives.

Galton’s entire program rested on his conviction that intelligence and other mental qualities were determined by heredity, that nature was supreme over nurture\footnote{nature vs nurture debate...}. This conclusion was opposed to the scientific orthodoxy of the 18605; nor did Galton have the scientific evidence to back up his claims at first. Most scientists and educators asserted that environmental effects were decisive in shaping character and intelligence. To this belief they added the idea that some characteristics acquired during a person’s lifetime, in response to the environment, could become hereditary. Darwin himself believed that the continued effects of habit could form the basis for instincts that became hereditary. Spencer’s theory of evolution was even more completely based on the Lamarckian mechanism of adjustment between individual and environment.\footnote{epigenetics?} Proposals for social reform in the environmentalist tradition naturally stressed improvement of the environment. Galton’s theory, in contrast, denied the possibility of the inheritance of acquired characteristics. Reform could only be achieved by changing human nature itself through selection of the fittest.

After 186 5 Galton continued to promote his ideas assiduously in private discussions. Publicly he set about creating a scientific foundation for his political program. His research took the form of statistical studies of human variation, buttressed by some experimental and mathematical work on the mechanisms of inheritance. Galton’s interest in statistics was a near-obsession, one that paid off at a time when few biologists had any mathematical training and when the science of heredity was in a state of confusion arising from a mountain of observations, collected over decades of breeding experi~


186 EVOLUTION AND HUMAN PROGRESS
ments, but unordered by any coherent theory. The mathematical method oneered by Galton succeeded in bringing some order to the subject, chiefly by allowing for more precise definition of such concepts as “heredity,” "variation," and “reversion to type” (the tendency of individuals to resemble the average type of the population). Darwin himself drew on Galton’s findings to show that many mental traits were indeed inherited. In short, his research helped to turn the study of heredity into “positivist science,” where positivism meant the ideal of a science of exact measurement and mathematical law, following the trends in the physical sciences in the nineteenth century. But all of these achievements, Which secured Galton a place in history as a foun— der of biometry and population genetics, sprang directly from his enthusiasm for eugenics and his need to provide eugenics With a scientific basis.

The founder of eugenics had no children, but he did live to see his intellectual child flourish in the first decade of the twentieth century. He publicly reintroduced the concept of eugenics in 1901, this time to enthusiastic acclaim. By the turn of the century the hereditarian outlook and its implications for social reform had become the new orthodoxy among biologists as well as social scientists (Cravens, 1978). In this new climate Galton’s vision of a eugenic society seized the public imagination. In 1904 a Eugenics Record Office was established in London; in 1906 it was renamed the Francis Galton Eugenics Laboratory. Eugenics was forming the basis for movements of social reform that differed in their political affiliations but made a common appeal to the scientific principles of heredity. In Europe and America parallel eugenics movements began to have political impact.

The history of the eugenics movement, and the dominance of hereditarian attitudes in the biological and social sciences, have been described in recent books by Cravens (1978), Kevles (1985), Ludmerer (1972), and Weiss (1987). Leaders of the eugenics movement tended to be middle-class professionals—biologists and social scientists, clerics, professors, physicians, and politicians—and the movement reflected the biases of that class. Belief in the superiority of the Caucasian race was widespread, although there was no agreement as to the superiority of the Nordic type Within that race. The intelligence tests introduced in the early twentieth century as educational diagnostic tools quickly became one of the "scientific" weapons of those who argued for the inherent inferiority of certain populations (Gould, 1981). The heritability of several kinds of defects, as well as of alcoholism, certain diseases such as tuberculosis, and various sorts of moral degeneracy, formed a large part of the eugenics literature. In the United States, geneticists embraced eugenics along With the new Mendelism and set about collecting data on the hereditary traits of various populations. Much of the data was stored at the Eugenics Record Office at Cold Spring Harbor, New York, set up in 1910 to provide "a sort of inventory of the blood of the community" (Kevles, 1985, p. 55).

SHARON KINGSLAND 187
Debates about prostitution, sex education, and birth control were tied to eugenic discussions. Some eugenicists opposed birth control, on the grounds that it would be used only by middle-class women, While birth control advocates such as Margaret Sanger tried to forge an alliance with the eugenics movement as a way to gain credibility for the birth control movement (Borell, 1987; Reed, 1984).\footnote{birth control and eugenics... of course they are connected...} Sanger also solicited the support of biologists to organize the first World Population Conference, held in Geneva in 1927, which was intended to stimulate interest in population issues. Although the conference tried to uphold a dispassionate scientific image, avoiding controversial questions involving birth control, Sanger did hope to increase scientific support specifically for contraceptive research (Borell, 1987).

Advocates of eugenics came from the educated middle class, but they represented no single political outlook. Just as Darwinism had been bent to serve many political views, both conservative and liberal (Jones, 1980), so the eugenics ideal found supporters from the entire political spectrum. What eugenicists had in common was a definition of “fitness” in terms of social and cultural achievement, reflecting the values of their class. The same was true of the racial hygiene movement in Germany, which prior to Hitler’s rise to power was no different from many other eugenics movements in its aims and political diversity (Weiss, 1987). The main areas of legislation influenced by the eugenics movement concerned the passage of restrictive immigration laws and sterilization laws applied to poor people and the institutionalized population. In Britain the move to legalize voluntary sterilization failed, but in the United States several states passed compulsory sterilization laws, though not all were enforced (Kevles, 1985; Ludmerer, 1972).

If Galton had only conceived and promoted the idea of a eugenic utopia, he would merit a small place in history as a contributor to a relatively short-lived reform movement. His significance extends beyond this role because he saw that eugenics needed a solid scientific basis if its program of social engineering was to be accepted. Many of the leaders of eugenics were motivated to turn the study of heredity, population growth, and evolution into an experimental and mathematical science, whose goals were the discovery of scientific laws and whose methods were seen to be as objective as any part of human inquiry could be. When some biologists criticized the eugenics movement in the 192 Os, it was not because they disagreed with the principle of planned breeding per se, but because the movement had been wrested from the hands of scientists by laymen Whose ignorant pronouncements were destroying the movement’s credibility Eugenics was not just a call to political action; it was also a springboard to a great deal of research on human heredity.
The connection between eugenics and the positivist ideology in science is best seen in the work of Karl Pearson, Galton’s protege and successor as


SHARON KINGSLAND 191
ogls
colour prejudice, anti-feminismMand obstruction to educational progress” (p. 209). Too Wicists cm make simple-minded
POVGItY, wdE'controlled by the genes.

ot only did new genetic research indicate that gene interactions were far mortxomplex than those assumed by the early Mendelians, but the social sciences We emphasizing in the 1930s the importance of the environment as a determirwlt of human behavior. The opposition of biologists t0 eugenics was not so the principle of planned breeding or to the idea that truly deleterious genesfliould be eliminated from the population, agto the fact that the movement Miibited none of the caution associated M application of the scientific metho‘MJennings, 1930). Pearl, for instgpfe, argued that it was time to replace with a more scientific appjdach aimed at human genetics. Partly With that gm in mind, he helped )(found the International

Population Conference in 1927. was meant to act as a coordinating body for population research condm in a “strictly scientific spirit" (Pearl, 1928). Similarly, the eugenic that Mderlay contraceptive research was eroded as scientists recognifl the limitations of using biology to justify political decisions. As F. Al Crew, a geneticistfl the University of Edinburgh, observed, "In the pi the biologist has justifieNeudalism, Manchester Liberalism, socialisp’and every other type of sociaNganization and political programme” reference to selected biological phenomena” (quoted in Borell, 1W7, p. 84).

Although eugenics texts continued to spread the gospel throughout the 1930s, the movement was thoroughly discredited by the 19405 as a result of Hitler’s use of eugenic laws to enforce mass sterilizations and, starting in 1939, euthanasia. The term "eugenics" disappeared by the 19405 and gave way to scientific research into human genetics, Which culminated finally in programs of genetic testing of human populations in an attempt to eliminate certain hereditary defects (Kevles, 1985). The naive enthusiasm for the eugenics movement as a panacea for all social ills had disappeared, but the ideal of engineering human reproduction continued to guide society. The questions of What criteria to use to judge the “fitness” of a human being and how to respond to the presence of “unfit” individuals are still problematic.

Sociobiology and progress
While the eugenics movement revealed the dangers of applying biology to politics and showed the errors of naive genetic determinism, it by no means


192 EVOLUTION AND HUMAN PROGRESS
dispelled the belief that human social behavior was a legitimate subject for biological analysis.\footnote{is this positivism? the belief that science can claim something true? tell us something about truth?} Darwin’s theory of natural selection, after a period of eclipse around the turn of the century, gained support from the 19205 on, with advances in the science of population genetics and the final demise of the Lamarckian theory of the inheritance of acquired characteristics. Interest in the biological study of social behavior increased especially after World War I and culminated in the creation of a new field called sociobiology, inaugurated by the publication in 1975 of Sociobiology: The New Synthesis, by the Harvard zoologist Edward O. Wilson. Wilson defined sociobiology as "the systematic study of the biological basis of all forms of social behavior, including sexual and parental behavior, in all kinds of organisms, including man" (Wilson, 1978, p. 10). The new field synthesized several decades of research in the fields of population genetics, population ecology, and ethology (or, in the case of human studies, anthropology). The characteristic perspective of sociobiology was evolutionary: social behaviors are seen as adaptive traits molded by natural selection. Sociobiology was advanced as a logical extension of Darwin’s theory, differing from past research on behavior in that it was methodologically more sound, relying not on anecdote and analogy but on rigorous testing of hypotheses.

The central precept of sociobiology was that the evolution of social behavior in animals could be understood through an understanding of demography and of the genetic structure of populations (Wilson, 1980).\footnote{central precept of sociobiolgy...} The goal of the theory ultimately was to predict the features of social organization from a knowledge of these population parameters and of the way the genes set constraints on behavior. In the 1950s and 19605, evolutionary biologists had come to understand that the demographic parameters of animal populations, for example the intrinsic rate of increase of a population, were determined by the genetic composition of the population. Therefore they were subject to evolution by natural selection. Thus one could speak of species as having evolved reproductive strategies to suit different environments. Selection in unstable environments, for instance, might favor rapid rates of population increase, Whereas selection in a stable environment would favor a different reproductive strategy (MacArthur and Wilson, 1967). The demographic features of a population, such as survivorship or mortality schedules, were thought to represent close to the optimal (or most productive) schedules that members of the population could achieve in the environment in which the species lived (Wilson, 1975). The significance for sociobiology of the idea that reproductive strategies had evolved lay in the connection between a species’ method of reproduction and its social evolution, for some reproductive strategies entailed extended parental care and fairly complex social behavior.

The use of natural selection to explain social, and especially altruistic, behavior required extending the concept of fitness to include not just the individual but also groups of related individuals Within the population. Dar



\end{document}
