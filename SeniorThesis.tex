%%%%%%%%%%%%%%%%%%%%%%%%%%%%%%%%%%%%%%%%%%%%%%%%%%%%%%%%%%%%%%%%%%%%%%%%%%%%%%%%%%
%%%%  PLS Senior Thesis  %%%%%%%%%%%%%%%%%%%
%%%%%%%%%%%%%%%%%%%%%%%%%%%%%%%%%%%%%%%%%%%%%%%%%%%%%%%%%%%%%%%%%%%%%%%%%%%%%%%%%%
%~ There are a few main files to this:
%~ the chapters (chap1.tex and chap2.tex), the acknowledgements and the style guides (.sty) files. Just make sure to keep everything in the same folder.

%%%%%%%%%%%%%%%%%%%%%%%%%%%%%%%%%%%%%%%%%%%%%%%%%%%%%%%%%%%%%%%%%%%%%%%%%%%%%%%%%%
%%%%%%%%%%%%%%%%         all the preamble material            %%%%%%%%%%%%%%%%%%%%
%%%%%%%%%%%%%%%%%%%%%%%%%%%%%%%%%%%%%%%%%%%%%%%%%%%%%%%%%%%%%%%%%%%%%%%%%%%%%%%%%%

\documentclass[12pt]{report}
      \usepackage{utathesis}
      \usepackage[top=1.5in, bottom=1.50in, left=1.50in, right=1.5in]{geometry}
      \usepackage{inputenc}
      \usepackage[hidelinks]{hyperref}
      %\usepackage{fontspec}
      %\usepackage{fancyhdr}
      % \fontspec[Ligatures={TeX},BoldFont = {BaskervilleBookBQ-Medium}, ItalicFont = {BaskervilleBookBQ-Italic}]{BaskervilleBookBQ-Regular}
      % \setmainfont[Ligatures=TeX]{Goudy Bookletter 1911}
      % \defaultfontfeatures{Mapping=tex-text}
		%\setmainfont[Ligatures=TeX,
		%UprightFont = BaskervBookExpBQ-RegularOsF,
		%BoldFont = BaskervBookExpBQ-MediumOsF,
		%ItalicFont=BaskervBookExpBQ-ItalicOsF,
		%BoldItalicFont=BaskervBookExpBQ-MediumItalicOsF
		%]{BaskervBookExpBQ-Regular}
		


%\setmainfont{Baskerville No.2}

      
%%%%%%%%%%%%%%%%%%%%%%%%%%%%%%%%%%%%%%%%%%%%%%%%%%%%%%%%%%%%%%%%%%%%%%%%%%%%%%%%%%
%%%%%%%%%%%%%%%%%%%%   load any packages which are needed   %%%%%%%%%%%%%%%%%%%%%%
%%%%%%%%%%%%%%%%%%%%%%%%%%%%%%%%%%%%%%%%%%%%%%%%%%%%%%%%%%%%%%%%%%%%%%%%%%%%%%%%%%
%     \usepackage{latexsym} % to get LASY symbols
%     \usepackage{graphicx} % to insert PostScript figures
%     \usepackage{rotating} % for sideways tables/figures
%%%%%%%%%%%%%%%%%%%%%%%%%%%%%%%%%%%%%%%%%%%%%%%%%%%%%%%%%%%%%%%%%%%%%%%%%%%%%%%%%%
%%%%%%%%%%%%%%%%         all the preamble material            %%%%%%%%%%%%%%%%%%%%
%%%%%%%%%%%%%%%%%%%%%%%%%%%%%%%%%%%%%%%%%%%%%%%%%%%%%%%%%%%%%%%%%%%%%%%%%%%%%%%%%%
      \usepackage{graphicx}
      \usepackage{amsmath}
      % Define the Hanging Indent
      % These lines define the hanging indent! NEAT!
\makeatletter
% \renewcommand\@biblabel[1]{#1} % No brackets for the references
\def\@biblabel#1{}
\renewenvironment{thebibliography}[1]
     {\section*{\refname}%
      \@mkboth{\MakeUppercase\refname}{\MakeUppercase\refname}%
      \list{\@biblabel{\@arabic\c@enumiv}}%
           {\settowidth\labelwidth{\@biblabel{#1}}%
            \leftmargin\labelwidth
            \advance\leftmargin20pt% change 20 pt according to your needs
            \advance\leftmargin\labelsep
            \setlength\itemindent{-20pt}% change using the inverse of the length used before
            \@openbib@code
            \usecounter{enumiv}%
            \let\p@enumiv\@empty
            \renewcommand\theenumiv{\@arabic\c@enumiv}}%
      \sloppy
      \clubpenalty4000
      \@clubpenalty \clubpenalty
      \widowpenalty4000%
      \sfcode`\.\@m}
     {\def\@noitemerr
       {\@latex@warning{Empty `thebibliography' environment}}%
      \endlist}
\renewcommand\newblock{\hskip .11em\@plus.33em\@minus.07em}
\makeatother

\usepackage[compact]{titlesec}  
\titlespacing{\section}{0pt}{0pt}{0pt}
\usepackage{setspace}
\doublespacing
% End Definition of Hanging Indent
      
      
    \begin{document}
    \input{psfig.sty}
    \urlstyle{same}
%     SAMPLE FRONT MATTER:
       \graduationmonth{}
       \graduationyear{}
       \defensedate{}
       \author{Gabriel Griggs}
       \committee{}{}{}{}{}
       \title{``Active Love'' in the \emph{Brothers Karamazov}: \\ A Response to Ivan Karamazov's Problem of Suffering}

%%%%%%%%%%%%%%%%%%%%%%%%%%%%%%%%%%%%%%%%%%%%%%%%%%%%%%%%%%%%%%%%
%%%%%%%%%%%%%%  title page  %%%%%%%%%%%%%%%%%%%%%%%%%%%%%%%%
%%%%%%%%%%%%%%%%%%%%%%%%%%%%%%%%%%%%%%%%%%%%%%%%%%%%%%%%%%%%%%%%

          \titlepage


\begin{acknowledgements}

     \input acknowledge.tex

\end{acknowledgements}

%~ \input chap1.tex      % file containing Chapter 2 contents
%~ \input chap2.tex      % file containing Chapter 2 contents
\input ThesisTextBody_Revisions_FreshStart.tex
% Bibliography
\pagebreak
\begingroup
\renewcommand{\section}[2]{}	% in article, this becomes reference, so we suppress the normal \section\refname
\centerline{\textbf{Bibliography}} 
\begin{thebibliography}{9} 
\bibitem{dostoyevsky}
	Dostoevsky, Fyodor. Translated by Richard Pevear, and Larissa Volokhonsky. \emph{The Brothers Karamazov: A Novel in Four Parts with Epilogue.} New York: Farrar, Straus and Giroux, 2002.
\bibitem{jens}
	Jens, Walter, and Hans Küng. ``Religion in the Controversy Over the End of Religion.'' In \emph{Literature and Religion: Pascal, Gryphius, Lessing, Hölderlin, Novalis, Kierkegaard, Dostoyevsky, Kafka}. New York: Paragon House, 1991. 227-242.
\bibitem{may}
	Edited by May, Herbert G., and Bruce Manning Metzger. ``The Book of Job.'' In The \emph{New Oxford Annotated Bible with the Apocrypha: Revised standard version, containing the second edition of the New Testament and an expanded edition of the Apocrypha.} New York: Oxford University Press, 1977. 613-655.
\bibitem{morson}
	Morson, Gary. ``The God of Onions''. In \emph{The Brothers Karamazov}, trans. Constance Garnett. New York: Random House, 1950.
\bibitem{ramsey}
	Ramsey, O.P., Boniface, and Pope Benedict XVI. \emph{In the Beginning: A Catholic Understanding of the Story of Creation and the Fall.} Grand Rapids, Mich.: W.B. Eerdmans Pub. Co., 1995.
\bibitem{rosen}
	Rosen, Nathan. ``Style and Structure in \emph{The Brothers Karamazov}''. In \emph{The Brothers Karamazov}, trans. Constance Garnett. New York: Random House, 1950.
\bibitem{schifferdecker}
	Schifferdecker, Kathryn. \emph{Out of the Whirlwind: Creation Theology in the Book of Job}. Cambridge, Mass.: Harvard Theological Studies, Harvard Divinity School, 2008.
\bibitem{williams}
	Williams, Rowan. \emph{Dostoevsky: Language, Faith, and Fiction}. London: Continuum, 2009.
\bibitem{benedict}
	XVI, Pope Benedict. \emph{Introduction to Christianity}. San Francisco: Ignatius Press, 2004.
\end{thebibliography}

\endgroup


\end{document}
